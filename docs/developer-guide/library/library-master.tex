% Library Master File

%%%%%%%%%%%%%%%%%%%%%%%%%%%%%%%
\chapter{Inside the Library: LibUtilities}

In this chapter, we walk the reader through the different components of the LibUtilities Directory.
We have ordered them in alphabetical order by directory name, not by level of importance or
relevance to the code.  Since all of these items are considered foundational to Nektar++, they
should all be considered equally important and relevant.   Along the same lines -- since all of
these areas of the code represent the deepest members of the code hierarchy, these
items should rarely be modified. 
%
\import{library/LibUtilities/}{basicconst.tex}
%
\import{library/LibUtilities/}{basicutils.tex}
%
\import{library/LibUtilities/}{communication.tex}
%
\import{library/LibUtilities/}{fft.tex}
%
\import{library/LibUtilities/}{foundations.tex}
%
\import{library/LibUtilities/}{nek-interpreter.tex}
%
\import{library/LibUtilities/}{kernel.tex}
%
\import{library/LibUtilities/}{linearalgebra.tex}
%
\import{library/LibUtilities/}{memory.tex}
%
\import{library/LibUtilities/}{polylib.tex}
%
\import{library/LibUtilities/}{timeintegration.tex}

%%%%%%%%%%%%%%%%%%%%%%%%%%%%%%%
\chapter{Inside the Library: StdRegions}
\label{chap:stdregions}

In this chapter, we walk the reader through the different components of the StdRegions Directory.
We begin with a discussion of the mathematical fundamentals, for which we use the book
by Karniadakis and Sherwin \cite{KaSh05} as our principle reference.  We then provide
the reader with an overview of the primary data structures introduced within the
StdRegions Directory (often done through C++ objects), and then present the major 
algorithms -- expressed as either object methods or functions -- employed over these data structures.  

\import{library/StdRegions/}{stdreg-fundamentals.tex}
%
\import{library/StdRegions/}{stdreg-datastructures.tex}
%
\import{library/StdRegions/}{stdreg-algorithms.tex}


%%%%%%%%%%%%%%%%%%%%%%%%%%%%%%%
\chapter{Inside the Library: SpatialDomains}
\label{chap:spatialdomains}


In this chapter, we walk the reader through the different components
of the SpatialDomains Directory.  We begin with a discussion of the
mathematical fundamentals, for which we use the book by Karniadakis
and Sherwin \cite{KaSh05} as our principle reference.  We then provide
the reader with an overview of the primary data structures introduced
within the SpatialDomains Directory (often done through C++ objects),
and then present the major algorithms -- expressed as either object
methods or functions -- employed over these data structures.


The SpatialDomains Directory and its corresponding class definitions serve two principal purposes:
\begin{enumerate}
\item To hold the elemental geometric information (i.e. vertex information, curve information and reference-to-world mapping information); and
\item To facility reading in and writing out geometry-related information.
\end{enumerate}

When designing Nektar++, developing a class hierarchy for StdRegions
(those fundamental domains over which we define integration and
differentiation) and LocalRegions (i.e. elements in world-space) was
fairly straightforward following \cite{KaSh05}.  For instance, a
triangle in world-space {\em is-a} standard triangle.  The first
question that arose was where to store geometric information, as
information within the LocalRegions element or as information
encapsulated from the element so that multiple Expansions could all
point to the same geometric information.  The decision we made was to
store geometric information -- that is, the vertex information in
world-space that defines an element and the edge and face curvature
information -- in its own data structure that could be shared by
multiple Expansions (functions) over the same domain (element) in
world-space.  Hence SpatialDomains started as the directory containing
Geometry and GeomFactors class definitions to meet the first item
listed above.  A LocalRegion {\em is-a} StdRegion and {\em has-a}
SpatialDomain (i.e. Geometry and GeomFactors).

We then realized that in order to jump-start the process of
constructing elements and combining them together into MultiRegions
(collections of elements that represent a (sub)-domain of interest),
we needed devise a light-weight data structure into which we could
load geometric information from our geometry file and from which we
could then construct Expansions (with their mappings, etc.).  The
light-weight data structure we devised was MeshGraph, and it was meant
to meet the second item listed above.


\import{library/SpatialDomains/}{spdomains-fundamentals.tex}
%
\import{library/SpatialDomains/}{spdomains-datastructures.tex}
%
\import{library/SpatialDomains/}{spdomains-algorithms.tex}


%%%%%%%%%%%%%%%%%%%%%%%%%%%%%%%
\chapter{Inside the Library: LocalRegions}
\label{chap:localregions}


In this chapter, we walk the reader through the different components
of the LocalRegions Directory.  We begin with a discussion of the
mathematical fundamentals, for which we use the book by Karniadakis
and Sherwin \cite{KaSh05} as our principle reference.  We then provide
the reader with an overview of the primary data structures introduced
within the LocalRegions Directory (often done through C++ objects),
and then present the major algorithms -- expressed as either object
methods or functions -- employed over these data structures.

\import{library/LocalRegions/}{localreg-fundamentals.tex}
%
\import{library/LocalRegions/}{localreg-datastructures.tex}
%
\import{library/LocalRegions/}{localreg-algorithms.tex}


%%%%%%%%%%%%%%%%%%%%%%%%%%%%%%%
\chapter{Inside the Library: Collections}

In this chapter, we walk the reader through the different components
of the Collections Directory.  We begin with a discussion of the
mathematical fundamentals, for which we use the book by Karniadakis
and Sherwin \cite{KaSh05} as our principle reference.  We then provide
the reader with an overview of the primary data structures introduced
within the Collections Directory (often done through C++ objects), and
then present the major algorithms -- expressed as either object
methods or functions -- employed over these data structures.

\import{library/Collections/}{collections-fundamentals.tex}
%
\import{library/Collections/}{collections-datastructures.tex}
%
\import{library/Collections/}{collections-algorithms.tex}


%%%%%%%%%%%%%%%%%%%%%%%%%%%%%%%
\chapter{Inside the Library: MultiRegions}
\label{chap:multiregions}


In this chapter, we walk the reader through the different components
of the MultiRegions Directory.  We begin with a discussion of the
mathematical fundamentals, for which we use the book by Karniadakis
and Sherwin \cite{KaSh05} as our principle reference.  We then provide
the reader with an overview of the primary data structures introduced
within the MultiRegions Directory (often done through C++ objects),
and then present the major algorithms -- expressed as either object
methods or functions -- employed over these data structures.

\import{library/MultiRegions/}{multireg-fundamentals.tex}
%
\import{library/MultiRegions/}{multireg-datastructures.tex}
%
\import{library/MultiRegions/}{multireg-algorithms.tex}
%
\import{library/MultiRegions/}{multireg-preconditioners.tex}



%%%%%%%%%%%%%%%%%%%%%%%%%%%%%%%
\chapter{Inside the Library: GlobalMapping}
\label{chapter:globalmapping} 

In this chapter, we walk the reader through the different components
of the GlobalMapping Directory.  We begin with a discussion of the
mathematical fundamentals, for which we use research article by
Cantwell et al. \cite{CantwellYKPS14} and the book \cite{Ar89} as our
principle references.  We then provide the reader with an overview of
the primary data structures introduced within the GlobalMapping
Directory (often done through C++ objects), and then present the major
algorithms -- expressed as either object methods or functions --
employed over these data structures.

\import{library/GlobalMapping/}{globalmapping-fundamentals.tex}
%
\import{library/GlobalMapping/}{globalmapping-datastructures.tex}
%
\import{library/GlobalMapping/}{globalmapping-algorithms.tex}


%%%%%%%%%%%%%%%%%%%%%%%%%%%%%%%
\chapter{Inside the Library: FieldUtils}

In this chapter, we walk the reader through the different components of the FieldUtils Directory.
We begin with a discussion of the mathematical fundamentals, for which we use the book
by Karniadakis and Sherwin \cite{KaSh05} as our principle reference.  We then provide
the reader with an overview of the primary data structures introduced within the
FieldUtils Directory (often done through C++ objects), and then present the major 
algorithms -- expressed as either object methods or functions -- employed over these data structures.  

\import{library/FieldUtils/}{fieldutils-fundamentals.tex}
%
\import{library/FieldUtils/}{fieldutils-datastructures.tex}
%
\import{library/FieldUtils/}{fieldutils-algorithms.tex}


%%%%%%%%%%%%%%%%%%%%%%%%%%%%%%%
\chapter{Inside the Library: SolverUtils}

In this chapter, we walk the reader through the different components of the SolverUtils Directory.
We begin with a discussion of the mathematical fundamentals, for which we use the book
by Karniadakis and Sherwin \cite{KaSh05} as our principle reference.  We then provide
the reader with an overview of the primary data structures introduced within the
SolverUtils Directory (often done through C++ objects), and then present the major 
algorithms -- expressed as either object methods or functions -- employed over these data structures.  

\import{library/SolverUtils/}{solverutils-fundamentals.tex}
%
\import{library/SolverUtils/}{solverutils-datastructures.tex}
%
\import{library/SolverUtils/}{solverutils-algorithms.tex}


