%
\section{Time Integration}

This directory consists of the following files:


\begin{center}
\begin{tabular}{|c | c | } \hline
TimeIntegrationScheme.h & TimeIntegrationScheme.cpp	\\ \hline 
TimeIntegrationWrapper.h & TimeIntegrationWrapper.cpp \\ \hline
\end{tabular}
\end{center}

The original time-stepping classes (found in TimeIntegrationScheme)
where implemented by Vos et al. \cite{Vostime} and a detailed
explanation of the mathematics and computer science concepts are
provided there.  After the original implementation, Cantwell et
al. extended Vos' original work by adding the time-stepping routines
into a factory pattern (found in TimeIntegrationWrapper).

\subsection{General Linear Methods}
General linear methods (GLMs) can be considered as the generalization of a broad
range of different numerical methods for ordinary differential equations.  They
were introduced by Butcher and they provide a unified formulation for
traditional methods such as the Runge-Kutta methods and the linear multi-step
methods such as Adams-Bashforth.  
From an implementation point of view, this means that all these
numerical methods can be abstracted in a similar way. As this allows a high
level of generality, it is chosen in {\nek} to cast all time integration
schemes in the framework of general linear methods.

For background information about general linear methods, please consult
\cite{Bu06}.

\subsection{Introduction}
The standard initial value problem can written in the form
\begin{align*}
\frac{d\boldsymbol{y}}{dt} = \boldsymbol{f}(t,\boldsymbol{y}),\quad
\boldsymbol{y}(t_0)=\boldsymbol{y}_0
\end{align*}
where $\boldsymbol{y}$ is a vector containing the variable (or an
array of array containing the variables).

In the formulation of general linear methods, it is more convenient to consider
the ODE in autonomous form, i.e.
\begin{align*}
\frac{d\boldsymbol{\hat{y}}}{dt}=\boldsymbol{\hat{f}}(\boldsymbol{\hat{y}}),\quad
\boldsymbol{\hat{y}}(t_0)= \boldsymbol{\hat{y}}_0.
\end{align*}

\subsection{Formulation}
Suppose the governing differential equation is given in autonomous form, the
$n^{th}$ step of the GLM comprising
\begin{itemize}
\item $r$ steps (as in a multi-step method)
\item $s$ stages (as in a Runge-Kutta method)
\end{itemize}
is formulated as:
\begin{align*}
\boldsymbol{Y}_i &= \Delta
t\sum_{j=0}^{s-1}a_{ij}\boldsymbol{F}_j+\sum_{j=0}^{r-1}u_{ij}\boldsymbol{\hat{y}}_{j}^{[n-1]},
\qquad i=0,1,\ldots,s-1 \\
\boldsymbol{\hat{y}}_{i}^{[n]}&=\Delta
t\sum_{j=0}^{s-1}b_{ij}\boldsymbol{F}_j+\sum_{j=0}^{r-1}v_{ij}
\boldsymbol{\hat{y}}_{j}^{[n-1]}, \qquad i=0,1,\ldots,r-1
\end{align*}
where $\boldsymbol{Y}_i$ are referred to as the stage values and
$\boldsymbol{F}_j$ as the stage derivatives. Both quantities are
related by the differential equation:
\begin{align*}
\boldsymbol{F}_i = \boldsymbol{\hat{f}}(\boldsymbol{Y}_i).
\end{align*}

The matrices $A=[a_{ij}]$, $U=[u_{ij}]$,
$B=[b_{ij}]$, $V=[v_{ij}]$ are characteristic of a
specific method. Each scheme can then be uniquely defined by a partitioned
$(s+r)\times(s+r)$ matrix
\begin{align*}
\left[
\begin{array}{cc}
  A & U\\
  B & V \end{array}\right]
\end{align*}


\subsection{Matrix notation}
Adopting the notation:
\begin{align*}
\boldsymbol{\hat{y}}^{[n-1]}= \left[\begin{array}{c}
\boldsymbol{\hat{y}}^{[n-1]}_0\\
\boldsymbol{\hat{y}}^{[n-1]}_1\\
\vdots\\
\boldsymbol{\hat{y}}^{[n-1]}_{r-1}
\end{array}\right],\quad \boldsymbol{\hat{y}}^{[n]}= \left[\begin{array}{c}
\boldsymbol{\hat{y}}^{[n]}_0\\
\boldsymbol{\hat{y}}^{[n]}_1\\
\vdots\\
\boldsymbol{\hat{y}}^{[n]}_{r-1}
\end{array}\right],\quad \boldsymbol{Y}= \left[\begin{array}{c}
\boldsymbol{Y}_0\\
\boldsymbol{Y}_1\\
\vdots\\
\boldsymbol{Y}_{s-1}
\end{array}\right],\quad \boldsymbol{F}= \left[\begin{array}{c}
\boldsymbol{F}_0\\
\boldsymbol{F}_1\\
\vdots\\
\boldsymbol{F}_{s-1}
\end{array}\right]\quad
\end{align*}
the general linear method can be written more compactly in the following form:
\begin{align*}
\left[ \begin{array}{c} \boldsymbol{Y}\\
\boldsymbol{\hat{y}}^{[n]}
\end{array}\right] = \left[ \begin{array}{cc} A\otimes I_N & U\otimes I_N \\
B\otimes I_N & V\otimes I_N \end{array}\right] \left[ \begin{array}{c} \Delta
t\boldsymbol{F}\\
\boldsymbol{\hat{y}}^{[n-1]}
\end{array}\right]
\end{align*}
where $I_N$ is the identity matrix of dimension $N\times N$.


\subsection{General Linear Methods in Nektar++}
Although the GLM is essentially presented for ODE's in its autonomous form, in
Nektar++ it will be used to solve ODE's formulated in non-autonomous form.
Given the ODE,
\begin{align*}
\frac{d\boldsymbol{y}}{dt}=\boldsymbol{f}(t,\boldsymbol{y}),\quad
\boldsymbol{y}(t_0)=\boldsymbol{y}_0
\end{align*}
a single step of GLM can then be evaluated in the following way:
\begin{itemize}
\item \textbf{input}

$\boldsymbol{y}^{[n-1]}$, \emph{i.e.} the $r$
    subvectors comprising $\boldsymbol{y}^{[n-1]}_i$ -
    $t^{[n-1]}$, \emph{i.e.} the equivalent of
    $\boldsymbol{y}^{[n-1]}$ for the time variable $t$.

\item \textbf{step 1}: The stage values $\boldsymbol{Y}_i$,
$\boldsymbol{T}_i$ and the stage derivatives
$\boldsymbol{F}_i$ are calculated through the relations:
\begin{enumerate}
\item $\boldsymbol{Y}_i = \Delta
    t\sum_{j=0}^{s-1}a_{ij}\boldsymbol{F}_j+\sum_{j=0}^{r-1}u_{ij}\boldsymbol{y}_{j}^{[n-1]},
    \qquad i=0,1,\ldots,s-1$
\item $T_i\ =\ \Delta
    t\sum_{j=0}^{s-1}a_{ij}+\sum_{j=0}^{r-1}u_{ij}t_{j}^{[n-1]}, \qquad
    i=0,1,\ldots,s-1$
\item $\boldsymbol{F}_i\ =\
    f(T_i,\boldsymbol{Y}_i), \qquad i=0,1,\ldots,s-1$
\end{enumerate}

\item \textbf{step 2}: The approximation at the new time level
$\boldsymbol{y}^{[n]}$ is calculated as a linear combination of the
stage derivatives $\boldsymbol{F}_i$ and the input vector
$\boldsymbol{y}^{[n-1]}$. In addition, the time vector
$t^{[n]}$ is also updated

\begin{enumerate}
\item $\boldsymbol{y}_{i}^{[n]} = \Delta
    t\sum_{j=0}^{s-1}b_{ij}\boldsymbol{F}_j+\sum_{j=0}^{r-1}v_{ij}\boldsymbol{y}_{j}^{[n-1]},
    \qquad i=0,1,\ldots,r-1$
\item $t_{i}^{[n]}\ =\ \Delta
    t\sum_{j=0}^{s-1}b_{ij}+\sum_{j=0}^{r-1}v_{ij}t_{j}^{[n-1]}, \qquad
    i=0,1,\ldots,r-1$
\end{enumerate}

\item \textbf{output}
\begin{enumerate}
\item $\boldsymbol{y}^{[n]}$, i.e. the $r$ subvectors
    comprising $\boldsymbol{y}^{[n]}_i$.
    $\boldsymbol{y}^{[n]}_0$ corresponds to the actual approximation
    at the new time level.
\item $t^{[n]}$ where $t^{[n]}_0$ is equal to the new time
    level $t+\Delta t$.
\end{enumerate}
\end{itemize}

For a detailed description of the formulation and a deeper insight of the
numerical method see \cite{Vostime}.


\subsection{Types of time Integration Schemes}
Nektar++ contains various classes and methods which implement the concept of
GLMs. This toolbox is capable of numerically solving the generalised ODE using a
broad range of different time-stepping methods. We distinguish the following
types of general linear methods:
\begin{itemize}
\item \textbf{Formally Explicit Methods}: These types of methods are
considered explicit from an ODE point of view. They are characterised by a lower
  triangular coefficient matrix $A$, ''i.e.'' $a_{ij} = 0$
  for $j\geq i$. To avoid confusion, we make a further distinction:
  \begin{itemize}
    \item \textbf{direct explicit method}: Only forward operators are required.
    \item \textbf{indirect explicit method}: The inverse operator is required.
  \end{itemize}
  \item \textbf{Diagonally Implicit Methods}: Compared to explicit methods,
  the coefficient matrix $A$ has now non-zero entries on the diagonal.
  This means that each stage value depend on the stage derivative at the same
  stage, requiring an implicit step. However, the calculation of the different
  stage values is still uncoupled. Best known are the DIRK schemes.
  \item \textbf{IMEX schemes}: These schemes support the concept of being able
  to split right hand forcing term into an explicit and implicit component. This is
  useful in advection diffusion type problems where the advection is handled
  explicitly and the diffusion is handled implicit.
  \item \textbf{Fully Implicit Methods Methods}: The coefficient matrix has a
  non-zero upper triangular part. The calculation of all stages values is fully coupled.
\end{itemize}

The aim in {\nek} is to fully support the first three types of GLMs.
Fully implicit methods are currently not implemented.

\subsection{Usage}
The goal of abstracting the concept of general linear methods is to provide
users with a single interface for time-stepping, independent of the chosen
method. The classes tree allows the user to numerically integrate ODE's using
high-order complex schemes, as if it was done using the Forward-Euler method.
Switching between time-stepping schemes is as easy as changing a parameter in an
input file.

In the already implemented solvers the time-integration schemes have been set up
according to the nature of the equations.
For example the incompressible Navier-Stokes equations solver allows the use of
three different Implicit-Explicit time-schemes if solving the equations with a
splitting-scheme.
This is because this kind of scheme has an explicit and an implicit operator
that combined solve the ODE's system.

Once aware of the problem's nature and implementation, the user can easily
switch between some (depending on the problem) of the following
time-integration schemes:

\begin{center}
\footnotesize
\begin{tabular}{p{4cm}p{10cm}}
\toprule
AdamsBashforthOrder1 & Adams-Bashforth Forward multi-step scheme of order 1\\
AdamsBashforthOrder2 & Adams-Bashforth Forward multi-step scheme of order 2\\
AdamsBashforthOrder3 & Adams-Bashforth Forward multi-step scheme of order 3\\
AdamsMoultonOrder1   & Adams-Moulton Forward multi-step scheme of order 1\\
AdamsMoultonOrder2   & Adams-Moulton Forward multi-step scheme of order 2\\
BDFImplicitOrder1    & BDF multi-step scheme of order 1 (implicit)\\
BDFImplicitOrder2    & BDF multi-step scheme of order 2 (implicit)\\
ClassicalRungeKutta4 & Runge-Kutta multi-stage scheme 4th order explicit (old name)\\
RungeKutta4 & Runge-Kutta multi-stage scheme 4th order explicit (new name for ClassicalRungeKutta4)\\
RungeKutta3\_SSP & Nonlinear SSP RungeKutta3 explicit\\
RungeKutta2\_ImprovedEuler & Improved RungeKutta2 explicit (old name meaning Heun's method)\\
RungeKutta2\_SSP & Nonlinear SSP RungeKutta2 explicit (surrogate for RungeKutta2\_ImprovedEuler)\\
ForwardEuler         & Forward-Euler scheme\\
BackwardEuler       & Backward Euler scheme\\
IMEXOrder1           & IMEX 1st order scheme using Euler Backwards Euler
                       Forwards \\
IMEXOrder2           & IMEX 2nd order scheme using Backward Different Formula \&
                       Extrapolation\\
IMEXOrder3           & IMEX 3rd order scheme using Backward Different Formula \&
                       Extrapolation\\
Midpoint             & Midpoint method (old name)\\
RungeKutta2     & Classical RungeKutta2 method (new name for Midpoint)\\
DIRKOrder2           & Diagonally Implicit Runge-Kutta scheme of order 2\\
DIRKOrder3           & Diagonally Implicit Runge-Kutta scheme of order 3\\
CNAB                 & Crank-Nicolson-Adams-Bashforth Order 2 (CNAB)\\
IMEXGear             & IMEX Gear Order 2\\
MCNAB                & Modified Crank-Nicolson-Adams-Bashforth Order 2 (MCNAB)\\
IMEXdirk\_1\_1\_1    & Forward-Backward Euler IMEX DIRK(1,1,1)\\
IMEXdirk\_1\_2\_1    & Forward-Backward Euler IMEX DIRK(1,2,1)\\
IMEXdirk\_1\_2\_2    & Implicit-Explicit Midpoint IMEX DIRK(1,2,2)\\
IMEXdirk\_2\_2\_2    & L-stable, two stage, second order IMEX DIRK(2,2,2)\\
IMEXdirk\_2\_3\_2    & L-stable, three stage, third order IMEX DIRK(3,4,3)\\
IMEXdirk\_2\_3\_3    & L-stable, two stage, third order IMEX DIRK(2,3,3)\\
IMEXdirk\_3\_4\_3    & L-stable, three stage, third order IMEX DIRK(3,4,3)\\
IMEXdirk\_4\_4\_3    & L-stable, four stage, third order IMEX DIRK(4,4,3)\\
\bottomrule
\end{tabular}
\end{center}

{\nek} input file for your problem will ask you just the string corresponding
the time-stepping scheme you want to use (between quotation marks in the
previous list), and few parameters to define your integration in time (time-step
and number of steps or final time). For example:\\
\begin{lstlisting}[style=XMLStyle]
<SOLVERINFO>
   <I PROPERTY="EQTYPE" VALUE="UnsteadyNavierStokes" />
   <I PROPERTY="SolverType" VALUE="VelocityCorrectionScheme" />
   <I PROPERTY="AdvectionForm" VALUE="Convective" /> 
   <I PROPERTY="Projection" VALUE="Galerkin" /> 
   <I PROPERTY="TimeIntegrationMethod" VALUE="IMEXOrder1" />
 </SOLVERINFO>
 
 <PARAMETERS>
   <P> TimeStep = 0.001     </P> 
   <P> NumSteps = 1000      </P> 
   <P> Kinvis = 1           </P>
</PARAMETERS>
\end{lstlisting}


\subsection{Implementation of a time-dependent problem}
In order to implement a new solver which takes advantage of the time-integration
class in {\nek}, two main ingredients are required:
\begin{itemize}
\item A main files in which the time-integration of you ODE's system is
initialized and performed.
\item A class representing the spatial discretization of your problem, which
reduces your system of PDE's to a system of ODE's.
\end{itemize}

Your pseudo-main file, where you actually loop over the time steps, will look
like
\begin{lstlisting}[style=C++Style]
NekDouble timestep    = 0.1; 
NekDouble time        = 0.0; 
int NumSteps          = 1000;

YourClass solver(Input);
    
LibUtilities::TimeIntegrationMethod  TIME_SCHEME;
LibUtilities::TimeIntegrationSchemeOperators ODE;

ODE.DefineOdeRhs(&YourClass::YourExplicitOperatorFunction,solver);
ODE.DefineProjection(&YourClass::YourProjectionFunction,solver);
ODE.DefineImplicitSolve(&YourClass::YourImplicitOperatorFunction,solver);
    
Array<OneD, LibUtilities::TimeIntegrationSchemeSharedPtr> IntScheme;
    
LibUtilities::TimeIntegrationSolutionSharedPtr ode_solution;
    
Array<OneD, Array<OneD, NekDouble> > U;

TIME_SCHEME = LibUtilities::eForwardEuler; 
int numMultiSteps=1; 
IntScheme = Array<OneD,LibUtilities::TimeIntegrationSchemeSharedPtr>(numMultiSteps); 
LibUtilities::TimeIntegrationSchemeKey IntKey(TIME_SCHEME); 
IntScheme[0] = LibUtilities::TimeIntegrationSchemeManager()[IntKey]; 
ode_solution = IntScheme[0]->InitializeScheme(timestep,U,time,ODE);

for(int n = 0; n < NumSteps; ++n) {
    U = IntScheme[0]->TimeIntegrate(timestep,ode_solution,ODE);
}
\end{lstlisting}

We can distinguish three different sections in the code above

\subsubsection{Definitions} 
\begin{lstlisting}[style=C++Style]
NekDouble timestep    = 0.1; 
NekDouble time        = 0.0;
int NumSteps          = 1000;

YourClass equation(Input);
    
LibUtilities::TimeIntegrationMethod  TIME_SCHEME;
LibUtilities::TimeIntegrationSchemeOperators ODE;

ODE.DefineOdeRhs(&YourClass::YourExplicitOperatorFunction,equation);
ODE.DefineProjection(&YourClass::YourProjectionFunction, equation);
ODE.DefineImplicitSolve(&YourClass::YourImplicitOperatorFunction, equation);
\end{lstlisting}
In this section you define the basic parameters (like time-step, initial time,
etc.) and the time-integration objects.
The operators are not all required, it depends on the nature of your problem and
on the type of time integration schemes you want to use. In this case, the
problem has been set up to work just with Forward-Euler, then for sure you will
not need the implicit operator.
An object named \texttt{equation} has been initialized, is an object of type
\texttt{YourClass}, where your spatial discretization and the functions which
actually represent your operators are implemented. An example of this class will
be shown later in this page.

\subsubsection{Initialisations}
\begin{lstlisting}[style=C++Style]
Array<OneD,LibUtilities::TimeIntegrationSchemeSharedPtr> IntScheme;
    
LibUtilities::TimeIntegrationSolutionSharedPtr ode_solution;
    
Array<OneD, Array<OneD, NekDouble> > U;

TIME_SCHEME = LibUtilities::eForwardEuler; 
int numMultiSteps=1; 
IntScheme = Array<OneD,LibUtilities::TimeIntegrationSchemeSharedPtr>(numMultiSteps); 
LibUtilities::TimeIntegrationSchemeKey IntKey(TIME_SCHEME); 
IntScheme[0] = LibUtilities::TimeIntegrationSchemeManager()[IntKey];
ode_solution = IntScheme[0]->InitializeScheme(timestep,U,time,ODE);
\end{lstlisting}
The second part consists in the scheme initialization. In this example we set up
just Forward-Euler, but we can set up more then one time-integration scheme and
quickly switch between them from the input file.
Forward-Euler does not require any other scheme for the start-up procedure. High
order multi-step schemes may need lower-order schemes for the start up.

\subsubsection{Integration}
\begin{lstlisting}[style=C++Style]
for(int n = 0; n < NumSteps; ++n) {
    U = IntScheme[0]->TimeIntegrate(timestep,ode_solution,ODE);
}
\end{lstlisting}
The last step is the typical time-loop, where you iterate in time to get
your new solution at each time-level.
The solution at time $t^{n+1}$ is stored into vector \texttt{U} (you need
to properly initialize this vector).
\texttt{U} is an Array of Arrays, where the first dimension corresponds to
the number of variables (eg. u,v,w) and the second dimension corresponds to the
variables size (e.g. the number of modes or the number of physical points).

The variable \texttt{ODE} is an object which contains the methods. A class
representing a PDE equation (or a system of equations) must have a series of
functions representing the implicit/explicit part of the method, which
represents the reduction of the PDE's to a system of ODE's.
The spatial discretization and the definition of this method should be
implemented in \texttt{YourClass}.
\texttt{\&YourClass::YourExplicitOperatorFunction} is a functor, i.e. a pointer
to a function where the method is implemented. \texttt{equation} is a pointer
to the object, i.e. the class, where the function/method is implemented.
Here a pseudo-example of the .h file of your hypothetical class representing the
set of equations.
The implementation of the functions is meant to be in the related .cpp file.

\begin{lstlisting}[style=C++Style]
class YourCalss
{ 
public:
    YourClass(INPUT);
    
    ~YourClass(void);

    void YourExplicitOperatorFunction(
        const Array<OneD, Array<OneD, NekDouble> > & inarray,
              Array<OneD, Array<OneD, NekDouble> > & outarray,
        const NekDouble time);

    void YourProjectionFunction(
        const Array<OneD, Array<OneD, NekDouble> > & inarray,
              Array<OneD, Array<OneD, NekDouble> > & outarray, const
              NekDouble time);

    void YourImplicitOperatorFunction(
        const Array<OneD, Array<OneD, NekDouble> > & inarray,
              Array<OneD, Array<OneD, NekDouble> > & outarray, const
              NekDouble time,
        const NekDouble lambda);
    
    void InternalMethod1(...);

    NekDouble internalvariale1;

protected:
    ...

private:
    ...
};
\end{lstlisting}

 
\subsection{Strongly imposed essential boundary conditions}
Dirichlet boundary conditions can be strongly imposed by lifting the known
Dirichlet solution.
This is equivalent to decompose the approximate solution $y$ into an
known part, $y^{\mathcal{D}}$, which satisfies the Dirichlet boundary
conditions, and an unknown part, $y^{\mathcal{H}}$, which is zero on
the Dirichlet boundaries, i.e.
\begin{align*}
y = y^{\mathcal{D}} + y^{\mathcal{H}}
\end{align*}

In a Finite Element discretisation, this corresponds to splitting the solution
vector of coefficients $\boldsymbol{y}$ into the known Dirichlet
degrees of freedom $\boldsymbol{y}^{\mathcal{D}}$ and the unknown
homogeneous degrees of freedom $\boldsymbol{y}^{\mathcal{H}}$.
Ordering the known coefficients first, this corresponds to:
\begin{align*}
\boldsymbol{y} = \left[ \begin{array}{c}
\boldsymbol{y}^{\mathcal{D}} \\
\boldsymbol{y}^{\mathcal{H}} \end{array} \right]
\end{align*}

The generalised formulation of the general linear method (i.e. the introduction
of a left hand side operator) allows for an easier treatment of these types of
boundary conditions. To better appreciate this, consider the equation for the
stage values for an explicit general linear method where both the left and right
hand side operator are linear operators, i.e. they can be represented by a
matrix.
\begin{align*}
\boldsymbol{M}\boldsymbol{Y}_i = \Delta
t\sum_{j=0}^{i-1}a_{ij}\boldsymbol{L}\boldsymbol{Y}_j+\sum_{j=0}^{r-1}u_{ij}\boldsymbol{y}_{j}^{[n-1]},
\qquad i=0,1,\ldots,s-1
\end{align*}

In case of a lifted known solution, this can be written as:
\begin{align*}
\left[ \begin{array}{cc}
\boldsymbol{M}^{\mathcal{DD}} & \boldsymbol{M}^{\mathcal{DH}} \\
\boldsymbol{M}^{\mathcal{HD}} & \boldsymbol{M}^{\mathcal{HH}} \end{array} \right]
\left[ \begin{array}{c}
\boldsymbol{Y}^{\mathcal{D}}_i \\
\boldsymbol{Y}^{\mathcal{H}}_i \end{array} \right]
= \Delta t\sum_{j=0}^{i-1}a_{ij}
\left[ \begin{array}{cc}
\boldsymbol{L}^{\mathcal{DD}} & \boldsymbol{L}^{\mathcal{DH}} \\
\boldsymbol{L}^{\mathcal{HD}} & \boldsymbol{L}^{\mathcal{HH}} \end{array} \right]
\left[ \begin{array}{c}
\boldsymbol{Y}^{\mathcal{D}}_j \\
\boldsymbol{Y}^{\mathcal{H}}_j \end{array} \right]
+\sum_{j=0}^{r-1}u_{ij}
\left[ \begin{array}{c}
\boldsymbol{y}^{\mathcal{D}[n-1]}_j \\
\boldsymbol{y}^{\mathcal{H}[n-1]}_j \end{array} \right],\\
\qquad i=0,1,\ldots,s-1
\end{align*}
 
In order to calculate the stage values correctly, the explicit operator
should now be implemented to do the following:
\begin{align*}
\left[ \begin{array}{c}
\boldsymbol{b}^{\mathcal{D}} \\
\boldsymbol{b}^{\mathcal{H}} \end{array} \right]
=
\left[ \begin{array}{cc}
\boldsymbol{L}^{\mathcal{DD}} & \boldsymbol{L}^{\mathcal{DH}} \\
\boldsymbol{L}^{\mathcal{HD}} & \boldsymbol{L}^{\mathcal{HH}} \end{array} \right]
\left[ \begin{array}{c}
\boldsymbol{y}^{\mathcal{D}} \\
\boldsymbol{y}^{\mathcal{H}} \end{array} \right]
\end{align*}
    
Note that only the homogeneous part $\boldsymbol{b}^{\mathcal{H}}$ will be used
to calculate the stage values. This means essentially that only the bottom part
of the operation above, i.e.
$\boldsymbol{L}^{\mathcal{HD}}\boldsymbol{y}^{\mathcal{D}} +
\boldsymbol{L}^{\mathcal{HH}}\boldsymbol{y}^{\mathcal{H}}$ is required. However,
sometimes it might be more convenient to use/implement routines for the explicit
operator that also calculate $\boldsymbol{b}^{\mathcal{D}}$.\\
 
An implicit method should solve the system:
\begin{align*}
\left(\left[ \begin{array}{cc}
\boldsymbol{M}^{\mathcal{DD}} & \boldsymbol{M}^{\mathcal{DH}} \\
\boldsymbol{M}^{\mathcal{HD}} & \boldsymbol{M}^{\mathcal{HH}} \end{array} \right]
- \lambda \left[ \begin{array}{cc}
\boldsymbol{L}^{\mathcal{DD}} & \boldsymbol{L}^{\mathcal{DH}} \\
\boldsymbol{L}^{\mathcal{HD}} & \boldsymbol{L}^{\mathcal{HH}} \end{array} \right]\right)
\left[ \begin{array}{c}
\boldsymbol{y}^{\mathcal{D}} \\
\boldsymbol{y}^{\mathcal{H}} \end{array} \right]
=
\left[ \begin{array}{cc}
\boldsymbol{H}^{\mathcal{DD}} & \boldsymbol{H}^{\mathcal{DH}} \\
\boldsymbol{H}^{\mathcal{HD}} & \boldsymbol{H}^{\mathcal{HH}} \end{array} \right]
\left[ \begin{array}{c}
\boldsymbol{y}^{\mathcal{D}} \\
\boldsymbol{y}^{\mathcal{H}} \end{array} \right]
=
\left[ \begin{array}{c}
\boldsymbol{b}^{\mathcal{D}} \\
\boldsymbol{b}^{\mathcal{H}} \end{array} \right]
\end{align*}
    
for the unknown vector $\boldsymbol{y}$. This can be done in three steps:
\begin{itemize}
\item Set the known solution $\boldsymbol{y}^{\mathcal{D}}$
\item Calculate the modified right hand side term
 $\boldsymbol{b}^{\mathcal{H}} -
 \boldsymbol{H}^{\mathcal{HD}}\boldsymbol{y}^{\mathcal{D}}$
\item Solve the system below for the unknown
 $\boldsymbol{y}^{\mathcal{H}}$, i.e.
 $\boldsymbol{H}^{\mathcal{HH}}\boldsymbol{y}^{\mathcal{H}} =
 \boldsymbol{b}^{\mathcal{H}} -
 \boldsymbol{H}^{\mathcal{HD}}\boldsymbol{y}^{\mathcal{D}}$
\end{itemize}

\subsection{How to add a new time-stepping method}

To add a new time integration scheme, follow the steps below:
\begin{itemize}
\item Choose a name for the method and add it to the
\texttt{TimeIntegrationMethod} enum list.
\item Populate the switch statement in the \texttt{TimeIntegrationScheme}
constructor with the coefficients of the new method.
\item Use ( or modify) the function \texttt{InitializeScheme} to select (or
implement) a proper initialisation strategy for the method.
\item Add documentation for the method (especially indicating what the auxiliary
 parameters of the input and output vectors of the multi-step method represent)
\end{itemize}

\subsection{Examples of already implemented time stepping schemes}

Here we show some examples time-stepping schemes implemented in {\nek}, to give an idea of what is required
to add one of them.

\subsubsection{Forward Euler}
\begin{align*}
\left[\begin{array}{c|c}
A & U \\
\hline
B & V
\end{array}\right] =
\left[\begin{array}{c|c}
0 & 1 \\
\hline
1 & 1
\end{array}\right],\qquad
\boldsymbol{y}^{[n]}=
\left[\begin{array}{c}
\boldsymbol{y}^{[n]}_0
\end{array}\right]=
\left[\begin{array}{c}
\boldsymbol{m}\left(t^n,\boldsymbol{y}^{n}\right)
\end{array}\right]
\end{align*}

\subsubsection{Backward Euler}
\begin{align*}
\left[\begin{array}{c|c}
A & U \\
\hline
 B & V
\end{array}\right] =
\left[\begin{array}{c|c}
1 & 1 \\
\hline
1 & 1
\end{array}\right],\qquad
\boldsymbol{y}^{[n]}=
\left[\begin{array}{c}
\boldsymbol{y}^{[n]}_0
\end{array}\right]=
\left[\begin{array}{c}
\boldsymbol{m}\left(t^n,\boldsymbol{y}^{n}\right)
\end{array}\right]
\end{align*}

\subsubsection{2$^{nd}$ order Adams-Bashforth}
\begin{align*}
\left[\begin{array}{c|c}
A & U \\
\hline
B & V
\end{array}\right] =
\left[\begin{array}{c|cc}
0 & 1 & 0 \\
\hline
\frac{3}{2} & 1 & \frac{-1}{2}  \\
1 & 0 & 0
\end{array}\right],\qquad
\boldsymbol{y}^{[n]}=
\left[\begin{array}{c}
\boldsymbol{y}^{[n]}_0\\
\boldsymbol{y}^{[n]}_1\\
\end{array}\right]=
\left[\begin{array}{c}
\boldsymbol{m}\left(t^n,\boldsymbol{y}^{n}\right)\\
\Delta t \boldsymbol{l}(t^{n-1},\boldsymbol{y}^{n-1})
\end{array}\right]
\end{align*}

\subsubsection{1$^{st}$ order IMEX Euler-Backward/ Euler-Forward}
\begin{align*}
\left[ \begin{array}{cc|c}
A^{\mathrm{IM}} & A^{\mathrm{EM}} & U \\
\hline
B^{\mathrm{IM}} & B^{\mathrm{EM}} & V
\end{array} \right ] =
\left[\begin{array}{cc|c}
\left [ \begin{array}{c} 1 \end{array} \right ] & \left [ \begin{array}{c} 0 \end{array} \right ] &
\left [ \begin{array}{cc} 1 & 1 \end{array} \right ] \\
\hline
\left [ \begin{array}{c} 1 \\ 0 \end{array} \right ] & \left [ \begin{array}{c} 0 \\ 1 \end{array} \right ]&
\left [ \begin{array}{cc} 1 & 1 \\ 0 & 0 \end{array} \right ]
\end{array}\right] \quad \mathrm{with}\quad
\boldsymbol{y}^{[n]}=
\left[\begin{array}{c}
\boldsymbol{y}^{[n]}_0\\
\boldsymbol{y}^{[n]}_1
\end{array}\right]=
\left[\begin{array}{c}
\boldsymbol{m}\left(t^n,\boldsymbol{y}^{n}\right)\\
\Delta t \boldsymbol{l}\left ( t^n, \boldsymbol{y}^{n}\right)
\end{array}\right]
\end{align*}

\subsubsection{2$^{nd}$ order IMEX Backward Different Formula \& Extrapolation}
\begin{align*}
\left[ \begin{array}{cc|c}
A^{\mathrm{IM}} & A^{\mathrm{EM}} & U \\
\hline
B^{\mathrm{IM}} & B^{\mathrm{EM}} & V
\end{array} \right ] =
\left[\begin{array}{cc|c}
\left [ \begin{array}{c} \frac{2}{3} \end{array} \right ] & \left [ \begin{array}{c} 0 \end{array} \right ] &
\left [ \begin{array}{cccc} \frac{4}{3} & -\frac{1}{3} & \frac{4}{3} &  -\frac{2}{3} \end{array} \right ] \\
\hline
\left [ \begin{array}{c} \frac{2}{3} \\ 0 \\ 0 \\ 0 \end{array} \right ] & \left [ \begin{array}{c} 0 \\0 \\ 1 \\ 0 \end{array} \right ]&
\left [ \begin{array}{cccc} \frac{4}{3} & -\frac{1}{3} & \frac{4}{3} & -\frac{2}{3} \\ 1 & 0 & 0 & 0 \\ 0 & 0 & 0 & 0\\ 0 & 0 & 1 & 0 \end{array} \right ]
\end{array}\right]
\end{align*}
with
\begin{align*}
\boldsymbol{y}^{[n]}=
\left[\begin{array}{c}
\boldsymbol{y}^{[n]}_0\\
\boldsymbol{y}^{[n]}_1 \\
\boldsymbol{y}^{[n]}_2 \\
\boldsymbol{y}^{[n]}_3
\end{array}\right]=
\left[\begin{array}{l}
\boldsymbol{m}\left(t^n,\boldsymbol{y}^{n}\right)\\
\boldsymbol{m}\left(t^{n-1},\boldsymbol{y}^{n-1}\right)\\
\Delta t \boldsymbol{l}\left ( t^n, \boldsymbol{y}^{n}\right)\\
\Delta t \boldsymbol{l}\left ( t^{n-1}, \boldsymbol{y}^{n-1}\right)
\end{array}\right]
\end{align*}

\begin{notebox}
The first two rows are normalised so the coefficient on
$\boldsymbol{y}_n^{[n+1]}$ is one. In the standard formulation it is $3/2$.
\end{notebox}

\subsubsection{3rdorder IMEX Backward Different Formula \& Extrapolation}
\begin{align*}
  \left[ \begin{array}{cc|c}
  A^{\mathrm{IM}} & A^{\mathrm{EM}} & U \\
  \hline
  B^{\mathrm{IM}} & B^{\mathrm{EM}} & V
  \end{array} \right ] =
  \left[\begin{array}{cc|c}
   \left [ \begin{array}{c} \frac{6}{11} \end{array} \right ] & \left [ \begin{array}{c} 0 \end{array} \right ] &
   \left [ \begin{array}{cccccc} \frac{18}{11} & -\frac{9}{11} & \frac{2}{11} & \frac{18}{11} &  -\frac{18}{11} & \frac{6}{11} \end{array} \right ] \\
  \hline
   \left [ \begin{array}{c} \frac{6}{11}\\ 0 \\ 0 \\ 0 \\0 \\0 \end{array} \right ] & \left [ \begin{array}{c} 0 \\0 \\ 0 \\ 1 \\ 0 \\ 0 \end{array} \right ]&
   \left [ \begin{array}{cccccc} \frac{18}{11} & -\frac{9}{11} & \frac{2}{11} & \frac{18}{11} & -\frac{18}{11} & \frac{6}{11} \\ 1 & 0 & 0 & 0 & 0 & 0 \\ 0 & 1 & 0 & 0 & 0 & 0 \\ 0 & 0 & 0 & 0 & 0 & 0\\ 0 & 0 & 0 & 1 & 0 & 0 \\ 0 & 0 &  0 & 0 & 1 & 0 \end{array} \right ]
  \end{array}\right]
\end{align*}
with
\begin{align*}
  \boldsymbol{y}^{[n]}=
  \left[\begin{array}{l}
  \boldsymbol{y}^{[n]}_0 \\
  \boldsymbol{y}^{[n]}_1 \\
  \boldsymbol{y}^{[n]}_2 \\
  \boldsymbol{y}^{[n]}_3 \\
  \boldsymbol{y}^{[n]}_4 \\
  \boldsymbol{y}^{[n]}_5
  \end{array}\right]=
  \left[\begin{array}{l}
  \boldsymbol{m}\left(t^n,\boldsymbol{y}^{n}\right)\\
  \boldsymbol{m}\left(t^{n-1},\boldsymbol{y}^{n-1}\right)\\
  \boldsymbol{m}\left(t^{n-2},\boldsymbol{y}^{n-2}\right)\\
  \Delta t \boldsymbol{l}\left ( t^n, \boldsymbol{y}^{n}\right)\\
  \Delta t \boldsymbol{l}\left ( t^{n-1}, \boldsymbol{y}^{n-1}\right)\\
  \Delta t \boldsymbol{l}\left ( t^{n-2}, \boldsymbol{y}^{n-2}\right)
  \end{array}\right]
\end{align*}

\begin{notebox}
The first two rows are normalised so the coefficient on
$\boldsymbol{y}_n^{[n+1]}$ is one. In the standard formulation it is $11/6$.
\end{notebox}

\subsubsection{2$^{nd}$ order Adams-Moulton}
\begin{align*}
  \left[\begin{array}{c|c}
  A & U \\
  \hline
  B & V
  \end{array}\right] =
  \left[\begin{array}{c|cc}
  \frac{1}{2} & 1 & \frac{1}{2} \\
  \hline
  \frac{1}{2} & 1 & \frac{1}{2} \\
  1 & 0 & 0
  \end{array}\right],\qquad
  \boldsymbol{y}^{[n]}=
  \left[\begin{array}{c}
  \boldsymbol{y}^{[n]}_0\\
  \boldsymbol{y}^{[n]}_1\\
  \end{array}\right]=
  \left[\begin{array}{c}
  \boldsymbol{m}\left(t^n,\boldsymbol{y}^{n}\right)\\
  \Delta t \boldsymbol{l}(t^{n},\boldsymbol{y}^{n})
  \end{array}\right]
\end{align*}

\subsubsection{Midpoint Method}
\begin{align*}
  \left[\begin{array}{c|c}
  A & U \\
  \hline
  B & V
  \end{array}\right] =
  \left[\begin{array}{cc|c}
  0 & 0 & 1 \\
  \frac{1}{2} & 0 & 1 \\
  \hline
  0 & 1 & 1
  \end{array}\right],\qquad
  \boldsymbol{y}^{[n]}=
  \left[\begin{array}{c}
  \boldsymbol{y}^{[n]}_0
  \end{array}\right]=
  \left[\begin{array}{c}
  \boldsymbol{m}\left(t^n,\boldsymbol{y}^{n}\right)
  \end{array}\right]
\end{align*}

\subsubsection{RK4: the standard fourth-order Runge-Kutta scheme}
\begin{align*}
  \left[\begin{array}{c|c}
  A & U \\
  \hline
  B & V
  \end{array}\right] =
  \left[\begin{array}{cccc|c}
  0 & 0 & 0 & 0 & 1 \\
  \frac{1}{2} & 0 & 0 & 0 & 1 \\
  0 & \frac{1}{2} & 0 & 0 & 1 \\
  0 & 0 & 1 & 0 & 1 \\
  \hline
  \frac{1}{6} & \frac{1}{3} & \frac{1}{3} & \frac{1}{6} & 1
  \end{array}\right],\qquad
  \boldsymbol{y}^{[n]}=
  \left[\begin{array}{c}
  \boldsymbol{y}^{[n]}_0
  \end{array}\right]=
  \left[\begin{array}{c}
  \boldsymbol{m}\left(t^n,\boldsymbol{y}^{n}\right)
  \end{array}\right]
\end{align*}

\subsubsection{2$^{nd}$ order Diagonally Implicit Runge-Kutta (DIRK)}
\begin{align*}
  \left[\begin{array}{c|c}
  A & U \\
  \hline
  B & V
  \end{array}\right] =
  \left[\begin{array}{cc|c}
  \lambda & 0 & 1 \\
  \left(1-\lambda\right) & \lambda & 1 \\
  \hline
  \left(1-\lambda\right) & \lambda & 1
  \end{array}\right]\quad \mathrm{with}\quad \lambda=\frac{2-\sqrt{2}}{2},\qquad
  \boldsymbol{y}^{[n]}=
  \left[\begin{array}{c}
  \boldsymbol{y}^{[n]}_0
  \end{array}\right]=
  \left[\begin{array}{c}
  \boldsymbol{m}\left(t^n,\boldsymbol{y}^{n}\right)
  \end{array}\right]
\end{align*}

\subsubsection{3$^{rd}$ order Diagonally Implicit Runge-Kutta (DIRK)}
\begin{align*}
  \left[\begin{array}{c|c}
  A & U \\
  \hline
  B & V
  \end{array}\right] =
  \left[\begin{array}{ccc|c}
  \lambda & 0 & 0 & 1 \\
  \frac{1}{2}\left(1-\lambda\right) & \lambda & 0 & 1 \\
  \frac{1}{4}\left(-6\lambda^2+16\lambda-1\right) & \frac{1}{4}\left(6\lambda^2-20\lambda+5\right) & \lambda & 1 \\
  \hline
  \frac{1}{4}\left(-6\lambda^2+16\lambda-1\right) & \frac{1}{4}\left(6\lambda^2-20\lambda+5\right) & \lambda & 1
  \end{array}\right]
\end{align*}
with
\begin{align*}
  \lambda=0.4358665215,\qquad
  \boldsymbol{y}^{[n]}=
  \left[\begin{array}{c}
  \boldsymbol{y}^{[n]}_0
  \end{array}\right]=
  \left[\begin{array}{c}
  \boldsymbol{m}\left(t^n,\boldsymbol{y}^{n}\right)
  \end{array}\right]
\end{align*}

\subsubsection{3$^{rd}$ order L-stable, three stage IMEX DIRK(3,4,3)}
\tiny
\begin{align*}
  &\left[\begin{array}{cc|c}
  A^{\mathrm{IM}} & A^{\mathrm{EM}} & U \\
  \hline
  B^{\mathrm{IM}} & B^{\mathrm{EM}} & V
  \end{array}\right] =\\
  &\left[\begin{array}{cc|c}
  \left[\begin{array}{cccc}
  0 & 0 & 0 & 0 \\
  0 & \lambda & 0 & 0 \\
  0 & \frac{1}{2}\left(1-\lambda\right) & \lambda & 0 \\
  0 & \frac{1}{4}\left(-6\lambda^2+16\lambda-1\right) & \frac{1}{4}\left(6\lambda^2-20\lambda+5\right) & \lambda
  \end{array}\right] &
  \left[\begin{array}{cccc}
  0 & 0 & 0 & 0 \\
  \lambda & 0 & 0 & 0 \\
  0.3212788860 & 0.3966543747 & 0 & 0 \\
  -0.105858296 & 0.5529291479 & 0.5529291479 & 0
  \end{array}\right] &
  \left[\begin{array}{c}
  1\\
  1\\
  1\\
  1
  \end{array}\right] \\
  \hline
  \left[\begin{array}{cccc}
  0 & \frac{1}{4}\left(-6\lambda^2+16\lambda-1\right) & \frac{1}{4}\left(6\lambda^2-20\lambda+5\right) & \lambda
  \end{array}\right] &
  \left[\begin{array}{cccc}
  0 & \frac{1}{4}\left(-6\lambda^2+16\lambda-1\right) & \frac{1}{4}\left(6\lambda^2-20\lambda+5\right) & \lambda
  \end{array}\right] &
  \left[\begin{array}{c}
  1
  \end{array}\right]
  \end{array}\right]
\end{align*}
\normalsize
with
\begin{align*}
  \lambda=0.4358665215,\qquad
  \boldsymbol{y}^{[n]}=
  \left[\begin{array}{c}
  \boldsymbol{y}^{[n]}_0
  \end{array}\right]=
  \left[\begin{array}{c}
  \boldsymbol{m}\left(t^n,\boldsymbol{y}^{n}\right)
  \end{array}\right]
\end{align*}


