\section{Installing from Source}
\label{s:installation:source}

This section explains how to build Nektar++ from the source-code package.

Nektar++ uses a number of third-party libraries. Some of these are required,
others are optional. It is generally more straightforward to use versions of
these libraries supplied pre-packaged for your operating system, but if you run
into difficulties with compilation errors or failing regression tests, the
Nektar++ build system can automatically build tried-and-tested versions of these
libraries for you. This requires enabling the relevant options in the CMake
configuration.


\subsection{Obtaining the source code}
\label{s:installation:source:obtainingsource}
There are two ways to obtain the source code for \nekpp:
\begin{itemize}
	\item Download the latest source-code archive from the
	\href{http://www.nektar.info/downloads}{Nektar++ downloads page}.
	\item Clone the git repository
	\begin{itemize}
	\item Using anonymous access. This does not require
	credentials but any changes to the code cannot be pushed to the
	public repository. Use this initially if you would like to try using
	Nektar++ or make local changes to the code.
    \begin{lstlisting}[style=BashInputStyle]
        git clone https://gitlab.nektar.info/nektar/nektar.git nektar++
    \end{lstlisting}
	\item Using authenticated access. This will allow you to directly contribute
	back into the code.
    \begin{lstlisting}[style=BashInputStyle]
        git clone git@gitlab.nektar.info:nektar/nektar.git nektar++
    \end{lstlisting}
    \begin{tipbox}
    You can easily switch to using the authenticated access from anonymous
    access at a later date by running\\
    \footnotesize\texttt{git remote set-url origin git@gitlab.nektar.info:nektar/nektar.git}
    \end{tipbox}
	\end{itemize}
\end{itemize}

\subsection{Linux}
\subsubsection{Prerequisites}
\nekpp uses a number of external programs and libraries for some or all of its
functionality. Some of these are \emph{required} and must be installed prior to
compiling Nektar++, most of which are available as pre-built \emph{system}
packages on most Linux distributions or can be installed manually by a
\emph{user}. Typically, the development packages, with a \emph{-dev} or \emph{-devel} suffix, are required to compile codes against these libraries. Others are optional and required only for specific features, or can
be downloaded and compiled for use with Nektar++ \emph{automatically} (but not
installed system-wide).

\begin{center}
\begin{tabularx}{\linewidth}{lccccX}
\toprule
        &      & \multicolumn{3}{c}{Installation} & \\ \cmidrule(r){3-5}
Package & Req. & Sys. & User & Auto.              & Note \\
\midrule
C++ compiler    & \cmark & \cmark & & & gcc, icc, etc, supporting C++11 \\
CMake $\geq 3.5.1$ & \cmark & \cmark & \cmark &        & Ncurses
GUI optional
\\
BLAS            & \cmark & \cmark & \cmark & \cmark & Or MKL,
ACML, OpenBLAS
\\
LAPACK          & \cmark & \cmark & \cmark & \cmark & \\
Boost $>=1.56$   & \cmark & \cmark & \cmark & \cmark & Compile
with iostreams
\\
TinyXML         & \cmark & \cmark & \cmark & \cmark & For reading XML input files\\
Scotch          & \cmark & \cmark & \cmark & \cmark & Required
for multi-level static condensation, highly recommended\\
METIS           &        & \cmark & \cmark & \cmark &
Alternative mesh partitioning\\
FFTW $>3.0$     &        & \cmark & \cmark & \cmark & For
high-performance FFTs\\
ARPACK $>2.0$   &        & \cmark & \cmark & \cmark & For
arnoldi algorithms\\
MPI             &        & \cmark & \cmark &        & For
parallel execution (OpenMPI, MPICH, Intel MPI, etc)\\
GSMPI           &        &        &        & \cmark & For
parallel execution\\
HDF5            &        & \cmark & \cmark & \cmark & For
large-scale parallel I/O (requires CMake >3.1)\\
OpenCascade CE  &        & \cmark & \cmark & \cmark & For mesh generation and optimisation\\
PETSc           &        & \cmark & \cmark & \cmark &
Alternative linear solvers\\
PT-Scotch       &        & \cmark & \cmark & \cmark & Required when MPI enabled\\
Tetgen          &        & \cmark & \cmark & \cmark & For 3D mesh generation\\
Triangle        &        & \cmark & \cmark & \cmark & For 2D mesh generation\\
VTK $>5.8$      &        & \cmark & \cmark &        & Not required to convert field output files to VTK, only mesh files\\
\bottomrule
\end{tabularx}
\end{center}


\subsubsection{Quick Start}
Open a terminal.

If you have downloaded the tarball, first unpack it:
\begin{lstlisting}[style=BashInputStyle]
tar -zxvf nektar++-(*\nekver*).tar.gz
\end{lstlisting}
Change into the \inlsh{nektar++} source code directory
\begin{lstlisting}[style=BashInputStyle]
    mkdir -p build && cd build
    ccmake ../
    make install
\end{lstlisting}

\subsubsection{Detailed instructions}
From a terminal:
\begin{enumerate}
    \item If you have downloaded the tarball, first unpack it
    \begin{lstlisting}[style=BashInputStyle]
tar -zxvf nektar++-(*\nekver*).tar.gz
    \end{lstlisting}

    \item Change into the source-code directory, create a \inltt{build}
    subdirectory and enter it
    \begin{lstlisting}[style=BashInputStyle]
    mkdir -p build && cd build
    \end{lstlisting}

    \item Run the CMake GUI and configure the build by pressing \inltt{c}
    \begin{lstlisting}[style=BashInputStyle]
    ccmake ../
    \end{lstlisting}
    \begin{itemize}
        \item Select the components of Nektar++ (prefixed with
        \inltt{NEKTAR\_BUILD\_}) you would like to build. Disabling solvers
        which you do not require will speed up the build process.
        \item Select the optional libraries you would like to use (prefixed with
        \inltt{NEKTAR\_USE\_}) for additional functionality.
        \item Select the libraries not already available on your system which
        you wish to be compiled automatically (prefixed with
        \inltt{THIRDPARTY\_BUILD\_}). Some of these will be automatically enabled if not found on your system.
        \item Choose the installation location by adjusting \inltt{CMAKE\_INSTALL\_PREFIX}. By default, this will be a \inltt{dist} subdirectory within the \inltt{build} directory, which is satisfactory for most users initially.
    \end{itemize}
    A full list of configuration options can be found in
    Section~\ref{s:installation:source:cmake}.

    \begin{notebox}
    Selecting \inltt{THIRDPARTY\_BUILD\_} options will request CMake to
    automatically download thirdparty libraries and compile them within the
    \nekpp directory. If you have administrative access to your machine, it is
    recommended to install the libraries system-wide through your
    package-management system.\\[3pt]

    If you have installed additional system packages since running CMake, you may need to wipe your build directory and rerun CMake for them to be detected.
    \end{notebox}


    \item Press \inltt{c} to configure the build. If errors arise relating to
    missing libraries, review the \inltt{THIRDPARTY\_BUILD\_} selections in
    the configuration step above or install the missing libraries manually or
    from system packages.

    \item When configuration completes without errors, press \inltt{c} again
    until the option \inltt{g} to generate build files appears. Press \inltt{g}
    to generate the build files and exit CMake.

    \item Compile the code
    \begin{lstlisting}[style=BashInputStyle]
        make install
    \end{lstlisting}
    During the build, missing third-party libraries will be automatically
    downloaded, configured and built in the \nekpp \inlsh{build} directory.

    % Hacky way to get an lstlisting to an argument of a macro
    \newsavebox\installationLinuxTip
    \begin{lrbox}{\installationLinuxTip}\begin{minipage}{0.8\linewidth}
    \begin{lstlisting}[style=BashInputStyle]
    make -j4 install
    \end{lstlisting}
    \end{minipage}
    \end{lrbox}

    \begin{tipbox}
    If you have multiple processors/cores on your system, compilation can be
    significantly increased by adding the \inlsh{-jX} option to make, where X is
    the number of simultaneous jobs to spawn. For example, use

    \noindent\usebox\installationLinuxTip

    on a quad-core system.
    \end{tipbox}

    \item Test the build by running unit and regression tests.
    \begin{lstlisting}[style=BashInputStyle]
    ctest
    \end{lstlisting}
\end{enumerate}

\subsection{OS X}

\subsubsection{Prerequisites}
\nekpp uses a number of external programs and libraries for some or all of its
functionality. Some of these are \emph{required} and must be installed prior to
compiling Nektar++, most of which are available on \emph{MacPorts}
(www.macports.org) or can be installed manually by a \emph{user}. Others are
optional and required only for specific features, or can be downloaded and
compiled for use with Nektar++ \emph{automatically} (but not installed
system-wide).

\begin{notebox}
  To compile \nekpp on OS X, Apple's Xcode Developer Tools must be
  installed. They can be installed either from the App Store (only on Mac OS
  10.7 and above) or downloaded directly from
  \href{http://connect.apple.com/}{http://connect.apple.com/} (you are required
  to have an Apple Developer Connection account).  Xcode includes Apple
  implementations of BLAS and LAPACK (called the Accelerate Framework).
\end{notebox}

\begin{center}
\begin{tabularx}{\linewidth}{lccccX}
\toprule
        &      & \multicolumn{3}{c}{Installation} & \\ \cmidrule(r){3-5}
Package & Req. & MacPorts & User & Auto.          & Note \\
\midrule
Xcode           & \cmark &        & & & Provides developer tools \\
CMake $\geq 3.5.1$ & \cmark & \texttt{cmake}   & \cmark &        & Ncurses
GUI optional \\
BLAS            & \cmark &                  &        &        & Part of
Xcode \\
LAPACK          & \cmark &                  &        &        & Part of
Xcode \\
Boost $>=1.56$  & \cmark & \texttt{boost}   & \cmark & \cmark & Compile
with iostreams \\
TinyXML         & \cmark & \texttt{tinyxml} & \cmark & \cmark & \\
Scotch          & \cmark & \texttt{scotch}  & \cmark & \cmark & Required
for multi-level static condensation, highly recommended\\
METIS           &        & \texttt{metis}   & \cmark & \cmark &
Alternative mesh partitioning\\
FFTW $>3.0$     &        & \texttt{fftw-3}  & \cmark & \cmark & For
high-performance FFTs\\
ARPACK $>2.0$   &        & \texttt{arpack}  & \cmark &        & For
arnoldi algorithms\\
OpenMPI         &        & \texttt{openmpi} &        &        & For
parallel execution\\
GSMPI           &        &                  &        & \cmark & For
parallel execution\\
HDF5            &        &                  & \cmark & \cmark & For
large-scale parallel I/O (requires CMake >3.1)\\
OpenCascade CE  &        &                  & \cmark & \cmark & For mesh generation and optimisation\\
PETSc           &        & \texttt{petsc}   & \cmark & \cmark &
Alternative linear solvers\\
PT-Scotch       &        &  & \cmark & \cmark & Required when MPI enabled\\
Tetgen          &        &  & \cmark & \cmark & For 3D mesh generation\\
Triangle        &        &  & \cmark & \cmark & For 2D mesh generation\\
VTK $>5.8$      &        & \texttt{vtk}     & \cmark &        &
Not required to convert field output files to VTK, only mesh files\\
\bottomrule
\end{tabularx}
\end{center}

\newsavebox\installationOSXMacPortsTip
\begin{lrbox}{\installationOSXMacPortsTip}\begin{minipage}{0.8\linewidth}
\begin{lstlisting}[style=BashInputStyle]
sudo port install cmake
\end{lstlisting}
\end{minipage}
\end{lrbox}

\begin{tipbox}
CMake, and some other software, is available from MacPorts
(\url{http://macports.org}) and can be installed using, for example,

\noindent\usebox\installationOSXMacPortsTip

Package names are given in the table above. Similar packages also exist in other
package managers such as Homebrew.
\end{tipbox}


\subsubsection{Quick Start}
Open a terminal (Applications->Utilities->Terminal).

If you have downloaded the tarball, first unpack it:
\begin{lstlisting}[style=BashInputStyle]
tar -zxvf nektar++-(*\nekver*).tar.gz
\end{lstlisting}
Change into the \inlsh{nektar++} source code directory
\begin{lstlisting}[style=BashInputStyle]
    mkdir -p build && cd build
    ccmake ../
    make install
\end{lstlisting}

\subsubsection{Detailed instructions}
From a terminal (Applications->Utilities->Terminal):
\begin{enumerate}
    \item If you have downloaded the tarball, first unpack it
    \begin{lstlisting}[style=BashInputStyle]
tar -zxvf nektar++-(*\nekver*).tar.gz
    \end{lstlisting}

    \item Change into the source-code directory, create a \inltt{build}
    subdirectory and enter it
    \begin{lstlisting}[style=BashInputStyle]
    mkdir -p build && cd build
    \end{lstlisting}

    \item Run the CMake GUI and configure the build
    \begin{lstlisting}[style=BashInputStyle]
    ccmake ../
    \end{lstlisting}
    Use the arrow keys to navigate the options and \inlsh{ENTER} to select/edit
    an option.
    \begin{itemize}
        \item Select the components of Nektar++ (prefixed with
        \inltt{NEKTAR\_BUILD\_}) you would like to build. Disabling solvers
        which you do not require will speed up the build process.
        \item Select the optional libraries you would like to use (prefixed with
        \inltt{NEKTAR\_USE\_}) for additional functionality.
        \item Select the libraries not already available on your system which
        you wish to be compiled automatically (prefixed with
        \inltt{THIRDPARTY\_BUILD\_})
        \item Choose the installation location by adjusting \inltt{CMAKE\_INSTALL\_PREFIX}. By default, this will be a \inltt{dist} subdirectory within the \inltt{build} directory, which is satisfactory for most users initially.
    \end{itemize}
    A full list of configuration options can be found in
    Section~\ref{s:installation:source:cmake}.

    \begin{notebox}
    Selecting \inltt{THIRDPARTY\_BUILD\_} options will request CMake to
    automatically download thirdparty libraries and compile them within the
    \nekpp directory. If you have administrative access to your machine, it is
    recommended to install the libraries system-wide through MacPorts.
    \end{notebox}

    \item Press \inltt{c} to configure the build. If errors arise relating to
        missing libraries (variables set to \inlsh{NOTFOUND}), review the
        \inltt{THIRDPARTY\_BUILD\_} selections in the previous
    step or install the missing libraries manually or through MacPorts.

    \item When configuration completes without errors, press \inltt{c} again
    until the option \inltt{g} to generate build files appears. Press \inltt{g}
    to generate the build files and exit CMake.

    \item Compile the code
    \begin{lstlisting}[style=BashInputStyle]
        make install
    \end{lstlisting}
    During the build, missing third-party libraries will be automatically
    downloaded, configured and built in the \nekpp \inlsh{build} directory.

    % Hacky way to get an lstlisting to an argument of a macro
    \newsavebox\installationMacTip
    \begin{lrbox}{\installationMacTip}\begin{minipage}{0.8\linewidth}
    \begin{lstlisting}[style=BashInputStyle]
    make -j4 install
    \end{lstlisting}
    \end{minipage}
    \end{lrbox}

    \begin{tipbox}
    If you have multiple processors/cores on your system, compilation can be
    significantly increased by adding the \inlsh{-jX} option to make, where X is
    the number of simultaneous jobs to spawn. For example,
    \noindent\usebox\installationMacTip
    \end{tipbox}

    \item Test the build by running unit and regression tests.
    \begin{lstlisting}[style=BashInputStyle]
    ctest
    \end{lstlisting}
\end{enumerate}

\subsection{Windows}

\lstset{showstringspaces=false}

Windows compilation is supported but there are some complexities with building
additional features on this platform at present. As such, only builds with
a minimal amount of additional build packages are currently supported. These can
either be installed by the user, or automatically as part of the build process.
Support has recently been added for building with MPI on Windows. This enables
parallel computations to be carried out with \nekpp on Windows where only
sequential computations were previously supported.

\begin{center}
\begin{tabularx}{\linewidth}{lcccX}
\toprule
                  &        & \multicolumn{2}{c}{Installation} & \\ \cmidrule(r){3-4}
Package           & Req.   & User   & Auto.        & Note \\
\midrule
MS Visual Studio  & \cmark & \cmark &              & 2015, 2017 and 2019 known working\\
CMake $\geq 3.5.1$  & \cmark & \cmark &              & 3.16+ recommended, see info below\\
BLAS              & \cmark & \cmark & \cmark       & \\
LAPACK            & \cmark & \cmark & \cmark       & \\
Boost $\geq 1.61$ & \cmark & \cmark & \cmark       & Recommend installing from binaries\\ %Compile with iostreams\\
Microsoft MPI $\geq 10.1.2$    &                & \cmark &        & Required for parallel execution. Install both runtime and SDK\\
\bottomrule
\end{tabularx}
\end{center}

  \begin{notebox}
    These instructions assume you are using a 64-bit version of Windows 10.
  \end{notebox}

  \begin{notebox}
    There have been issues with automatically building Boost from source as
    a third party dependency during the \nekpp build when using MS Visual
    Studio 2015, 2017 and 2019. This should now be possible but it is,
    nonetheless, recommended you install a suitable version of Boost from
    binaries as detailed in the instructions below.
  \end{notebox}

\subsubsection{Detailed instructions}

\begin{enumerate}
  \item Install Microsoft Visual Studio 2019 (preferred), 2017 or 2015 (both
  known to work). This can be obtained from Microsoft free of charge by using
  their Community developer tools from
  \url{https://visualstudio.microsoft.com/vs/community/}.

  \item Install CMake from \url{http://www.cmake.org/download/}. For building
  on Windows you are strongly recommended to use a recent version of CMake,
  e.g 3.16+. Minimum required CMake versions for building \nekpp on Windows
  with Visual Studio are CMake 3.5.1+ (VS2015), 3.7+ (VS2017), or 3.15+
  (VS2019). When prompted, select the option to add CMake to the system PATH.

  \item (Optional) As highlighted above, it is possible to have Boost built
  from source as a third-party library during the \nekpp build. However, it
  is currently recommended to install the Boost binaries that can be found at
  \url{http://sourceforge.net/projects/boost/files/boost-binaries}. By
  default these install into \\ \inlsh{C:\textbackslash local\textbackslash
    boost\_<version>}. We recommend installing a specific version of the
    binaries depending on the version of Visual Studio you are using,
    these are known to be working with the \nekpp build:
  \begin{itemize}
  	\item For Visual Studio 2015, install boost 1.61 using the package
	 \texttt{boost\_1\_61\_0-\\msvc-14.0-64.exe} from \url{http://sourceforge.net/projects/boost/files/boost-binaries/1.61.0/}
  	\item For Visual Studio 2017, install boost 1.68 using the package
	 \texttt{boost\_1\_68\_0-\\msvc-14.1-64.exe} from \url{http://sourceforge.net/projects/boost/files/boost-binaries/1.68.0/}
  	\item For Visual Studio 2019, install boost 1.72 using the package
	 \texttt{boost\_1\_72\_0-\\msvc-14.2-64.exe}  from \url{http://sourceforge.net/projects/boost/files/boost-binaries/1.72.0/}
  \end{itemize}
  If you use these libraries, you will need to:
  \begin{itemize}
  	\item Add a \inlsh{BOOST\_HOME} environment variable. To do so, click the
	Start menu and type `env', you should be presented with an \emph{``Edit
  the system environment variables''} option. Alternatively, from the Start
  menu, navigate to \emph{Settings > System > About > System info
	(under Related Settings on the right hand panel)}, select
	\emph{Advanced System Settings}, and in the \emph{Advanced} tab click the
	\emph{Environment Variables} button. In the \emph{System variables} box,
  click \emph{New}. In the \emph{New System Variable} window, type
  \inlsh{BOOST\_HOME} next to \emph{Variable name} and
  \inlsh{C:\textbackslash local\textbackslash
	<boost\_dir>} next to\emph{Variable value}, where \emph{<boost\_dir>}
  corresponds to the directory that boost has been installed to, based on the
  boost version you have installed. (e.g. \emph{boost\_1\_61\_0},
  \emph{boost\_1\_68\_0}, \emph{boost\_1\_72\_0}).

%    \item Rename \texttt{libs-msvc14.0} to \texttt{lib}
%    \item Inside the \texttt{lib} directory, create duplicates of
%    \texttt{boost\_zlib.dll} and \texttt{boost\_bzip2.dll} called
%    \texttt{zlib.dll} and \texttt{libbz2.dll}
  \end{itemize}
  \item (Optional) Install Git for Windows from
  \url{https://gitforwindows.org/} to use the development versions of \nekpp.
  You can accept the default set of components in the \emph{Select Components}
  panel. When prompted, in the \emph{``Adjusting your PATH environment''}
  panel, select the option \emph{``Git from the command line and also from
  3rd-party software''}. You do not need to select the option to add Unix
  tools to the PATH.

  \item If you've downloaded the source code archive (as described in
  Section~\ref{s:installation:source:obtainingsource}), unpack
  \inlsh{nektar++-\nekver.zip}.

  \begin{notebox}
    Some Windows versions do not recognise the path of a folder which has
    \inltt{++} in the name. If you are not using Windows 10 and think that your
    Windows version cannot handle paths containing special characters, you
    should rename \inlsh{nektar++-\nekver} to \inlsh{nektar-\nekver}.
  \end{notebox}
  \item Create a \inlsh{build} directory within the \inlsh{nektar++-\nekver}
  subdirectory. \textit{If you cloned the source code from the git repository,
  your \nekpp subdirectory will be called \inlsh{nektar} rather than
  \inlsh{nektar++-\nekver}}

  \item Open a Visual Studio terminal (\emph{Developer Command Prompt for
  VS [2015/2017/2019]} or \emph{x64 Native Tools Command Prompt}. From the
  Start menu, this can be found under \emph{Visual Studio [2015/2017/2019]}.

  \item Change directory into the \texttt{build} directory and run CMake to
  generate the build files. You need to set the \emph{generator} to the
  correct Visual Studio version using the -G switch on the command line, e.g.
  for VS2019:
  \begin{lstlisting}[style=BashInputStyle]
    cd C:\path\to\nektar\builds
    cmake -G "Visual Studio 16 2019" ..
  \end{lstlisting}
  You can see the list of available generators using \emph{cmake --help}. For
  VS2017 use \emph{``Visual Studio 15 2017 Win64''} and for VS2015 use
  \emph{``Visual Studio 14 2015 Win64''}.

  If you want to build a parallel version of \nekpp with MPI support, you need
  to add the \emph{-DNEKTAR\_USE\_MPI=ON} switch to the cmake command, e.g.:
  \begin{lstlisting}[style=BashInputStyle]
    cmake -G "Visual Studio 16 2019" -D NEKTAR_USE_MPI=ON ..
  \end{lstlisting}

  \begin{notebox}
  If you installed Boost binaries, as described above, you should ensure at
  this stage that the version of Boost that you installed has been correctly
  detected by CMake. You should see a number of lines of output from CMake
  saying \emph{-- -- Found boost <library name> library: } followed by paths to
  one or more files which should be located in the directory where you
  installed your Boost binaries. If you do not see this output, CMake has
  failed to detect the installed Boost libraries and the build process will
  instead try to build Boost from source as part of building \nekpp.\\

  If you experience any issues with CMake finding pre-installed Boost, binaries
  ensure that you are working in a Visual Studio command prompt that was
  opened \emph{after} you installed boost and set up the \inlsh{BOOST\_HOME}
  environment variable.
  \end{notebox}

  \item Assuming the configuration completes successfully and you see the message \emph{Build files have been written to: ...}, you should now be ready to issue the build command:
  \begin{lstlisting}[style=BashInputStyle]
    msbuild INSTALL.vcxproj /p:Configuration=Release
  \end{lstlisting}
  To build in parallel with, for example, 12 processors, issue:
  \begin{lstlisting}[style=BashInputStyle]
    msbuild INSTALL.vcxproj /p:Configuration=Release /m:12
  \end{lstlisting}
  \item After the build and installation process has completed, the executables will be
  available in \inlsh{build\textbackslash dist\textbackslash bin}.
  \item To use these executables, you need to modify your system \inlsh{PATH} to
  include the \inlsh{bin} directory and library directories where DLLs are stored. To do this, click the
	Start menu and type `env', you should be presented with an \emph{``Edit the system environment variables''}
	option. Alternatively, navigate to \emph{Settings > System > About > System info
	(under Related Settings on the right hand panel)}, select
	\emph{Advanced System Settings}, and in the \emph{Advanced} tab click the
	\emph{Environment Variables} button. In the \emph{System Variables} box, select \inlsh{Path} and click
  \emph{Edit}. Add the \textbf{full paths} to the following directories to the end of the list of paths shown in the \emph{``Edit environment variable''} window:
  \begin{itemize}
    \item \inlsh{nektar++-\nekver\textbackslash build\textbackslash dist\textbackslash lib\textbackslash nektar++-\nekver}
    \item \inlsh{nektar++-\nekver\textbackslash build\textbackslash dist\textbackslash bin}
    \item \inlsh{nektar++-\nekver\textbackslash ThirdParty}
    \item \inlsh{C:\textbackslash local\textbackslash boost\_<boost\_version>\textbackslash <boost\_lib\_dir>}
    where \emph{boost\_<boost\_version>} is the directory where the boost binaries were installed to and
    \emph{<boost\_lib\_dir>} is the name of the library directory within this location, e.g.
    \emph{lib64-msvc-14.2} or similar depending on the version of Boost binaries you installed.
  \end{itemize}
  \item To run the test suite, open a \textbf{new} command line window, change
  to the \inlsh{build} directory, and then issue the command
  \begin{lstlisting}[style=BashInputStyle]
    ctest -C Release
  \end{lstlisting}
\end{enumerate}

\lstset{showstringspaces=true}

\subsection{CMake Option Reference}
\label{s:installation:source:cmake}
This section describes the main configuration options which can be set when
building \nekpp. The default options should work on almost all systems, but
additional features (such as parallelisation and specialist libraries) can be
enabled if needed.

\subsubsection{Components}
The first set of options specify the components of the \nekpp toolkit to
compile. Some options are dependent on others being enabled, so the available
options may change.

Components of the \nekpp package can be selected using the following options:
\begin{itemize}
    \item \inlsh{NEKTAR\_BUILD\_DEMOS} (Recommended)

    Compiles the demonstration programs. These are primarily used by the
    regression testing suite to verify the \nekpp library, but also provide an
    example of the basic usage of the framework.

    \item \inlsh{NEKTAR\_BUILD\_DOC}

    Compiles the Doxygen documentation for the code. This will be put in
    \begin{lstlisting}[style=BashInputStyle]
    $BUILDDIR/doxygen/html
    \end{lstlisting}

    \item \inlsh{NEKTAR\_BUILD\_LIBRARY} (Required)

    Compiles the \nekpp framework libraries. This is required for all other
    options.

    \item \inlsh{NEKTAR\_BUILD\_PYTHON}

    Installs the Python wrapper to \nekpp. Requires running the following command
    after installing \nekpp in order to install the Python package for the
    current user:
    \begin{lstlisting}[style=BashInputStyle]
    make nekpy-install-user
    \end{lstlisting}

    Alternatively, the Python package can be installed for all users by running
    the following command with appropriate priviledges:
    \begin{lstlisting}[style=BashInputStyle]
    make nekpy-install-system
    \end{lstlisting}

    \item \inlsh{NEKTAR\_BUILD\_SOLVERS} (Recommended)

    Compiles the solvers distributed with the \nekpp framework.

    If enabling \inlsh{NEKTAR\_BUILD\_SOLVERS}, individual solvers can be
    enabled or disabled. See Part~\ref{p:applications} for the list of available
    solvers. You can disable solvers which are not required to reduce
    compilation time. See the \inlsh{NEKTAR\_SOLVER\_X} option.

    \item \inlsh{NEKTAR\_BUILD\_TESTS} (Recommended)

    Compiles the testing program used to verify the \nekpp framework.

    \item \inlsh{NEKTAR\_BUILD\_TIMINGS}

    Compiles programs used for timing \nekpp operations.

    \item \inlsh{NEKTAR\_BUILD\_UNIT\_TESTS}

    Compiles tests for checking the core library functions.

    \item \inlsh{NEKTAR\_BUILD\_UTILITIES}

    Compiles utilities for pre- and post-processing simulation data, including the mesh conversion and generation tool \inltt{NekMesh} and the \inltt{FieldConvert} post-processing utility.

    \item \inlsh{NEKTAR\_SOLVER\_X}

    Enable compilation of the 'X' solver.

    \item \inlsh{NEKTAR\_UTILITY\_X}

    Enable compilation of the 'X' utility.
\end{itemize}

A number of ThirdParty libraries are required by \nekpp. There are
also optional libraries which provide additional functionality. These
can be selected using the following options:
\begin{itemize}
    \item \inlsh{NEKTAR\_USE\_ARPACK}

    Build \nekpp with support for ARPACK. This provides routines used for
    linear stability analyses. Alternative Arnoldi algorithms are also
    implemented directly in \nekpp.

    \item \inlsh{NEKTAR\_USE\_CCM}

    Use the ccmio library provided with the Star-CCM package for
    reading ccm files. This option is required as part of NekMesh
    if you wish to convert a Star-CCM mesh into the Nektar format. It
    is possible to read a Tecplot plt file from Star-CCM but this
    output currently needs to be converted to ascii format using the
    Tecplot package.

    \item \inlsh{NEKTAR\_USE\_CWIPI}

    Use the CWIPI library for enabling inter-process communication between two solvers. Solvers may also interface with third-party solvers using this package.

    \item \inlsh{NEKTAR\_USE\_FFTW}

    Build \nekpp with support for FFTW for performing Fast Fourier Transforms
    (FFTs). This is used only when using domains with homogeneous coordinate
    directions.

    \item \inlsh{NEKTAR\_USE\_HDF5}

    Build \nekpp with support for HDF5. This enables input/output in the HDF5
    parallel file format, which can be very efficient for large numbers of
    processes. HDF5 output can be enabled by using a command-line option or
    in the \inlsh{SOLVERINFO} section of the XML file. This option requires
    that \nekpp be built with MPI support with \inlsh{NEKTAR\_USE\_MPI} enabled
    and that HDF5 is compiled with MPI support.

    \item \inlsh{NEKTAR\_USE\_MESHGEN}

    Build the NekMesh utility with support for generating meshes from CAD geometries. This enables use of the OpenCascade Community Edition library, as well as Triangle and Tetgen.

    \item \inlsh{NEKTAR\_USE\_METIS}

    Build \nekpp with support for the METIS graph partitioning library. This
    provides both an alternative mesh partitioning algorithm to SCOTCH for
    parallel simulations.

    \item \inlsh{NEKTAR\_USE\_MPI} (Recommended)

    Build \nekpp with MPI parallelisation. This allows solvers to be run in
    serial or parallel.

    \item \inlsh{NEKTAR\_USE\_PETSC}

    Build \nekpp with support for the PETSc package for solving linear systems.

    \item \inlsh{NEKTAR\_USE\_PYTHON3} (Requires \inlsh{NEKTAR\_BUILD\_PYTHON})

    Enables the generation of Python3 interfaces.

    \item \inlsh{NEKTAR\_USE\_SCOTCH} (Recommended)

    Build \nekpp with support for the SCOTCH graph partitioning library. This
    provides both a mesh partitioning algorithm for parallel simulations and
    enabled support for multi-level static condensation, so is highly
    recommended and enabled by default. However for systems that do not support
    SCOTCH build requirements (e.g. Windows), this can be disabled.

    \item \inlsh{NEKTAR\_USE\_SYSTEM\_BLAS\_LAPACK} (Recommended)

    On Linux systems, use the default BLAS and LAPACK library on the system.
    This may not be the implementation offering the highest performance for your
    architecture, but it is the most likely to work without problem.

    \item \inlsh{NEKTAR\_USE\_VTK}

    Build \nekpp with support for VTK libraries. This is only needed for
    specialist utilities and is not needed for general use.

    \begin{notebox}
    The VTK libraries are not needed for converting the output of simulations to
    VTK format for visualization as this is handled internally.
    \end{notebox}
\end{itemize}

The \inlsh{THIRDPARTY\_BUILD\_X} options select which third-party libraries are
automatically built during the \nekpp build process. Below are the choices of X:
\begin{itemize}
    \item \inlsh{ARPACK}

    Library of iterative Arnoldi algorithms.

    \item \inlsh{BLAS\_LAPACK}

    Library of linear algebra routines.

    \item \inlsh{BOOST}

    The \emph{Boost} libraries are frequently provided by the operating system,
    so automatic compilation is not enabled by default. If you do not have
    Boost on your system, you can enable this to have Boost configured
    automatically.

    \item \inlsh{CCMIO}

    I/O library for the Star-CCM+ format.

    \item \inlsh{CWIPI}

    Library for inter-process exchange of data between different solvers.

    \item \inlsh{FFTW}

    Fast-Fourier transform library.

    \item \inlsh{GSMPI}

    (MPI-only) Parallel communication library.

    \item \inlsh{HDF5}

    Hierarchical Data Format v5 library for structured data storage.

    \item \inlsh{METIS}

    A graph partitioning library used for mesh partitioning when \nekpp is run
    in parallel.

    \item \inlsh{OCE}

    OpenCascade Community Edition 3D modelling library.

    \item \inlsh{PETSC}

    A package for the parallel solution of linear algebra systems.

    \item \inlsh{SCOTCH}

    A graph partitioning library used for mesh partitioning when \nekpp is run
    in parallel, and reordering routines that are used in multi-level static
    condensation.

    \item \inlsh{TETGEN}

    3D tetrahedral meshing library.

    \item \inlsh{TINYXML}

    Library for reading and writing XML files.

    \item \inlsh{TRIANGLE}

    2D triangular meshing library.
\end{itemize}

There are also a number of additional options to fine-tune the build:
\begin{itemize}

    \item \inlsh{NEKTAR\_TEST\_ALL}

    Enables an extra set of more substantial and long-running tests.

    \item \inlsh{NEKTAR\_TEST\_USE\_HOSTFILE}

    By default, MPI tests are run directly with the \inltt{mpiexec} command
    together with the number of cores. If your MPI installation requires a
    hostfile, enabling this option adds the command line argument
    \inltt{-hostfile hostfile} to the command line arguments when tests are run
    with \inltt{ctest} or the \inltt{Tester} executable.
\end{itemize}

We have recently added explicit support to SIMD (Single Instruction Multiple Data) x86 instruction set extensions (i.e. AVX2, AVX512).
Selected operators (the matrix free operators) utilize the SIMD types, if none of them is enabled these operators default to scalar types.
The various extensions available are marked as advanced options (to visualize them in the cmake gui you need to press the t-button):
\begin{itemize}
    \item \inlsh{NEKTAR\_ENABLE\_SIMD\_AVX2}

    Enables 256 bit wide vector types and set the appropriate compiler flags (gcc only).

    \item \inlsh{NEKTAR\_ENABLE\_SIMD\_AVX512}

    Enables 512 bit wide vector types and set the appropriate compiler flags (gcc only).
\end{itemize}

Note that if you are not configuring cmake for the first time, you need to delete the cached variable \inlsh{CMAKE\_CXX\_FLAGS} in order for the appropriate flags to be set. Alternatively you can manually set the flag to target the appropriate architecture.
