\chapter{Command-line Options}

%\begin{lstlisting}
%--verbose
%\end{lstlisting}
\lstinline[style=BashInputStyle]{--verbose}\\
\hangindent=1.5cm
Displays extra info.

\lstinline[style=BashInputStyle]{--version}\\
\hangindent=1.5cm
Displays software version, and source control information if applicable.

\lstinline[style=BashInputStyle]{--help}\\
\hangindent=1.5cm
Displays help information about the available command-line options for the executable.

\lstinline[style=BashInputStyle]{--parameter [key]=[value]}\\
\hangindent=1.5cm
Override a parameter (or define a new one) specified in the XML file.

\lstinline[style=BashInputStyle]{--solverinfo [key]=[value]}\\
\hangindent=1.5cm
Override a solverinfo (or define a new one) specified in the XML file.

\lstinline[style=BashInputStyle]{--io-format [format]}\\
\hangindent=1.5cm Determines the output format for writing \nekpp field files
that are used to store, for example, checkpoint and solution field files. The
default for \inlsh{format} is \inlsh{Xml}, which is an XML-based format, which
is written as one file per process. If \nekpp is compiled with HDF5 support,
then an alternative option is \inlsh{Hdf5}, which will write one file for all
processes and can be more efficient for very large-scale parallel jobs.

\lstinline[style=BashInputStyle]{--npx [int]}\\
\hangindent=1.5cm
When using a fully-Fourier expansion, specifies the number of processes to use in the x-coordinate direction.

\lstinline[style=BashInputStyle]{--npy [int]}\\
\hangindent=1.5cm
\quad When using a fully-Fourier expansion or 3D expansion with two Fourier directions, specifies the number of processes to use in the y-coordinate direction.

\lstinline[style=BashInputStyle]{--npz [int]}\\
\hangindent=1.5cm
When using Fourier expansions, specifies the number of processes to use in the z-coordinate direction.

\lstinline[style=BashInputStyle]{--part-info}\\
\hangindent=1.5cm
Prints detailed information about the generated partitioning, such as number of
elements, number of local degrees of freedom and the number of boundary degrees
of freedom.

\lstinline[style=BashInputStyle]{--part-only [int]}\\
\hangindent=1.5cm
Partition the mesh only into the specified number of partitions, write to file
and exit. This can be used to pre-partition a very large mesh on a single
high-memory node, prior to being executed on a multi-node cluster.

\lstinline[style=BashInputStyle]{--use-metis}\\
\hangindent=1.5cm
Forces the use of METIS for mesh partitioning. Requires the
\inltt{NEKTAR\_USE\_METIS} option to be set.

\lstinline[style=BashInputStyle]{--use-scotch}\\
\hangindent=1.5cm
Forces the use of Scotch for mesh partitioning. If \nekpp{} is compiled with
METIS support, the default is to use METIS.

\lstinline[style=BashInputStyle]{--use-hdf5-node-comm}\\
\hangindent=1.5cm
Partition the \inlsh{Hdf5}-format mesh in parallel to avoid one single thread runs out of memory in serial partitioning.

\lstinline[style=BashInputStyle]{--set-start-time [float]}\\
\hangindent=1.5cm
Set the starting time of the simulation. This overwrites the time in the file-type initial condition.

\lstinline[style=BashInputStyle]{--set-start-chknumber [int]}\\
\hangindent=1.5cm
Set the starting number of the checkpoint file. This overwrites the checkpoint number in the file-type initial condition.
