\section{Filters}

Filters are a method for calculating a variety of useful quantities from the
field variables as the solution evolves in time, such as time-averaged fields
and extracting the field variables at certain points inside the domain. Each
filter is defined in a \inltt{FILTER} tag inside a \inltt{FILTERS} block which
lies in the main \inltt{NEKTAR} tag. In this section we give an overview of the
modules currently available and how to set up these filters in the session file.

Here is an example \inltt{FILTER}:

\begin{lstlisting}[style=XMLStyle,gobble=2]
  <FILTER TYPE="FilterName">
      <PARAM NAME="Param1"> Value1 </PARAM>
      <PARAM NAME="Param2"> Value2 </PARAM>
  </FILTER>
\end{lstlisting}

A filter has a name -- in this case, \inltt{FilterName} -- together with
parameters which are set to user-defined values. Each filter expects different
parameter inputs, although where functionality is similar, the same parameter
names are shared between filter types for consistency. Numerical filter
parameters may be expressions and so may include session parameters defined in
the \inltt{PARAMETERS} section.

In the following we document the filters implemented. Note that some filters are
solver-specific and will therefore only work for a given subset of the available
solvers.

\subsection{FieldConvert checkpoints}

\begin{notebox}
  This filter is still at an experimental stage. Not all modules and options
  from FieldConvert are supported.
\end{notebox}

This filter applies a sequence of FieldConvert modules to the solution, 
writing an output file. An output is produced at the end of the simulation into
\inltt{session\_fc.fld}, or alternatively every $M$ timesteps as defined by the
user, into a sequence of files \inltt{session\_*\_fc.fld}, where \inltt{*} is
replaced by a counter.

The following parameters are supported:

\begin{center}
  \begin{tabularx}{0.99\textwidth}{lllX}
    \toprule
    \textbf{Option name} & \textbf{Required} & \textbf{Default} & 
    \textbf{Description} \\
    \midrule
    \inltt{OutputFile}      & \xmark   & \texttt{session.fld} &
    Output filename. If no extension is provided, it is assumed as .fld\\
    \inltt{OutputFrequency} & \xmark   & \texttt{NumSteps} &
    Number of timesteps after which output is written, $M$.\\
    \inltt{Modules} & \xmark   &  &
    FieldConvert modules to run, separated by a white space.\\
    \bottomrule
  \end{tabularx}
\end{center}

As an example, consider:

\begin{lstlisting}[style=XMLStyle,gobble=2]
  <FILTER TYPE="FieldConvert">
      <PARAM NAME="OutputFile">MyFile.vtu</PARAM>
      <PARAM NAME="OutputFrequency">100</PARAM>
      <PARAM NAME="Modules"> vorticity isocontour:fieldid=0:fieldvalue=0.1 </PARAM>
  </FILTER>
\end{lstlisting}

This will create a sequence of files named \inltt{MyFile\_*\_fc.vtu} containing isocontours. 
The result will be output every 100 time steps.

\subsection{Time-averaged fields}

This filter computes time-averaged fields for each variable defined in the
session file. Time averages are computed by either taking a snapshot of the
field every timestep, or alternatively at a user-defined number of timesteps
$N$. An output is produced at the end of the simulation into
\inltt{session\_avg.fld}, or alternatively every $M$ timesteps as defined by the
user, into a sequence of files \inltt{session\_*\_avg.fld}, where \inltt{*} is
replaced by a counter. This latter option can be useful to observe statistical
convergence rates of the averaged variables.

This filter is derived from FieldConvert filter, and therefore support all parameters
available in that case. The following additional parameter is supported:

\begin{center}
  \begin{tabularx}{0.99\textwidth}{lllX}
    \toprule
    \textbf{Option name} & \textbf{Required} & \textbf{Default} & 
    \textbf{Description} \\
    \midrule
    \inltt{SampleFrequency} & \xmark   & 1 &
    Number of timesteps at which the average is calculated, $N$.\\
    \inltt{RestartFile} & \xmark   &   &
    Restart file used as initial average.
    If no extension is provided, it is assumed as .fld\\
  \end{tabularx}
\end{center}

As an example, consider:

\begin{lstlisting}[style=XMLStyle,gobble=2]
  <FILTER TYPE="AverageFields">
      <PARAM NAME="OutputFile">MyAverageField</PARAM>
      <PARAM NAME="RestartFile">MyRestartAvg.fld</PARAM>
      <PARAM NAME="OutputFrequency">100</PARAM>
      <PARAM NAME="SampleFrequency"> 10 </PARAM>		
  </FILTER>
\end{lstlisting}

This will create a file named \inltt{MyAverageField.fld} averaging the
instantaneous fields every 10 time steps. The averaged field is however only
output every 100 time steps.

\subsection{Moving average of fields}

This filter computes the exponential moving average (in time) of
fields for each variable defined in the session file. The moving average 
is defined as:
\[
\bar{u}_n = \alpha u_n + (1 - \alpha)\bar{u}_{n-1}
\]
with $0 < \alpha < 1$ and $\bar{u}_1 = u_1$.

The same parameters of the time-average filter are supported, with the output file
in the form \inltt{session\_*\_movAvg.fld}. In addition,
either $\alpha$ or the time-constant $\tau$ must be defined. They are related by:
\[
\alpha = \frac{t_s}{\tau + t_s}
\]
where $t_s$ is the time interval between consecutive samples.

As an example, consider:

\begin{lstlisting}[style=XMLStyle,gobble=2]
  <FILTER TYPE="MovingAverage">
      <PARAM NAME="OutputFile">MyMovingAverage</PARAM>
      <PARAM NAME="OutputFrequency">100</PARAM>
      <PARAM NAME="SampleFrequency"> 10 </PARAM>
      <PARAM NAME="tau"> 0.1 </PARAM>
  </FILTER>
\end{lstlisting}

This will create a file named \inltt{MyMovingAverage\_movAvg.fld} with a moving average
sampled every 10 time steps. The averaged field is however only
output every 100 time steps.

\subsection{Reynolds stresses}

\begin{notebox}
  This filter is only supported for the incompressible Navier-Stokes solver.
\end{notebox}

This filter is an extended version of the time-average filter. It outputs
not only the time-average of the fields, but also the Reynolds stresses.
The same parameters supported in the time-average case can be used,
for example:

\begin{lstlisting}[style=XMLStyle,gobble=2]
  <FILTER TYPE="ReynoldsStresses">
      <PARAM NAME="OutputFile">MyAverageField</PARAM>
      <PARAM NAME="RestartFile">MyAverageRst.fld</PARAM>
      <PARAM NAME="OutputFrequency">100</PARAM>
      <PARAM NAME="SampleFrequency"> 10 </PARAM>
  </FILTER>
\end{lstlisting}

By default, this filter uses a simple average. Optionally, an exponential
moving average can be used, in which case the output contains the moving
averages and the Reynolds stresses calculated based on them. For example:

\begin{lstlisting}[style=XMLStyle,gobble=2]
  <FILTER TYPE="ReynoldsStresses">
      <PARAM NAME="OutputFile">MyAverageField</PARAM>
      <PARAM NAME="MovingAverage">true</PARAM>
      <PARAM NAME="OutputFrequency">100</PARAM>
      <PARAM NAME="SampleFrequency"> 10 </PARAM>
      <PARAM NAME="alpha"> 0.01 </PARAM>
  </FILTER>
\end{lstlisting}

\subsection{Checkpoint fields}
 
The checkpoint filter writes a checkpoint file, containing the instantaneous
state of the solution fields at at given timestep. This can subsequently be used
for restarting the simulation or examining time-dependent behaviour. This
produces a sequence of files, by default named \inltt{session\_*.chk}, where
\inltt{*} is replaced by a counter. The initial condition is written to
\inltt{session\_0.chk}. Existing files are not overwritten, but renamed to e.g.
\inltt{session\_0.bak0.chk}. In case this file already exists, too, the \inltt{chk}-file
is renamed to \inltt{session\_0.bak*.chk} and so on.


\begin{notebox}
  This functionality is equivalent to setting the \inltt{IO\_CheckSteps}
  parameter in the session file.
\end{notebox}

The following parameters are supported:

\begin{center}
  \begin{tabularx}{0.99\textwidth}{lllX}
    \toprule
    \textbf{Option name} & \textbf{Required} & \textbf{Default} & 
    \textbf{Description} \\
    \midrule
    \inltt{OutputFile}      & \xmark   & \texttt{session} &
    Prefix of the output filename to which the checkpoints are written.\\
    \inltt{OutputFrequency} & \cmark   & - &
    Number of timesteps after which output is written.\\
    \bottomrule
  \end{tabularx}
\end{center}

For example, to output the fields every 100 timesteps we can specify:

\begin{lstlisting}[style=XMLStyle,gobble=2]
  <FILTER TYPE="Checkpoint">
      <PARAM NAME="OutputFile">IntermediateFields</PARAM>
      <PARAM NAME="OutputFrequency">100</PARAM>
  </FILTER>
\end{lstlisting}
 
\subsection{History points}

The history points filter can be used to evaluate the value of the fields in
specific points of the domain as the solution evolves in time. By default this 
produces a file called \inltt{session.his}. For each timestep, and then each 
history point, a line is output containing the current solution time, followed 
by the value of each of the field variables. Commented lines are created at the
top of the file containing the location of the history points and the order of 
the variables.

The following parameters are supported:

\begin{center}
  \begin{tabularx}{0.99\textwidth}{lllX}
    \toprule
    \textbf{Option name} & \textbf{Required} & \textbf{Default} & 
    \textbf{Description} \\
    \midrule
    \inltt{OutputFile}      & \xmark   & \texttt{session} &
    Prefix of the output filename to which the checkpoints are written.\\
    \inltt{OutputFrequency} & \xmark   & 1 &
    Number of timesteps after which output is written.\\
    \inltt{OutputPlane}     & \xmark   & 0 &
    If the simulation is homogeneous, the plane on which to evaluate the 
    history point. (No Fourier interpolation is currently implemented.)\\
    \inltt{Points      }    & \cmark   & - &
    A list of the history points. These should always be given in three
    dimensions. \\
    \bottomrule
  \end{tabularx}
\end{center}

For example, to output the value of the solution fields at three points
$(1,0.5,0)$, $(2,0.5,0)$ and $(3,0.5,0)$ into a file \inltt{TimeValues.his}
every 10 timesteps, we use the syntax:

\begin{lstlisting}[style=XMLStyle,gobble=2]
  <FILTER TYPE="HistoryPoints">
      <PARAM NAME="OutputFile">TimeValues</PARAM>
      <PARAM NAME="OutputFrequency">10</PARAM>
      <PARAM NAME="Points">
          1 0.5 0
          2 0.5 0
          3 0.5 0
      </PARAM>
  </FILTER>
\end{lstlisting}

\subsection {ThresholdMax}

The threshold value filter writes a field output containing a variable $m$,
defined by the time at which the selected variable first exceeds a specified
threshold value. The default name of the output file is the name of the session
with the suffix \inlsh{\_max.fld}. Thresholding is applied based on the first
variable listed in the session by default.

The following parameters are supported:

\begin{center}
  \begin{tabularx}{0.99\textwidth}{lllX}
    \toprule
    \textbf{Option name} & \textbf{Required} & \textbf{Default} & 
    \textbf{Description} \\
    \midrule
    \inltt{OutputFile}      & \xmark   & \emph{session}\_max.fld &
    Output filename to which the threshold times are written.\\
    \inltt{ThresholdVar}    & \xmark   & \emph{first variable name} &
    Specifies the variable on which the threshold will be applied.\\
    \inltt{ThresholdValue}  & \cmark   & - &
    Specifies the threshold value.\\
    \inltt{InitialValue}    & \cmark   & - &
    Specifies the initial time.\\
    \inltt{StartTime}       & \xmark   & 0.0 &
    Specifies the time at which to start recording.\\
    \bottomrule
  \end{tabularx}
\end{center}
 
An example is given below:
 
\begin{lstlisting}[style=XMLStyle]
  <FILTER TYPE="ThresholdMax">
      <PARAM NAME="OutputFile"> threshold_max.fld </PARAM>
      <PARAM NAME="ThresholdVar"> u </PARAM>
      <PARAM NAME="ThresholdValue"> 0.1 </PARAM>
      <PARAM NAME="InitialValue">  0.4 </PARAM>
  </FILTER>
\end{lstlisting}

which produces a field file \inlsh{threshold\_max.fld}.

\subsection{ThresholdMin value}

Performs the same function as the \inltt{ThresholdMax} filter but records the
time at which the threshold variable drops below a prescribed value.

\subsection{One-dimensional energy}

This filter is designed to output the energy spectrum of one-dimensional
elements. It transforms the solution field at each timestep into a orthogonal
basis defined by the functions
\[
\psi_p(\xi) = L_p(\xi)
\]
where $L_p$ is the $p$-th Legendre polynomial. This can be used to show the
presence of, for example, oscillations in the underlying field due to numerical
instability. The resulting output is written into a file called
\inltt{session.eny} by default. The following parameters are supported:

\begin{center}
  \begin{tabularx}{0.99\textwidth}{lllX}
    \toprule
    \textbf{Option name} & \textbf{Required} & \textbf{Default} &
    \textbf{Description} \\
    \midrule
    \inltt{OutputFile}      & \xmark   & \inltt{session} &
    Prefix of the output filename to which the energy spectrum is written.\\
    \inltt{OutputFrequency} & \xmark   & 1 &
    Number of timesteps after which output is written.\\
    \bottomrule
  \end{tabularx}
\end{center}

An example syntax is given below:

\begin{lstlisting}[style=XMLStyle,gobble=2]
  <FILTER TYPE="Energy1D">
      <PARAM NAME="OutputFile">EnergyFile</PARAM>
      <PARAM NAME="OutputFrequency">10</PARAM>
  </FILTER>
\end{lstlisting}

\subsection{Modal energy}

\begin{notebox}
  This filter is only supported for the incompressible Navier-Stokes solver.
\end{notebox}

This filter calculates the time-evolution of the kinetic energy. In the case of
a two- or three-dimensional simulation this is defined as
\[
E_k(t) = \frac{1}{2} \int_{\Omega} \|\mathbf{u}\|^2\, dx
\]
However if the simulation is written as a one-dimensional homogeneous expansion
so that
\[
\mathbf{u}(\mathbf{x},t) = \sum_{k=0}^N \mathbf{\hat{u}}_k(t)e^{2\pi ik\mathbf{x}}
\]
then we instead calculate the energy spectrum
\[
E_k(t) = \frac{1}{2} \int_{\Omega} \|\mathbf{\hat{u}}_k\|^2\, dx.
\]
Note that in this case, each component of $\mathbf{\hat{u}}_k$ is a complex
number and therefore $N = \inltt{HomModesZ}/2$ lines are output for each
timestep. This is a particularly useful tool in examining turbulent and
transitional flows which use the homogeneous extension. In either case, the
resulting output is written into a file called \inltt{session.mdl} by default.

The following parameters are supported:

\begin{center}
  \begin{tabularx}{0.99\textwidth}{lllX}
    \toprule
    \textbf{Option name} & \textbf{Required} & \textbf{Default} & 
    \textbf{Description} \\
    \midrule
    \inltt{OutputFile}      & \xmark   & \inltt{session} &
    Prefix of the output filename to which the energy spectrum is written.\\
    \inltt{OutputFrequency} & \xmark   & 1 &
    Number of timesteps after which output is written.\\
    \bottomrule
  \end{tabularx}
\end{center}

An example syntax is given below:

\begin{lstlisting}[style=XMLStyle,gobble=2]
  <FILTER TYPE="ModalEnergy">
      <PARAM NAME="OutputFile">EnergyFile</PARAM>
      <PARAM NAME="OutputFrequency">10</PARAM>
  </FILTER>
\end{lstlisting}

\subsection{Aerodynamic forces}

\begin{notebox}
  This filter is only supported for the incompressible Navier-Stokes solver.
\end{notebox}

This filter evaluates the aerodynamic forces along a specific surface. The
forces are projected along the Cartesian axes and the pressure and viscous
contributions are computed in each direction.

The following parameters are supported:

\begin{center}
  \begin{tabularx}{0.99\textwidth}{lllX}
    \toprule
    \textbf{Option name} & \textbf{Required} & \textbf{Default} & 
    \textbf{Description} \\
    \midrule
    \inltt{OutputFile}      & \xmark   & \inltt{session} &
    Prefix of the output filename to which the forces are written.\\
    \inltt{Frequency}       & \xmark   & 1 &
    Number of timesteps after which output is written.\\
    \inltt{Boundary}        & \cmark   & - &
    Boundary surfaces on which the forces are to be evaluated.\\
    \bottomrule
  \end{tabularx}
\end{center}

An example is given below:

\begin{lstlisting}[style=XMLStyle]
  <FILTER TYPE="AeroForces">
      <PARAM NAME="OutputFile">DragLift.frc</PARAM>
      <PARAM NAME="OutputFrequency">10</PARAM>
      <PARAM NAME="Boundary"> B[1,2] </PARAM>		
  </FILTER>
\end{lstlisting}

During the execution a file named \inltt{DragLift.frc} will be created and the
value of the aerodynamic forces on boundaries 1 and 2, defined in the
\inltt{GEOMETRY} section, will be output every 10 time steps.

\subsection{Kinetic energy and enstrophy}

\begin{notebox}
  This filter is only supported for the incompressible and compressible
  Navier-Stokes solvers \textbf{in three dimensions}.
\end{notebox}

The purpose of this filter is to calculate the kinetic energy and enstrophy
%
\[
E_k = \frac{1}{2\mu(\Omega)}\int_{\Omega} \|\mathbf{u}\|^2\, dx, \qquad
\mathcal{E} = \frac{1}{2\mu(\Omega)}\int_{\Omega} \|\mathbf{\omega}\|^2\, dx
\]
%
where $\mu(\Omega)$ is the volume of the domain $\Omega$. This produces a file
containing the time-evolution of the kinetic energy and enstrophy fields. By
default this file is called \inltt{session.eny} where \inltt{session} is the
session name.

The following parameters are supported:
%
\begin{center}
  \begin{tabularx}{0.99\textwidth}{lllX}
    \toprule
    \textbf{Option name} & \textbf{Required} & \textbf{Default} & 
    \textbf{Description} \\
    \midrule
    \inltt{OutputFile}      & \xmark   & \texttt{session.eny} &
    Output file name to which the energy and enstrophy are written.\\
    \inltt{OutputFrequency} & \cmark   & - &
    Number of timesteps at which output is written.\\
    \bottomrule
  \end{tabularx}
\end{center}
%
To enable the filter, add the following to the \inltt{FILTERS} tag:
%
\begin{lstlisting}[style=XMLStyle,gobble=2]
  <FILTER TYPE="Energy">
      <PARAM NAME="OutputFrequency"> 1 </PARAM>
  </FILTER>
\end{lstlisting}
