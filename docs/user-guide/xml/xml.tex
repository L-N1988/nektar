\chapter{XML Session File}
\label{s:xml}

The Nektar++ native file format is compliant with XML version 1.0. The root
element is NEKTAR which contains a number of other elements which describe
configuration for different aspects of the simulation. The required elements are
shown below:
\begin{lstlisting}[style=XMLStyle]
<NEKTAR>
  <GEOMETRY>
    ...
  </GEOMETRY>
  <EXPANSIONS>
    ...
  </EXPANSIONS>
  <CONDITIONS>
    ...
  </CONDITIONS>
  ...
</NEKTAR>
\end{lstlisting}
The different sub-elements can be split across multiple files, however each
file must have a top-level NEKTAR tag. For example, one might store the
geometry information separate from the remaining configuration in two separate
files as illustrated below:

\inlsh{geometry.xml}
\begin{lstlisting}[style=XMLStyle]
<NEKTAR>
  <GEOMETRY>
    ...
  </GEOMETRY>
</NEKTAR>
\end{lstlisting}

\inlsh{conditions.xml}
\begin{lstlisting}[style=XMLStyle]
<NEKTAR>
  <CONDITIONS>
    ...
  </CONDITIONS>
  <EXPANSIONS>
    ...
  </EXPANSIONS>
  ...
</NEKTAR>
\end{lstlisting}

\begin{notebox}
    When specifying multiple files, repeated XML sub-elements are not merged.
    The sub-elements from files appearing later in the list will, in general,
    override those elements from earlier files.
    \medskip
    
    For example, the \inlsh{NekMesh} utility will produce a default
    \inltt{EXPANSIONS} element and blank \inltt{CONDITIONS} element. Specifying
    a custom-written XML file containing these sections \emph{after} the
    file produced by \inlsh{NekMesh} will override these defaults.

    The exception to this rule is when an empty XML sub-element would override a
    non-empty XML sub-element. In this case the empty XML sub-element will be
    ignored. If the custom-written XML file containing \inltt{CONDITIONS} were
    specified before the file produced by \inlsh{NEKMESH}, the empty
    \inltt{CONDITIONS} tag in the latter file would be ignored.
\end{notebox}

\section{Geometry}
This section defines the mesh. It specifies a list of vertices, edges (in two or
three dimensions) and faces (in three dimensions) and how they connect to create
the elemental decomposition of the domain. It also defines a list of composites
which are used in the Expansions and Conditions sections of the file to describe
the polynomial expansions and impose boundary conditions.

The GEOMETRY section is structured as
\begin{lstlisting}[style=XMLStyle]
<GEOMETRY DIM="2" SPACE="2">
  <VERTEX> ... </VERTEX>
  <EDGE> ... </EDGE>
  <FACE> ... </FACE>
  <ELEMENT> ... </ELEMENT>
  <CURVED> ... </CURVED>
  <COMPOSITE> ... </COMPOSITE>
  <DOMAIN> ... </DOMAIN>
</GEOMETRY>
\end{lstlisting}
It has two (required) attributes:
\begin{itemize}
    \item \inltt{DIM} specifies the dimension of the expansion elements.
    \item \inltt{SPACE} specifies the dimension of the space in which the
    elements exist.
\end{itemize}
These attributes allow, for example, a two-dimensional surface to be embedded in
a three-dimensional space.

\begin{notebox}
The attribute \inltt{PARTITION} may also appear in a partitioned mesh.
However, this attribute should not be explicitly specified by the user.
\end{notebox}

The contents of each of the \inltt{VERTEX}, \inltt{EDGE}, \inltt{FACE}, \inltt{ELEMENT}
and \inltt{CURVED} sections may optionally be compressed and stored in
base64-encoded gzipped binary form, using either little-endian or big-endian
ordering, as specified by the \inltt{COMPRESSED} attribute to these sections.
Currently supported values are:
\begin{itemize}
    \item \inltt{B64Z-LittleEndian}: Base64 Gzip compressed using little-endian
        ordering.
    \item \inltt{B64Z-BigEndian}: Base64 Gzip compressed using big-endian
        ordering.
\end{itemize}
When generating mesh input files for \nekpp using \inltt{NekMesh}, the binary compressed form will be used by default. To convert a compressed XML file into
human-readable ASCII format use, for example:
\begin{lstlisting}[style=BashInputStyle]
NekMesh file.msh newfile.xml:xml:uncompress
\end{lstlisting}

\begin{notebox}
The description in the remainder of this section explains how the \inltt{GEOMETRY} section is laid out in uncompressed ASCII format.
\end{notebox}


\subsection{Vertices}

Vertices have three coordinates. Each has a unique vertex ID. In
uncompressed form, they are defined within \inltt{VERTEX} subsection as follows:
\begin{lstlisting}[style=XMLStyle] <VERTEX>
<V ID="0"> 0.0  0.0  0.0 </V> ...
\end{lstlisting}
The \inltt{VERTEX} subsection has optional attributes which can be used to
apply a transformation to the mesh:\\
\inltt{XSCALE}, \inltt{YSCALE}, \inltt{ZSCALE},
\inltt{XMOVE}, \inltt{YMOVE}, \inltt{ZMOVE}

They specify scaling factors (centred at the origin) and translations to the
vertex coordinates. For example, the following snippet
\begin{lstlisting}[style=XMLStyle]
<VERTEX XSCALE="5">
  <V ID="0"> 0.0  0.0  0.0 </V>
  <V ID="1"> 1.0  2.0  0.0 </V>
</VERTEX>
\end{lstlisting}
defines two vertices with coordinates $(0.0,0.0,0.0), (1.0,2.0,0.0)$.

All of these attributes can be arbitrary analytic expressions depending on pre-
defined constants and parameters defined in the XML file and
mathematical operations/functions of the latter. If omitted, default scaling
factors 1.0, and translations of 0.0, are assumed.



\subsection{Edges}
\begin{tipbox}
    The \inltt{EDGES} section is only necessary when \inltt{DIM=2} or
    \inltt{DIM=3} in the parent \inltt{GEOMETRY} element and may be omitted for
    one-dimensional meshes.
\end{tipbox}

Edges are defined by two vertices. Each edge has a unique edge ID. In
uncompressed form, they are defined in the file with a line of the form
\begin{lstlisting}[style=XMLStyle]
<E ID="0"> 0 1 </E>
\end{lstlisting}


\subsection{Faces}
\begin{tipbox}
    The \inltt{FACES} section is only necessary when \inltt{DIM=3} in the
    parent \inltt{GEOMETRY} element and may otherwise be omitted.
\end{tipbox}

Faces are defined by three or more edges. Each face has a unique face ID. They
are defined in the file with a line of the form
\begin{lstlisting}[style=XMLStyle]
<T ID="0"> 0 1 2 </T>
<Q ID="1"> 3 4 5 6 </Q>
\end{lstlisting}
The choice of tag specified (T or Q), and thus the number of edges specified depends on the geometry of the face (triangle or quadrilateral).


\subsection{Element}
Elements define the top-level geometric entities in the mesh. Their definition depends upon the dimension of the expansion. For two-dimensional expansions, an element is defined by a sequence of three or four edges. For three-dimensional expansions, the element is defined by a list of faces. Elements are defined in the file with a line of the form
\begin{lstlisting}[style=XMLStyle]
<T ID="0"> 0 1 2 </T>
<H ID="1"> 3 4 5 6 7 8 </H>
\end{lstlisting}
Again, the choice of tag specified depends upon the geometry of the element. The element tags are:

\begin{itemize}
    \item \inltt{S} Segment
    \item \inltt{T} Triangle
    \item \inltt{Q} Quadrilateral
    \item \inltt{A} Tetrahedron
    \item \inltt{P} Pyramid
    \item \inltt{R} Prism
    \item \inltt{H} Hexahedron
\end{itemize}


\subsection{Curved Edges and Faces}
\begin{tipbox}
    The \inltt{CURVED} section is only necessary if curved edges or faces are
    present in the mesh and may otherwise be omitted.
\end{tipbox}

For mesh elements with curved edges and/or curved faces, a separate entry is used to describe the control points for the curve. Each curve has a unique curve ID and is associated with a predefined edge or face. The total number of points in the curve (including end points) and their distribution is also included as attributes. The control points are listed in order, each specified by three coordinates. Curved edges are defined in the file with a line of the form
\begin{lstlisting}[style=XMLStyle]
<E ID="3" EDGEID="7" TYPE="PolyEvenlySpaced" NUMPOINTS="3">
    0.0  0.0  0.0    0.5  0.5  0.0    1.0  0.0  0.0
</E>
\end{lstlisting}

\begin{notebox}
    In the compressed form, this section contains different sub-elements to
    efficiently encode the high-order curvature data. This is not described
    further in this document.
\end{notebox}

\subsection{Composites}
Composites define collections of elements, faces or edges. Each has a unique composite ID associated with it. All components of a composite entry must be of the same type. The syntax allows components to be listed individually, using ranges, or a mixture of the two. Examples include
\begin{lstlisting}[style=XMLStyle]
<C ID="0"> T[0-862] </C>
<C ID="1"> E[61-67,69,70,72-74] </C>
\end{lstlisting}

The composites can also optionally contain a name which is then used in the
multi-block VTK output to label the block descriptively rather than by ID, for
example
\begin{lstlisting}[style=XMLStyle]
<C NAME="Main domain" ID="0"> T[0-862] </C>
<C NAME="Walls" ID="1"> E[61-67,69,70,72-74] </C>
\end{lstlisting}

\subsection{Domain}
This tag specifies composites which describe the entire problem domain. It has the form of
\begin{lstlisting}[style=XMLStyle]
<DOMAIN> C[0] </DOMAIN>
\end{lstlisting}


\section{Expansions}
\label{sec:xml:expansions}
This section defines the polynomial expansions used on each of the defined
geometric composites and variables. Expansion entries specify the number of
modes and the expansion type, or a full list of data of basis type, number of
modes, points type and number of points. The short-hand version has the
following form

\begin{lstlisting}[style=XMLStyle]
<E COMPOSITE="C[0]" NUMMODES="5" FIELDS="u" TYPE="MODIFIED" />
\end{lstlisting}

or, if we have more then one variable we can apply the same basis to all using

\begin{lstlisting}[style=XMLStyle]
<E COMPOSITE="C[0]" NUMMODES="5" FIELDS="u,v,p" TYPE="MODIFIED" />
\end{lstlisting}

The expansion basis can also be specified in detail as a combination of
one-dimensional bases, and thus the user is able to, for example, increase the
quadrature order. For tet elements this takes the form:

\begin{lstlisting}[style=XMLStyle]
<E COMPOSITE="C[0]" 
   BASISTYPE="Modified_A,Modified_B,Modified_C" 
   NUMMODES="3,3,3"
   POINTSTYPE="GaussLobattoLegendre,GaussRadauMAlpha1Beta0,GaussRadauMAlpha2Beta0"
   NUMPOINTS="4,3,3"
   FIELDS="u" />
\end{lstlisting}

and for prism elements:

\begin{lstlisting}[style=XMLStyle]
<E COMPOSITE="C[1]" 
   BASISTYPE="Modified_A,Modified_A,Modified_B" 
   NUMMODES="3,3,3"
   POINTSTYPE="GaussLobattoLegendre,GaussLobattoLegendre,GaussRadauMAlpha1Beta0"
   NUMPOINTS="4,4,3"
   FIELDS="u" />
\end{lstlisting}

The expansions can be defined with a list of \inltt{<E>} elements (e.g., to
represent different polynomial orders for different variables or to address
different composites). The user can define a default expansion field by entering
\inltt{<E>} tags without the \inltt{FIELDS} attribute. The default expansion is
used to define any variables not explicitly listed in the \inltt{<E>} entries.
In the following example, the default expansion is used to define the expansions
for the composites C[0], C[1] and C[2]:

\begin{lstlisting}[style=XMLStyle]
<E COMPOSITE="C[0-2]" NUMMODES="5" TYPE="MODIFIED" />
<E COMPOSITE="C[3]"   NUMMODES="4" TYPE="MODIFIED" FIELDS="u,v"/>
<E COMPOSITE="C[3]"   NUMMODES="3" TYPE="MODIFIED" FIELDS="p"/>
\end{lstlisting}

The expansions of each field should be defined only once for each composite.


\section{Conditions}
This section of the file defines parameters and boundary conditions which
define the nature of the problem to be solved. These are enclosed in the
\inltt{CONDITIONS} tag.

\subsection{Parameters}

Numerical parameters may be required by a particular solver (for instance time-integration or physical parameters), or may be arbitrary and only used for the purpose of simplifying the problem specification in the session file (e.g. parameters which would otherwise be repeated in the definition of an initial
condition and boundary conditions). All parameters are enclosed in the \inltt{PARAMETERS} XML element.

\begin{lstlisting}[style=XMLStyle] 
<PARAMETERS>
    ...
</PARAMETERS>
\end{lstlisting}

A parameter may be of integer or real type and may reference other parameters
defined previous to it. It is expressed in the file as

\begin{lstlisting}[style=XMLStyle]
<P> [PARAMETER NAME] = [PARAMETER VALUE] </P>
\end{lstlisting}

For example,

\begin{lstlisting}[style=XMLStyle]
<P> NumSteps = 1000              </P>
<P> TimeStep = 0.01              </P>
<P> FinTime  = NumSteps*TimeStep </P>
\end{lstlisting}

A number of pre-defined constants may also be used in parameter expressions, for example \texttt{PI}. A full list of supported constants is provided in Section~\ref{sec:xml:expressions:syntax}.

\subsection{Time Integration Scheme}

These specify properties to define the parameters specific to the time
integration scheme to be used. The parameters are specified as XML
elements and require a string corresponding to the time-stepping method
and the order, and optionally the variant and free parameters. For
example,

\begin{lstlisting}[style=XMLStyle]
 <TIMEINTEGRATIONSCHEME>
   <METHOD> IMEX </METHOD>
   <VARIANT> DIRK </VARIANT>
   <ORDER> 2 </ORDER>
   <FREEPARAMETERS> 2 3 </FREEPARAMETERS>
 </TIMEINTEGRATIONSCHEME>
\end{lstlisting}

\subsection{Solver Information}

These specify properties to define the actions specific to solvers,
typically including the equation to solve and the projection type. The
property/value pairs are specified as XML attributes. For example,

\begin{lstlisting}[style=XMLStyle] 
<SOLVERINFO>
  <I PROPERTY="EQTYPE"                VALUE="UnsteadyAdvection"    /> 
  <I PROPERTY="Projection"            VALUE="Continuous"           /> 
</SOLVERINFO>
\end{lstlisting}

Boolean-valued solver properties are specified using \inltt{True} or \inltt{False}. The list of available solvers in Nektar++ can be found in
Part~\ref{p:applications}.

\subsubsection{Drivers}
Drivers are defined under the \inltt{CONDITIONS} section as properties of the 
\inltt{SOLVERINFO} XML element. The role of a driver is to manage the solver 
execution from an upper level. 

The default driver is called \inltt{Standard} and executes the following steps:
\begin{enumerate}
\item Prints out on screen a summary of all the conditions defined in the input file.
\item Sets up the initial and boundary conditions.
\item Calls the solver defined by \inltt{SolverType}  in the \inltt{SOLVERINFO} XML element.
\item Writes the results in the output (.fld) file.
\end{enumerate}

In the following example, the driver \inltt{Standard} is used to manage the 
execution of the incompressible Navier-Stokes equations:
\begin{lstlisting}[style=XMLStyle]
 <TIMEINTEGRATIONSCHEME>
    <METHOD> IMEX </METHOD>
    <ORDER> 2 </ORDER>
 </TIMEINTEGRATIONSCHEME>

 <SOLVERINFO>
    <I PROPERTY="EQTYPE"                VALUE="UnsteadyNavierStokes"     />
    <I PROPERTY="SolverType"            VALUE="VelocityCorrectionScheme" />
    <I PROPERTY="Projection"            VALUE="Galerkin"                 />
    <I PROPERTY="Driver"                VALUE="Standard"                 />
</SOLVERINFO>
\end{lstlisting}

If no driver is specified in the session file, the driver \inltt{Standard} is 
called by default. Other drivers can be used and are typically focused on
specific applications. As described in Sec.
\ref{SectionIncNS_SolverInfo} and  \ref{SectionIncNS_SolverInfo_Stab}, 
the other possibilities are:
\begin{itemize}
\item \inltt{ModifiedArnoldi}  - computes of the leading eigenvalues and 
eigenmodes using modified Arnoldi method.
\item \inltt{Arpack} - computes of eigenvalues/eigenmodes using Implicitly 
Restarted Arnoldi Method (ARPACK).
\item \inltt{SteadyState} - uses the Selective Frequency Damping method 
(see Sec. \ref{SectionSFD}) to obtain a steady-state solution of the 
Navier-Stokes equations (compressible or incompressible).
\end{itemize}


\subsection{Variables}

These define the number (and name) of solution variables. Each variable is
prescribed a unique ID. For example a two-dimensional flow simulation may define
the velocity variables using

\begin{lstlisting}[style=XMLStyle]
<VARIABLES>
  <V ID="0"> u </V>
  <V ID="1"> v </V>
</VARIABLES>
\end{lstlisting}

\subsection{Global System Solution Algorithm}

Many \nekpp solvers use an implicit formulation of their equations to, for
instance, improve timestep restrictions. This means that a large matrix system
must be constructed and a global system set up to solve for the unknown
coefficients. There are several approaches in the spectral/$hp$ element method
that can be used in order to improve efficiency in these methods, as well as
considerations as to whether the simulation is run in parallel or serial. \nekpp
opts for `sensible' default choices, but these may or may not be optimal
depending on the problem under consideration.

This section of the XML file therefore allows the user to specify the global
system solution parameters, which dictates the type of solver to be used for any
implicit systems that are constructed. This section is particularly useful when
using a multi-variable solver such as the incompressible Navier-Stokes solver,
as it allows us to select different preconditioning and residual convergence
options for each variable. As an example, consider the block defined by:

\begin{lstlisting}[style=XMLStyle]
<GLOBALSYSSOLNINFO>
  <V VAR="u,v,w">
    <I PROPERTY="GlobalSysSoln"             VALUE="IterativeStaticCond" />
    <I PROPERTY="Preconditioner"            VALUE="LowEnergyBlock"/>
    <I PROPERTY="IterativeSolverTolerance"  VALUE="1e-8"/>
  </V>
  <V VAR="p">
    <I PROPERTY="GlobalSysSoln"             VALUE="IterativeStaticCond" />
    <I PROPERTY="Preconditioner"     VALUE="FullLinearSpaceWithLowEnergyBlock"/>
    <I PROPERTY="IterativeSolverTolerance"  VALUE="1e-6"/>
  </V>
</GLOBALSYSSOLNINFO>
\end{lstlisting}

The above section specifies that the variables \texttt{u,v,w} should use the
\texttt{IterativeStaticCond} global solver alongside the \texttt{LowEnergyBlock}
preconditioner and an iterative tolerance of $10^{-8}$ on the residuals. However
the pressure variable \texttt{p} is generally stiffer: we therefore opt for a
more expensive \texttt{FullLinearSpaceWithLowEnergyBlock} preconditioner and a
larger residual of $10^{-6}$. We now outline the choices that one can use for
each of these parameters and give a brief description of what they mean.

Defaults for all fields can be defined by setting the equivalent property in
the \texttt{SOLVERINFO} section. Parameters defined in this section will
override any options specified there.


\subsubsection{\texttt{GlobalSysSoln} options}

\nekpp presently implements four methods of solving a global system:

\begin{itemize}
  \item \textbf{Direct} solvers construct the full global matrix and directly
  invert it using an appropriate matrix technique, such as Cholesky
  factorisation, depending on the properties of the matrix. Direct solvers
  \textbf{only} run in serial.
  \item \textbf{Iterative} solvers instead apply matrix-vector multiplications
  repeatedly, using the conjugate gradient method, to converge to a solution to
  the system. For smaller problems, this is typically slower than a direct
  solve. However, for larger problems it can be used to solve the system in
  parallel execution.
  \item \textbf{Xxt} solvers use the $XX^T$ library to perform a parallel direct
  solve. This option is only available if the \texttt{NEKTAR\_USE\_MPI} option
  is enabled in the CMake configuration.
  \item \textbf{PETSc} solvers use the PETSc library, giving access to a wide
  range of solvers and preconditioners. See section~\ref{sec:petsc} below for
  some additional information on how to use the PETSc solvers. This option is
  only available if the \texttt{NEKTAR\_USE\_PETSC} option is enabled in the
  CMake configuration.
\end{itemize}

\begin{warningbox}
Both the \textbf{Xxt} and \textbf{PETSc} solvers are considered advanced and are
under development -- either the direct or iterative solvers are recommended in
most scenarios.
\end{warningbox}

These solvers can be run in one of three approaches:

\begin{itemize}
  \item The \textbf{Full} approach constructs the global system based on all of
  the degrees of freedom contained within an element. For most of the \nekpp
  solvers, this technique is not recommended.
  \item The \textbf{StaticCond} approach applies a technique called \emph{static
    condensation} to instead construct the system using only the degrees of
  freedom on the boundaries of the elements, which reduces the system size
  considerably. This is the \textbf{default option in parallel}.
  \item \textbf{MultiLevelStaticCond} methods apply the static condensation
  technique repeatedly to further reduce the system size, which can improve
  performance by 25-30\% over the normal static condensation method. It is
  therefore the \textbf{default option in serial}. Note that whilst parallel
  execution technically works, this is under development and is likely to be
  slower than single-level static condensation: this is therefore not
  recommended.
\end{itemize}

The \texttt{GlobalSysSoln} option is formed by combining the method of solution
with the approach: for example \texttt{IterativeStaticCond} or
\texttt{PETScMultiLevelStaticCond}.

\subsubsection{Preconditioner options}

Preconditioners can be used in the iterative and PETSc solvers to reduce the
number of iterations needed to converge to the solution. There are a number of
preconditioner choices, the default being a simple Jacobi (or diagonal)
preconditioner, which is enabled by default. There are a number of choices that
can be enabled through this parameter, which are all generally discretisation
and dimension-dependent:

\begin{center}
  \begin{tabular}{lll}
    \toprule
    \textbf{Name}  & \textbf{Dimensions} & \textbf{Discretisations} \\
    \midrule
    \inltt{Null}                              & All  & All \\
    \inltt{Diagonal}                          & All  & All \\
    \inltt{FullLinearSpace}                   & 2/3D & CG  \\
    \inltt{LowEnergyBlock}                    & 3D   & CG  \\
    \inltt{Block}                             & 2/3D & All \\
    \midrule
    \inltt{FullLinearSpaceWithDiagonal}       & All  & CG  \\
    \inltt{FullLinearSpaceWithLowEnergyBlock} & 2/3D & CG  \\
    \inltt{FullLinearSpaceWithBlock}          & 2/3D & CG  \\
    \bottomrule
  \end{tabular}
\end{center}

For a detailed discussion of the mathematical formulation of these options, see
the developer guide.

\subsubsection{SuccessiveRHS options}

The \texttt{SuccessiveRHS} option can be used in the iterative solver only, to
attempt to reduce the number of iterations taken to converge to a solution. It
stores a number of previous solutions or right-hand sides, dictated by the setting of the
\texttt{SuccessiveRHS} option, to give a better initial guess for the iterative
process. This  method is better than any linear extrapolation method.

It can be activated by setting
\begin{lstlisting}[style=XMLStyle]
<GLOBALSYSSOLNINFO>
    <V VAR="u,v,w">
        <I PROPERTY="GlobalSysSoln"         VALUE="IterativeStaticCond" />
        <I PROPERTY="Preconditioner"       VALUE="LowEnergyBlock"/>
        <I PROPERTY="SuccessiveRHS"       VALUE="8" />
        <I PROPERTY="IterativeSolverTolerance"    VALUE="1e-4"/>
    </V>
    <V VAR="p">
        <I PROPERTY="GlobalSysSoln"           VALUE="IterativeStaticCond" />
        <I PROPERTY="Preconditioner"         VALUE="LowEnergyBlock"/>
        <I PROPERTY="SuccessiveRHS"         VALUE="8" />
        <I PROPERTY="IterativeSolverTolerance"    VALUE="1e-4"/>
    </V>
</GLOBALSYSSOLNINFO>
\end{lstlisting}
or
\begin{lstlisting}[style=XMLStyle]
<PARAMETERS>
    <P> SuccessiveRHS = 8 </P>
</PARAMETERS>
\end{lstlisting}
The typical value of \texttt{SuccessiveRHS} is $\le$10.

The linear problem to be solved is
\begin{equation}\label{eq:linearEquationNew}
A x = b,
\end{equation}
here $x$ and $b$ are both column vectors.
There are a sequence of  already solved linear problems
\begin{equation}\label{eq:linearEquationOld}
A x_n = b_n, n = 1, 2, ..., J.
\end{equation}
Assume $x_n$ are all linearly independent.
In the successive right-hand method (see \cite{FISCHER1998193}), the best approximation to $x$ is
\begin{equation}\label{eq:approximate}
\tilde{x}=\sum_{n=1}^J\alpha_n x_n
\end{equation}
which is found by minimizing the object function
\begin{equation}\label{eq:objectFunction1}
Q_1 = \left(A(\tilde{x}-x)\right)^TA(\tilde{x}-x),
\end{equation}
or
\begin{equation}\label{eq:objectFunction2}
Q_2 = (\tilde{x}-x)^TA(\tilde{x}-x).
\end{equation}

If $Q_1$ is used, the projection bases are $e_m = b_m, m = 1, 2, ..., J$. Using
\begin{equation}\label{eq:Atildex}
\left(A(\tilde{x}-x) \right)^T= \sum_{m=1}^J\alpha_me_m^T - b^T,
\end{equation}
there is
\begin{equation}\label{eq:objectFunctionQ1_v2}
Q_1= \sum_{m=1}^J\sum_{n=1}^J\alpha_m\alpha_n e_m^T b_n - 2 \sum_{m=1}^J \alpha_m e_m^T b + b^T b.
\end{equation}
To minimize $Q_1$, there should be $\partial Q_1/\partial \alpha_m =0, m=1, 2, ..., J$.
The corresponding linear problem is
\begin{equation}\label{eq:objectFunctionQ1_solution}
M (\alpha_1, \alpha_2, ..., \alpha_J)^T =   (e_1^Tb, e_2^Tb, ..., e_J^Tb)^T,
\end{equation}
with symmetric positive definite coefficient matrix $M_{mn}=e_m^T b_n$. In this case, both the solutions $x_m$ and the right-hand sides $b_m$ need to be stored.

If $Q_2$ is used, $A$ should be a symmetric positive definite matrix, as those encountered in the Poisson equation and the Helmholtz equation. Here, the projection bases are $\hat{e}_m = x_m, m = 1, 2, ..., J$. Using
\begin{equation}\label{eq:tildex}
(\tilde{x}-x) ^T= \sum_{m=1}^J\alpha_n\hat{e}_m^T - x^T,
\end{equation}
there is
\begin{equation}\label{eq:objectFunctionQ2_v2}
Q_2= \sum_{m=1}^J\sum_{n=1}^J\alpha_m\alpha_n \hat{e}_m^T b_n - 2 \sum_{m=1}^J \alpha_m \hat{e}_m^T b + x^T b.
\end{equation}
To minimize $Q_2$, there should be $\partial Q_2/\partial \alpha_m =0, m=1, 2, ..., J$.
The corresponding linear problem is
\begin{equation}\label{eq:objectFunctionQ2_solution}
M (\alpha_1, \alpha_2, ..., \alpha_J)^T =   (\hat{e}_1^Tb, \hat{e}_2^Tb, ..., \hat{e}_J^Tb)^T,
\end{equation}
with symmetric positive definite coefficient matrix $M_{mn}=\hat{e}_m^T b_n$. In this case, only the solutions $x_m$ need to be stored.

The formulations of $Q_1$ version and $Q_2$ version are the same, except the difference of projection bases. By default, $Q_2$ is used as the object function. If you want to use $Q_1$ instead, you can assign a negative value to \texttt{SuccessiveRHS}:
\begin{lstlisting}[style=XMLStyle]
<I PROPERTY="SuccessiveRHS"         VALUE="-8" />
\end{lstlisting}
or
\begin{lstlisting}[style=XMLStyle]
<P> SuccessiveRHS = -8 </P>
\end{lstlisting}

In the original paper of Fischer (1998) \cite{FISCHER1998193}, the Gram-Schmidt orthogonal process is applied to the projection bases, this method is very stable and avoids the calculation of $M^{-1}$. However, when the memory space is full, this approach makes it hard to decide which basis should be overwritten. In our implementation, we just store the normalized right-hand sides or old solutions, i.e. $e_m^T b_m=1$ or $\hat{e}_m^T b_m=1$, and overwrite the oldest ones. The linear problem (\ref{eq:objectFunctionQ1_solution}) or (\ref{eq:objectFunctionQ2_solution}) is solved using a direct method.

To make sure $M$ is positive definite, when a new basis $e_{J+1}$ arrives, we test the following condition to decide whether or not to accept it,
\begin{equation}\label{eq:thredhold}
r = (-y^T\tilde{M} ^{-1}, 1) \begin{pmatrix}
\tilde{M}     & y\\
y^T  & 1
\end{pmatrix}
\begin{pmatrix}
-\tilde{M} ^{-1}y\\
1
\end{pmatrix} =1 - y^T\tilde{M}^{-1}y \ge \varepsilon > 0.
\end{equation}
Assuming $e_J$ is the oldest basis, there are $\tilde{M}_{mn} = M_{mn}, y_m = e_{m}^Tb_{J+1}, (m, n=1, 2, ..., J-1)$.

\subsubsection{PETSc options and configuration}
\label{sec:petsc}

The PETSc solvers, although currently experimental, are operational both in
serial and parallel. PETSc gives access to a wide range of alternative solver
options such as GMRES, as well as any packages that PETSc can link against, such
as the direct multi-frontal solver MUMPS.

Configuration of PETSc options using its command-line interface dictates what
matrix storage, solver type and preconditioner should be used. This should be
specified in a \texttt{.petscrc} file inside your working directory, as command
line options are not currently passed through to PETSc to avoid conflict with
\nekpp options. As an example, to select a GMRES solver using an algebraic
multigrid preconditioner, and view the residual convergence, one can use the
configuration:

\begin{lstlisting}[style=BashInputStyle]
-ksp_monitor
-ksp_view
-ksp_type gmres
-pc_type gamg
\end{lstlisting}

Or to use MUMPS, one could use the options:

\begin{lstlisting}[style=BashInputStyle]
-ksp_type preonly
-pc_type lu
-pc_factor_mat_solver_package mumps
-mat_mumps_icntl_7 2
\end{lstlisting}

A final choice that can be specified is whether to use a \emph{shell}
approach. By default, \nekpp will construct a PETSc sparse matrix (or whatever
matrix is specified on the command line). This may, however, prove suboptimal
for higher order discretisations. In this case, you may choose to use the \nekpp
matrix-vector operators, which by default use an assembly approach that can
prove faster, by setting the \texttt{PETScMatMult} \texttt{SOLVERINFO} option to
\texttt{Shell}:

\begin{lstlisting}[style=XMLStyle]
<I PROPERTY="PETScMatMult" VALUE="Shell" />
\end{lstlisting}

The downside to this approach is that you are now constrained to using one of
the \nekpp preconditioners. However, this does give access to a wider range of
Krylov methods than are available inside \nekpp for more advanced users.

\subsection{Boundary Regions and Conditions}

Boundary conditions are defined by two XML elements. The first defines the
boundary regions in the domain in terms of composite entities from the
\inltt{GEOMETRY} section of the file. Each boundary region has a unique ID and
are defined as, 
\begin{lstlisting}[style=XMLStyle]
<BOUNDARYREGIONS>
    <B ID=[id]> [composite-list] </B>
    ...
</BOUNDARYREGIONS>
\end{lstlisting}
For example,
\begin{lstlisting}[style=XMLStyle]
<BOUNDARYREGIONS>
  <B ID="0"> C[2] </B>
  <B ID="1"> C[3] </B>
</BOUNDARYREGIONS>
\end{lstlisting}

The second XML element defines, for each variable, the condition to impose on
each boundary region, and has the form,
\begin{lstlisting}[style=XMLStyle]
<BOUNDARYCONDITIONS>
    <REGION REF="[regionID]">
      <[type1] VAR="[variable1]" VALUE="[expression1]" />
      ...
      <[typeN] VAR="[variableN]" VALUE="[expressionN]" />
    </REGION>
    ...
</BOUNDARYCONDITIONS>
\end{lstlisting}
There should be precisely one \inltt{REGION} entry for each \inltt{B} entry
defined in the \inltt{BOUNDARYREGION} section above. For example, to impose a
Dirichlet condition on both variables for a domain with a single region, 
\begin{lstlisting}[style=XMLStyle] 
<BOUNDARYCONDITIONS>
  <REGION REF="0">
    <D VAR="u" VALUE="sin(PI*x)*cos(PI*y)" /> 
    <D VAR="v" VALUE="sin(PI*x)*cos(PI*y)" />
  </REGION>
</BOUNDARYCONDITIONS>
\end{lstlisting}
Boundary condition specifications may refer to any parameters defined in the
session file. The \inltt{REF} attribute corresponds to a defined boundary
region. The tag used for each variable specifies the type of boundary condition
to enforce.

\subsubsection{Dirichlet (essential) condition}
Dirichlet conditions are specified with the \inltt{D} tag.

\begin{tabular}{llll}
Projection & Homogeneous support & Time-dependent support & Dimensions \\
\toprule
CG & Yes & Yes & 1D, 2D and 3D \\
DG & Yes & Yes & 1D, 2D and 3D \\
HDG& Yes & Yes & 1D, 2D and 3D
\end{tabular}

Example:
\begin{lstlisting}[style=XMLStyle]
<!-- homogeneous condition -->
<D VAR="u" VALUE="0" />
<!-- inhomogeneous condition -->
<D VAR="u" VALUE="x^2+y^2+z^2" />
<!-- time-dependent condition -->
<D VAR="u" USERDEFINEDTYPE="TimeDependent" VALUE="x+t" />
\end{lstlisting}

\subsubsection{Neumann (natural) condition}
Neumann conditions are specified with the \inltt{N} tag.

\begin{tabular}{llll}
Projection & Homogeneous support & Time-dependent support & Dimensions \\
\toprule
CG & Yes & Yes & 1D, 2D and 3D \\
DG & No  & No  & 1D, 2D and 3D \\
HDG & ? & ? & ?
\end{tabular}

Example:
\begin{lstlisting}[style=XMLStyle]
<!-- homogeneous condition -->
<N VAR="u" VALUE="0" />
<!-- inhomogeneous condition -->
<N VAR="u" VALUE="x^2+y^2+z^2" />
<!-- time-dependent condition -->
<N VAR="u" USERDEFINEDTYPE="TimeDependent" VALUE="x+t" />
<!-- high-order pressure boundary condition (for IncNavierStokesSolver) -->
<N VAR="u" USERDEFINEDTYPE="H" VALUE="0" />
\end{lstlisting}

\subsubsection{Periodic condition}
Periodic conditions are specified with the \inltt{P} tag.

\begin{tabular}{lll}
Projection & Homogeneous support & Dimensions \\
\toprule
CG  & Yes & 1D, 2D and 3D \\
DG  & No  & 2D and 3D
\end{tabular}

Example:
\begin{lstlisting}[style=XMLStyle]
<BOUNDARYREGIONS>
  <B ID="0"> C[1] </B>
  <B ID="1"> C[2] </B>
</BOUNDARYREGIONS>

<BOUNDARYCONDITIONS>
  <REGION REF="0">
    <P VAR="u" VALUE="[1]" />
  </REGION>
  <REGION REF="1">
    <P VAR="u" VALUE="[0]" />
  </REGION>
</BOUNDARYCONDITIONS>
\end{lstlisting}

Periodic boundary conditions are specified in a significantly different form to
other conditions. The \inltt{VALUE} property is used to specify which
\inltt{BOUNDARYREGION} is periodic with the current region in square brackets.

Caveats:
\begin{itemize}
\item A periodic condition must be set for '''both''' boundary regions; simply
 specifying a condition for region 0 or 1 in the above example is not enough.
\item The order of the elements inside the composites defining periodic
boundaries is important. For example, if `C[0]` above is defined as edge IDs 
`{0,5,4,3}` and `C[1]` as `{7,12,2,1}` then edge 0 is periodic with edge 7, 5 
with 12, and so on.
\item For the above reason, the composites must also therefore be of the same
size.
\item In three dimensions, care must be taken to correctly align triangular
faces which are intended to be periodic. The top (degenerate) vertex should be 
aligned so that, if the faces were connected, it would lie at the same point on 
both triangles.

\item It is possible specify periodic boundaries that are related by a
  rotation about a cartesian axis. In three-dimensions it is necessary
  to specify the rotational arguments to allow the orientation of each
  periodic face to be determined. This is not required in
  two-dimensions. An example of how two periodic boundaries are
  related by a rotation about the x-axis of $PI/6$ is shown below. The
  last number specifies an optional tolerance to which the rotation is
  considered as equivalent (default value is $1e-8$).
  
  \begin{lstlisting}[style=XMLStyle]
<BOUNDARYREGIONS>
  <B ID="0"> C[1] </B>
  <B ID="1"> C[2] </B>
</BOUNDARYREGIONS>

<BOUNDARYCONDITIONS>
  <REGION REF="0">
    <P VAR="u" USERDEFINEDTYPE="Rotated:x:PI/6:1e-6"  VALUE="[1]" />
  </REGION>
  <REGION REF="1">
    <P VAR="u" USERDEFINEDTYPE="Rotated:x:-PI/6:1e-6"  VALUE="[0]" />
  </REGION>
</BOUNDARYCONDITIONS>
\end{lstlisting}


\end{itemize}

\subsubsection{Time-dependent boundary conditions}
Time-dependent boundary conditions may be specified through setting the
\inltt{USERDEFINEDTYPE} attribute and using the parameter \inltt{t} where the
current time is required. For example,
\begin{lstlisting}[style=XMLStyle]
<D VAR="u" USERDEFINEDTYPE="TimeDependent" VALUE="sin(PI*(x-t))" />
\end{lstlisting}

\subsubsection{Boundary conditions from file}
Boundary conditions can also be loaded from file. The following example is from
the Incompressible Navier-Stokes solver,
\begin{lstlisting}[style=XMLStyle]
<REGION REF="1">
  <D VAR="u" FILE="Test_ChanFlow2D_bcsfromfiles_u_1.bc" />
  <D VAR="v" VALUE="0" />
  <N VAR="p" USERDEFINEDTYPE="H" VALUE="0" />
</REGION>
\end{lstlisting}

Boundary conditions can also be loaded simultaneously from a file and from an 
expression (currently only implemented in 3D).
For example, in the scenario where a spatial boundary 
condition is read from a file, but needs to be modulated by a time-dependent 
expression:
\begin{lstlisting}[style=XMLStyle]
<REGION REF="1">
  <D VAR="u" USERDEFINEDTYPE="TimeDependent" VALUE="sin(PI*(x-t))"
             FILE="bcsfromfiles_u_1.bc" />
</REGION>
\end{lstlisting}

In the case where both \inltt{VALUE} and \inltt{FILE} are specified, the values
are multiplied together to give the final value for the boundary condition. 

\subsection{Functions}

Finally, multi-variable functions such as initial conditions and analytic
solutions may be specified for use in, or comparison with, simulations. These
may be specified using expressions (\inltt{<E>}) or imported from a file
(\inltt{<F>}) using the Nektar++ FLD file format

\begin{lstlisting}[style=XMLStyle]
<FUNCTION NAME="ExactSolution">
  <E VAR="u" VALUE="sin(PI*x-advx*t))*cos(PI*(y-advy*t))" />
</FUNCTION>
<FUNCTION NAME="InitialConditions">
  <F VAR="u" FILE="session.rst" />
</FUNCTION>
\end{lstlisting}

A restart file is a solution file (in other words an .fld renamed as .rst) where
the field data is specified. The expansion order used to generate the .rst file
must be the same as that for the simulation.
.pts files contain scattered point data which needs to be interpolated to the field.
For further information on the file format and the different interpolation schemes, see
section~\ref{s:utilities:fieldconvert:sub:interppointdatatofld}.
All filenames must be specified relative to the location of the .xml file.

With the additional argument \inltt{TIMEDEPENDENT="1"}, different files can be
loaded for each timestep. The filenames are defined using
\href{http://www.boost.org/doc/libs/1_56_0/libs/format/doc/format.html#syntax}{boost::format syntax}
where the step time is used as variable. For example, the function
\inltt{Baseflow} would load the files \inltt{U0V0\_1.00000000E-05.fld},
\inltt{U0V0\_2.00000000E-05.fld} and so on.

\begin{lstlisting}[style=XMLStyle]
<FUNCTION NAME="Baseflow">
       <F VAR="U0,V0" TIMEDEPENDENT="1" FILE="U0V0_%14.8E.fld"/>
</FUNCTION>
\end{lstlisting}

For .pts files, the time consuming computation of interpolation weights is only
performed for the first timestep. The weights are stored and reused in all subsequent steps, 
which is why all consecutive .pts files must use the same ordering, number and location of
data points.

Other examples of this input feature can be the insertion of a forcing term,

\begin{lstlisting}[style=XMLStyle]
<FUNCTION NAME="BodyForce">
  <E VAR="u" VALUE="0" />
  <E VAR="v" VALUE="0" />
</FUNCTION>
<FUNCTION NAME="Forcing">
  <E VAR="u" VALUE="-(Lambda + 2*PI*PI)*sin(PI*x)*sin(PI*y)" />
</FUNCTION>
\end{lstlisting}

or of a linear advection term

\begin{lstlisting}[style=XMLStyle]
<FUNCTION NAME="AdvectionVelocity">
  <E VAR="Vx" VALUE="1.0" />
  <E VAR="Vy" VALUE="1.0" />
  <E VAR="Vz" VALUE="1.0" />
</FUNCTION>
\end{lstlisting}

\subsubsection{Remapping variable names}

Note that it is sometimes the case that the variables being used in the solver
do not match those saved in the FLD file. For example, if one runs a
three-dimensional incompressible Navier-Stokes simulation, this produces an FLD
file with the variables \inltt{u}, \inltt{v}, \inltt{w} and \inltt{p}. If we
wanted to use this velocity field as input for an advection velocity, the
advection-diffusion-reaction solver expects the variables \inltt{Vx}, \inltt{Vy}
and \inltt{Vz}. We can manually specify this mapping by adding a colon to the
filename, indicating the variable names in the target file that align with the
desired function variable names. This gives a definition such as:

\begin{lstlisting}[style=XMLStyle]
<FUNCTION NAME="AdvectionVelocity">
  <F VAR="Vx,Vy,Vz" FILE="file.fld:u,v,w" />
</FUNCTION>
\end{lstlisting}

There are some caveats with this syntax:

\begin{itemize}
  \item The same number of fields must be defined for both the \inltt{VAR}
  attribute and in the comma-separated list after the colon. For example, the
  following is not valid:
  \begin{lstlisting}[style=XMLStyle,gobble=4]
    <FUNCTION NAME="AdvectionVelocity">
      <F VAR="Vx,Vy,Vz" FILE="file.fld:u" />
    </FUNCTION>\end{lstlisting}
  \item This syntax is not valid with the wildcard operator \inltt{*}, so one
  cannot write for example:
  \begin{lstlisting}[style=XMLStyle,gobble=4]
    <FUNCTION NAME="AdvectionVelocity">
      <F VAR="*" FILE="file.fld:u,v,w" />
    </FUNCTION>
  \end{lstlisting}
\end{itemize}

\subsubsection{Time-dependent file-based functions}

With the additional argument \inltt{TIMEDEPENDENT="1"}, different files can be
loaded for each timestep. The filenames are defined using
\href{http://www.boost.org/doc/libs/1_56_0/libs/format/doc/format.html#syntax}{boost::format
  syntax} where the step time is used as variable. For example, the function
\inltt{Baseflow} would load the files \inltt{U0V0\_1.00000000E-05.fld},
\inltt{U0V0\_2.00000000E-05.fld} and so on.

\begin{lstlisting}[style=XMLStyle]
<FUNCTION NAME="Baseflow">
  <F VAR="U0,V0" TIMEDEPENDENT="1" FILE="U0V0_%14.8R.fld" />
</FUNCTION>
\end{lstlisting}

Section~\ref{sec:xml:expressions} provides the list of acceptable
mathematical functions and other related technical details.

\subsection{Quasi-3D approach}

To generate a Quasi-3D appraoch with Nektar++ we only need to create a 2D or a
1D mesh, as reported above, and then specify the parameters to extend the
problem to a 3D case. For a 2D spectral/hp element problem, we have a 2D mesh
and along with the parameters we need to define the problem (i.e. equation type,
boundary conditions, etc.). The only thing we need to do, to extend it to a
Quasi-3D approach, is to specify some additional parameters which characterise
the harmonic expansion in the third direction. First we need to specify in the
solver information section that that the problem will be extended to have one
homogeneouns dimension; here an example

\begin{lstlisting}[style=XMLStyle]
<SOLVERINFO>
  ...
  <I PROPERTY="HOMOGENEOUS"           VALUE="1D"                       />
</SOLVERINFO>
\end{lstlisting}

then we need to specify the parameters which define the 1D harmonic expanson
along the z-axis, namely the homogeneous length (LZ) and the number of modes in
the homogeneous direction (HomModesZ). \inltt{HomModesZ} corresponds also to the number
of quadrature points in the homogenous direction, hence on the number of 2D
planes discretized with a spectral/hp element method.

\begin{lstlisting}[style=XMLStyle]
<PARAMETERS>
  ...
  <P> HomModesZ     = 4       </P>
  <P> LZ            = 1.0     </P>
</PARAMETERS>
\end{lstlisting}

In case we want to create a Quasi-3D approach starting from a 1D spectral/hp
element mesh, the procedure is the same, but we need to specify the parameters
for two harmonic directions (in Y and Z direction). For Example,

\begin{lstlisting}[style=XMLStyle]
<SOLVERINFO>
  ...
  <I PROPERTY="HOMOGENEOUS"           VALUE="2D"                         />
</SOLVERINFO>
<PARAMETERS>
  ...
  <P> HomModesY     = 10    </P>
  <P> LY            = 6.5   </P>
  <P> HomModesZ     = 6     </P>
  <P> LZ            = 2.0   </P>
</PARAMETERS>
\end{lstlisting}

By default the operations associated with the harmonic expansions are performed
with the Matrix-Vector-Multiplication (MVM) defined inside the code. The Fast
Fourier Transform (FFT) can be used to speed up the operations (if the FFTW
library has been compiled in ThirdParty, see the compilation instructions). To
use the FFT routines we need just to insert a flag in the solver information as
below:

\begin{lstlisting}[style=XMLStyle]
<SOLVERINFO>
  ...
  <I PROPERTY="HOMOGENEOUS"           VALUE="2D"                         />
  <I PROPERTY="USEFFT"                VALUE="FFTW"                       />
</SOLVERINFO>
\end{lstlisting}

The number of homogeneous modes has to be even. The Quasi-3D approach can be
created starting from a 2D mesh and adding one homogenous expansion or starting
form a 1D mesh and adding two homogeneous expansions. Not other options
available. In case of a 1D homogeneous extension, the homogeneous direction will
be the z-axis. In case of a 2D homogeneous extension, the homogeneous directions
will be the y-axis and the z-axis.

%%% Local Variables:
%%% mode: latex
%%% TeX-master: "../user-guide"
%%% End:


\section{Filters}

Filters are a method for calculating a variety of useful quantities from the
field variables as the solution evolves in time, such as time-averaged fields
and extracting the field variables at certain points inside the domain. Each
filter is defined in a \inltt{FILTER} tag inside a \inltt{FILTERS} block which
lies in the main \inltt{NEKTAR} tag. In this section we give an overview of the
modules currently available and how to set up these filters in the session file.

Here is an example \inltt{FILTER}:

\begin{lstlisting}[style=XMLStyle,gobble=2]
  <FILTER TYPE="FilterName">
      <PARAM NAME="Param1"> Value1 </PARAM>
      <PARAM NAME="Param2"> Value2 </PARAM>
  </FILTER>
\end{lstlisting}

A filter has a name -- in this case, \inltt{FilterName} -- together with
parameters which are set to user-defined values. Each filter expects different
parameter inputs, although where functionality is similar, the same parameter
names are shared between filter types for consistency. Numerical filter
parameters may be expressions and so may include session parameters defined in
the \inltt{PARAMETERS} section.

In the following we document the filters implemented. Note that some filters are
solver-specific and will therefore only work for a given subset of the available
solvers.

\subsection{FieldConvert checkpoints}

\begin{notebox}
  This filter is still at an experimental stage. Not all modules and options
  from FieldConvert are supported.
\end{notebox}

This filter applies a sequence of FieldConvert modules to the solution, 
writing an output file. An output is produced at the end of the simulation into
\inltt{session\_fc.fld}, or alternatively every $M$ timesteps as defined by the
user, into a sequence of files \inltt{session\_*\_fc.fld}, where \inltt{*} is
replaced by a counter.

The following parameters are supported:

\begin{center}
  \begin{tabularx}{0.99\textwidth}{lllX}
    \toprule
    \textbf{Option name} & \textbf{Required} & \textbf{Default} & 
    \textbf{Description} \\
    \midrule
    \inltt{OutputFile}      & \xmark   & \texttt{session.fld} &
    Output filename. If no extension is provided, it is assumed as .fld\\
    \inltt{OutputFrequency} & \xmark   & \texttt{NumSteps} &
    Number of timesteps after which output is written, $M$.\\
    \inltt{Modules} & \xmark   &  &
    FieldConvert modules to run, separated by a white space.\\
    \bottomrule
  \end{tabularx}
\end{center}

As an example, consider:

\begin{lstlisting}[style=XMLStyle,gobble=2]
  <FILTER TYPE="FieldConvert">
      <PARAM NAME="OutputFile">MyFile.vtu</PARAM>
      <PARAM NAME="OutputFrequency">100</PARAM>
      <PARAM NAME="Modules"> vorticity isocontour:fieldid=0:fieldvalue=0.1 </PARAM>
  </FILTER>
\end{lstlisting}

This will create a sequence of files named \inltt{MyFile\_*\_fc.vtu} containing isocontours. 
The result will be output every 100 time steps.

\subsection{Time-averaged fields}

This filter computes time-averaged fields for each variable defined in the
session file. Time averages are computed by either taking a snapshot of the
field every timestep, or alternatively at a user-defined number of timesteps
$N$. An output is produced at the end of the simulation into
\inltt{session\_avg.fld}, or alternatively every $M$ timesteps as defined by the
user, into a sequence of files \inltt{session\_*\_avg.fld}, where \inltt{*} is
replaced by a counter. This latter option can be useful to observe statistical
convergence rates of the averaged variables.

This filter is derived from FieldConvert filter, and therefore support all parameters
available in that case. The following additional parameter is supported:

\begin{center}
  \begin{tabularx}{0.99\textwidth}{lllX}
    \toprule
    \textbf{Option name} & \textbf{Required} & \textbf{Default} & 
    \textbf{Description} \\
    \midrule
    \inltt{SampleFrequency} & \xmark   & 1 &
    Number of timesteps at which the average is calculated, $N$.\\
    \inltt{RestartFile} & \xmark   &   &
    Restart file used as initial average.
    If no extension is provided, it is assumed as .fld\\
  \end{tabularx}
\end{center}

As an example, consider:

\begin{lstlisting}[style=XMLStyle,gobble=2]
  <FILTER TYPE="AverageFields">
      <PARAM NAME="OutputFile">MyAverageField</PARAM>
      <PARAM NAME="RestartFile">MyRestartAvg.fld</PARAM>
      <PARAM NAME="OutputFrequency">100</PARAM>
      <PARAM NAME="SampleFrequency"> 10 </PARAM>		
  </FILTER>
\end{lstlisting}

This will create a file named \inltt{MyAverageField.fld} averaging the
instantaneous fields every 10 time steps. The averaged field is however only
output every 100 time steps.

\subsection{Moving average of fields}

This filter computes the exponential moving average (in time) of
fields for each variable defined in the session file. The moving average 
is defined as:
\[
\bar{u}_n = \alpha u_n + (1 - \alpha)\bar{u}_{n-1}
\]
with $0 < \alpha < 1$ and $\bar{u}_1 = u_1$.

The same parameters of the time-average filter are supported, with the output file
in the form \inltt{session\_*\_movAvg.fld}. In addition,
either $\alpha$ or the time-constant $\tau$ must be defined. They are related by:
\[
\alpha = \frac{t_s}{\tau + t_s}
\]
where $t_s$ is the time interval between consecutive samples.

As an example, consider:

\begin{lstlisting}[style=XMLStyle,gobble=2]
  <FILTER TYPE="MovingAverage">
      <PARAM NAME="OutputFile">MyMovingAverage</PARAM>
      <PARAM NAME="OutputFrequency">100</PARAM>
      <PARAM NAME="SampleFrequency"> 10 </PARAM>
      <PARAM NAME="tau"> 0.1 </PARAM>
  </FILTER>
\end{lstlisting}

This will create a file named \inltt{MyMovingAverage\_movAvg.fld} with a moving average
sampled every 10 time steps. The averaged field is however only
output every 100 time steps.

\subsection{Reynolds stresses}

\begin{notebox}
  This filter is only supported for the incompressible Navier-Stokes solver.
\end{notebox}

This filter is an extended version of the time-average filter. It outputs
not only the time-average of the fields, but also the Reynolds stresses.
The same parameters supported in the time-average case can be used,
for example:

\begin{lstlisting}[style=XMLStyle,gobble=2]
  <FILTER TYPE="ReynoldsStresses">
      <PARAM NAME="OutputFile">MyAverageField</PARAM>
      <PARAM NAME="RestartFile">MyAverageRst.fld</PARAM>
      <PARAM NAME="OutputFrequency">100</PARAM>
      <PARAM NAME="SampleFrequency"> 10 </PARAM>
  </FILTER>
\end{lstlisting}

By default, this filter uses a simple average. Optionally, an exponential
moving average can be used, in which case the output contains the moving
averages and the Reynolds stresses calculated based on them. For example:

\begin{lstlisting}[style=XMLStyle,gobble=2]
  <FILTER TYPE="ReynoldsStresses">
      <PARAM NAME="OutputFile">MyAverageField</PARAM>
      <PARAM NAME="MovingAverage">true</PARAM>
      <PARAM NAME="OutputFrequency">100</PARAM>
      <PARAM NAME="SampleFrequency"> 10 </PARAM>
      <PARAM NAME="alpha"> 0.01 </PARAM>
  </FILTER>
\end{lstlisting}

\subsection{Checkpoint fields}
 
The checkpoint filter writes a checkpoint file, containing the instantaneous
state of the solution fields at at given timestep. This can subsequently be used
for restarting the simulation or examining time-dependent behaviour. This
produces a sequence of files, by default named \inltt{session\_*.chk}, where
\inltt{*} is replaced by a counter. The initial condition is written to
\inltt{session\_0.chk}. Existing files are not overwritten, but renamed to e.g.
\inltt{session\_0.bak0.chk}. In case this file already exists, too, the \inltt{chk}-file
is renamed to \inltt{session\_0.bak*.chk} and so on.


\begin{notebox}
  This functionality is equivalent to setting the \inltt{IO\_CheckSteps}
  parameter in the session file.
\end{notebox}

The following parameters are supported:

\begin{center}
  \begin{tabularx}{0.99\textwidth}{lllX}
    \toprule
    \textbf{Option name} & \textbf{Required} & \textbf{Default} & 
    \textbf{Description} \\
    \midrule
    \inltt{OutputFile}      & \xmark   & \texttt{session} &
    Prefix of the output filename to which the checkpoints are written.\\
    \inltt{OutputFrequency} & \cmark   & - &
    Number of timesteps after which output is written.\\
    \bottomrule
  \end{tabularx}
\end{center}

For example, to output the fields every 100 timesteps we can specify:

\begin{lstlisting}[style=XMLStyle,gobble=2]
  <FILTER TYPE="Checkpoint">
      <PARAM NAME="OutputFile">IntermediateFields</PARAM>
      <PARAM NAME="OutputFrequency">100</PARAM>
  </FILTER>
\end{lstlisting}
 
\subsection{History points}

The history points filter can be used to evaluate the value of the fields in
specific points of the domain as the solution evolves in time. By default this 
produces a file called \inltt{session.his}. For each timestep, and then each 
history point, a line is output containing the current solution time, followed 
by the value of each of the field variables. Commented lines are created at the
top of the file containing the location of the history points and the order of 
the variables.

The following parameters are supported:

\begin{center}
  \begin{tabularx}{0.99\textwidth}{lllX}
    \toprule
    \textbf{Option name} & \textbf{Required} & \textbf{Default} & 
    \textbf{Description} \\
    \midrule
    \inltt{OutputFile}      & \xmark   & \texttt{session} &
    Prefix of the output filename to which the checkpoints are written.\\
    \inltt{OutputFrequency} & \xmark   & 1 &
    Number of timesteps after which output is written.\\
    \inltt{OutputPlane}     & \xmark   & 0 &
    If the simulation is homogeneous, the plane on which to evaluate the 
    history point. (No Fourier interpolation is currently implemented.)\\
    \inltt{Points      }    & \cmark   & - &
    A list of the history points. These should always be given in three
    dimensions. \\
    \bottomrule
  \end{tabularx}
\end{center}

For example, to output the value of the solution fields at three points
$(1,0.5,0)$, $(2,0.5,0)$ and $(3,0.5,0)$ into a file \inltt{TimeValues.his}
every 10 timesteps, we use the syntax:

\begin{lstlisting}[style=XMLStyle,gobble=2]
  <FILTER TYPE="HistoryPoints">
      <PARAM NAME="OutputFile">TimeValues</PARAM>
      <PARAM NAME="OutputFrequency">10</PARAM>
      <PARAM NAME="Points">
          1 0.5 0
          2 0.5 0
          3 0.5 0
      </PARAM>
  </FILTER>
\end{lstlisting}

\subsection {ThresholdMax}

The threshold value filter writes a field output containing a variable $m$,
defined by the time at which the selected variable first exceeds a specified
threshold value. The default name of the output file is the name of the session
with the suffix \inlsh{\_max.fld}. Thresholding is applied based on the first
variable listed in the session by default.

The following parameters are supported:

\begin{center}
  \begin{tabularx}{0.99\textwidth}{lllX}
    \toprule
    \textbf{Option name} & \textbf{Required} & \textbf{Default} & 
    \textbf{Description} \\
    \midrule
    \inltt{OutputFile}      & \xmark   & \emph{session}\_max.fld &
    Output filename to which the threshold times are written.\\
    \inltt{ThresholdVar}    & \xmark   & \emph{first variable name} &
    Specifies the variable on which the threshold will be applied.\\
    \inltt{ThresholdValue}  & \cmark   & - &
    Specifies the threshold value.\\
    \inltt{InitialValue}    & \cmark   & - &
    Specifies the initial time.\\
    \inltt{StartTime}       & \xmark   & 0.0 &
    Specifies the time at which to start recording.\\
    \bottomrule
  \end{tabularx}
\end{center}
 
An example is given below:
 
\begin{lstlisting}[style=XMLStyle]
  <FILTER TYPE="ThresholdMax">
      <PARAM NAME="OutputFile"> threshold_max.fld </PARAM>
      <PARAM NAME="ThresholdVar"> u </PARAM>
      <PARAM NAME="ThresholdValue"> 0.1 </PARAM>
      <PARAM NAME="InitialValue">  0.4 </PARAM>
  </FILTER>
\end{lstlisting}

which produces a field file \inlsh{threshold\_max.fld}.

\subsection{ThresholdMin value}

Performs the same function as the \inltt{ThresholdMax} filter but records the
time at which the threshold variable drops below a prescribed value.

\subsection{One-dimensional energy}

This filter is designed to output the energy spectrum of one-dimensional
elements. It transforms the solution field at each timestep into a orthogonal
basis defined by the functions
\[
\psi_p(\xi) = L_p(\xi)
\]
where $L_p$ is the $p$-th Legendre polynomial. This can be used to show the
presence of, for example, oscillations in the underlying field due to numerical
instability. The resulting output is written into a file called
\inltt{session.eny} by default. The following parameters are supported:

\begin{center}
  \begin{tabularx}{0.99\textwidth}{lllX}
    \toprule
    \textbf{Option name} & \textbf{Required} & \textbf{Default} &
    \textbf{Description} \\
    \midrule
    \inltt{OutputFile}      & \xmark   & \inltt{session} &
    Prefix of the output filename to which the energy spectrum is written.\\
    \inltt{OutputFrequency} & \xmark   & 1 &
    Number of timesteps after which output is written.\\
    \bottomrule
  \end{tabularx}
\end{center}

An example syntax is given below:

\begin{lstlisting}[style=XMLStyle,gobble=2]
  <FILTER TYPE="Energy1D">
      <PARAM NAME="OutputFile">EnergyFile</PARAM>
      <PARAM NAME="OutputFrequency">10</PARAM>
  </FILTER>
\end{lstlisting}

\subsection{Modal energy}

\begin{notebox}
  This filter is only supported for the incompressible Navier-Stokes solver.
\end{notebox}

This filter calculates the time-evolution of the kinetic energy. In the case of
a two- or three-dimensional simulation this is defined as
\[
E_k(t) = \frac{1}{2} \int_{\Omega} \|\mathbf{u}\|^2\, dx
\]
However if the simulation is written as a one-dimensional homogeneous expansion
so that
\[
\mathbf{u}(\mathbf{x},t) = \sum_{k=0}^N \mathbf{\hat{u}}_k(t)e^{2\pi ik\mathbf{x}}
\]
then we instead calculate the energy spectrum
\[
E_k(t) = \frac{1}{2} \int_{\Omega} \|\mathbf{\hat{u}}_k\|^2\, dx.
\]
Note that in this case, each component of $\mathbf{\hat{u}}_k$ is a complex
number and therefore $N = \inltt{HomModesZ}/2$ lines are output for each
timestep. This is a particularly useful tool in examining turbulent and
transitional flows which use the homogeneous extension. In either case, the
resulting output is written into a file called \inltt{session.mdl} by default.

The following parameters are supported:

\begin{center}
  \begin{tabularx}{0.99\textwidth}{lllX}
    \toprule
    \textbf{Option name} & \textbf{Required} & \textbf{Default} & 
    \textbf{Description} \\
    \midrule
    \inltt{OutputFile}      & \xmark   & \inltt{session} &
    Prefix of the output filename to which the energy spectrum is written.\\
    \inltt{OutputFrequency} & \xmark   & 1 &
    Number of timesteps after which output is written.\\
    \bottomrule
  \end{tabularx}
\end{center}

An example syntax is given below:

\begin{lstlisting}[style=XMLStyle,gobble=2]
  <FILTER TYPE="ModalEnergy">
      <PARAM NAME="OutputFile">EnergyFile</PARAM>
      <PARAM NAME="OutputFrequency">10</PARAM>
  </FILTER>
\end{lstlisting}

\subsection{Aerodynamic forces}

\begin{notebox}
  This filter is only supported for the incompressible Navier-Stokes solver.
\end{notebox}

This filter evaluates the aerodynamic forces along a specific surface. The
forces are projected along the Cartesian axes and the pressure and viscous
contributions are computed in each direction.

The following parameters are supported:

\begin{center}
  \begin{tabularx}{0.99\textwidth}{lllX}
    \toprule
    \textbf{Option name} & \textbf{Required} & \textbf{Default} & 
    \textbf{Description} \\
    \midrule
    \inltt{OutputFile}      & \xmark   & \inltt{session} &
    Prefix of the output filename to which the forces are written.\\
    \inltt{Frequency}       & \xmark   & 1 &
    Number of timesteps after which output is written.\\
    \inltt{Boundary}        & \cmark   & - &
    Boundary surfaces on which the forces are to be evaluated.\\
    \bottomrule
  \end{tabularx}
\end{center}

An example is given below:

\begin{lstlisting}[style=XMLStyle]
  <FILTER TYPE="AeroForces">
      <PARAM NAME="OutputFile">DragLift</PARAM>
      <PARAM NAME="OutputFrequency">10</PARAM>
      <PARAM NAME="Boundary"> B[1,2] </PARAM>		
  </FILTER>
\end{lstlisting}

During the execution a file named \inltt{DragLift.fce} will be created and the
value of the aerodynamic forces on boundaries 1 and 2, defined in the
\inltt{GEOMETRY} section, will be output every 10 time steps.

\subsection{Kinetic energy and enstrophy}

\begin{notebox}
  This filter is only supported for the incompressible and compressible
  Navier-Stokes solvers \textbf{in three dimensions}.
\end{notebox}

The purpose of this filter is to calculate the kinetic energy and enstrophy
%
\[
E_k = \frac{1}{2\mu(\Omega)}\int_{\Omega} \|\mathbf{u}\|^2\, dx, \qquad
\mathcal{E} = \frac{1}{2\mu(\Omega)}\int_{\Omega} \|\mathbf{\omega}\|^2\, dx
\]
%
where $\mu(\Omega)$ is the volume of the domain $\Omega$. This produces a file
containing the time-evolution of the kinetic energy and enstrophy fields. By
default this file is called \inltt{session.eny} where \inltt{session} is the
session name.

The following parameters are supported:
%
\begin{center}
  \begin{tabularx}{0.99\textwidth}{lllX}
    \toprule
    \textbf{Option name} & \textbf{Required} & \textbf{Default} & 
    \textbf{Description} \\
    \midrule
    \inltt{OutputFile}      & \xmark   & \texttt{session.eny} &
    Output file name to which the energy and enstrophy are written.\\
    \inltt{OutputFrequency} & \cmark   & - &
    Number of timesteps at which output is written.\\
    \bottomrule
  \end{tabularx}
\end{center}
%
To enable the filter, add the following to the \inltt{FILTERS} tag:
%
\begin{lstlisting}[style=XMLStyle,gobble=2]
  <FILTER TYPE="Energy">
      <PARAM NAME="OutputFrequency"> 1 </PARAM>
  </FILTER>
\end{lstlisting}


\section{Forcing}
\label{sec:xml:forcing}
An optional section of the file allows forcing functions to be defined. These are enclosed in the
\inltt{FORCING} tag. The forcing type is enclosed within the \inltt{FORCE} tag and expressed in the file as:

\begin{lstlisting}[style=XMLStyle] 
<FORCE TYPE="[NAME]">
    ...
</FORCE>
\end{lstlisting}

The force type can be any one of the following.

\subsection{Absorption}
This force type allows the user to apply an absorption layer (essentially a porous region) anywhere in the domain. The user may also specify a velocity profile to be imposed at the start of this layer, and in the event of a time-dependent simulation, this profile can be modulated with a time-dependent function. These velocity functions and the function defining the region in which to apply the absorption layer are expressed in the \inltt{CONDITIONS} section, however the name of these functions are defined here by the \inltt{COEFF} tag for the layer, the \inltt{REFFLOW} tag for the velocity profile, and the \inltt{REFFLOWTIME} for the time-dependent function.  

\begin{lstlisting}[style=XMLStyle] 
<FORCE TYPE="Absorption">
    <COEFF> [FUNCTION NAME] <COEFF/>
    <REFFLOW> [FUNCTION NAME] <REFFLOW/>
    <REFFLOWTIME> [FUNCTION NAME] <REFFLOWTIME/>
    <BOUNDARYREGIONS> 1,4 <BOUNDARYREGIONS/>
</FORCE>
\end{lstlisting}
If a list of \inltt{BOUNDARYREGIONS} is specified, the distance to these regions is available as additional variable \inltt{r} in the definition of the \inltt{COEFF} function:
\begin{lstlisting}[style=XMLStyle]
<FUNCTION NAME="AbsorptionCoefficient">
    <E VAR="p" EVARS="r" VALUE="-5000 * exp(-0.5 * (3*r / 0.4)^2)" />
    <E VAR="u" EVARS="r" VALUE="-5000 * exp(-0.5 * (3*r / 0.4)^2)" />
    <E VAR="v" EVARS="r" VALUE="-5000 * exp(-0.5 * (3*r / 0.4)^2)" />
</FUNCTION>
\end{lstlisting}

\subsection{Body}
This force type specifies the name of a body forcing function expressed in the \inltt{CONDITIONS} section.

\begin{lstlisting}[style=XMLStyle] 
<FORCE TYPE="Body">
    <BODYFORCE> [FUNCTION NAME] <BODYFORCE/>
</FORCE>
\end{lstlisting}

\subsection{IncNSSyntheticTurbulence}
This force type allows the user to apply synthetic turbulence generation in the flow field. The Synthetic Eddy Method is implemented. The approach developed here is based on a source term formulation. This formulation allows the user to apply synthetic turbulence generation in any specific area of the domain, not only in the boundary condition as most methodologies do. So that, after defining a synthetic eddy region (box of eddies), the user can randomly release eddies inside this box which are going to be convected downstream and will produce turbulence depending on the flow conditions. Each eddy  that leaves the synthetic eddy region is reintroduced in the inlet plane of the box, so this mechanism re-energise the system, roughly speaking. 

Below it is shown how to define the Synthetic Eddy Method for a fully three-dimensional Navier-Stokes simulation. Note that this definition is under the \inltt{FORCING} tag. Firstly, in the \inltt{TYPE} entry, we define the force type as \texttt{IncNSSyntheticTurbulence}. In the \inltt{BoxOfEddies} tag, under the \inltt{FORCE} tag, the center plane of the synthetic eddy region is defined. The coordinates of its center are given by \texttt{x0, y0, z0} and lengths of its sides are \texttt{lyref}  and \texttt{lzref} in the $y$- and $z$-directions, respectively. In the \inltt{Sigma} tag, we define the standard deviation (\texttt{sigma}) of the Gaussian function with zero mean, which is used to compute the stochastic signal. After that, the bulk velocity (\texttt{Ub}) of the flow must be provided in the \inltt{BulkVelocity} tag. 

\begin{lstlisting}[style=XMLStyle] 
<FORCE TYPE="IncNSSyntheticTurbulence">
    <BoxOfEddies> x0 y0 z0 lyref lzref </BoxOfEddies>
    <Sigma> sigma </Sigma>
    <BulkVelocity> Ub </BulkVelocity>
    <ReynoldsStresses> [ReynoldsStresses FUNCTION NAME] </ReynoldsStresses>
    <CharLengthScales> [LenScales FUNCTION NAME] </CharLengthScales>
</FORCE>
\end{lstlisting}

In order to define the Reynolds stresses (\inltt{ReynoldsStresses} tag) and the characteristic length scales  (\inltt{CharLengthScales} tag) of the eddies, the name of the functions which define them must be given. These functions must be placed under the   \inltt{CONDITIONS} tag. Both functions are provided below. It is worthy mentioning that it is possible to define space-dependent functions for each Reynolds stress. In other words, the user can, for instance, provide the analytical solution of the Reynolds stresses close to the wall (boundary). This information is essential to calculate the velocity fluctuations. 

\begin{lstlisting}[style=XMLStyle] 
<FUNCTION NAME="ReynoldsStresses">
    <E VAR="r00" VALUE="1e-6" />
    <E VAR="r10" VALUE="10*y" />
    <E VAR="r20" VALUE="0.0"  />
    <E VAR="r11" VALUE="1e-6" />
    <E VAR="r21" VALUE="0.0"  />
    <E VAR="r22" VALUE="1e-6" />
 </FUNCTION>
\end{lstlisting}

Also, in the Synthetic Eddy Method implemented here, an isotropic or anisotropic turbulence can be described depending on the values provided in the characteristic length scale function.  For an isotropic turbulence, all the values must be the same.

\begin{lstlisting}[style=XMLStyle] 
<FUNCTION NAME="LenScales">
    <E VAR="l00" VALUE="1.0"   />
    <E VAR="l10" VALUE="0.085" />
    <E VAR="l20" VALUE="0.125" />
    <E VAR="l01" VALUE="0.4"   />
    <E VAR="l11" VALUE="0.085" />
    <E VAR="l21" VALUE="0.125" />
    <E VAR="l02" VALUE="0.4"   />
    <E VAR="l12" VALUE="0.170" />
    <E VAR="l22" VALUE="0.25"  />
</FUNCTION>
\end{lstlisting}

Note that the Synthetic Eddy Method is only supported for a fully three-dimensional Incompressible Navier-Stokes simulation.

\subsection{MovingReferenceFrame}
This force type allows the solution of incompressilbe Navier-Stokes in moving frame of reference. The moving frame is attached the to body and can have translational, rotational or both motions. Although the Navier-Stokes equations are solved in a moving reference frame, our formulation is based on the absolute velocity and pressure (in inertial frame). However, note that these absolute velocities and any other vector quantities are expressed using the coordinate basis of the moving frame. Further, note that if you are using the FilterAeroForces, the force vector $\left(F_x, F_y, F_z\right)$ is automatically converted and output in the inertial frame (ground reference frame).

To use this formulation the user need to specify the force type inside the \inltt{FORCING} tag as follwos:

\begin{lstlisting}[style=XMLStyle]
<FORCE TYPE="MovingReferenceFrame">
    <LinearVelocity> [LinearVelocity FUNCTION NAME] <LinearVelocity/>
    <AngularVelocity> [AngularVelocity FUNCTION NAME] <AngularVelocity/>
    <PivotPoint> x0 y0 z0  <PivotPoint/>
</FORCE>
\end{lstlisting}

Here we are required to provide at least one function for this force type which can be a function that defines the linear velocity of the reference frame or a function that defines the angular velocity of reference frame or both. In the case of rotating frame, i.e. when we are prescribing the angular velocity of reference frame, we can provide a coordinate of \inltt{PivotPoint}, around which the frame is rotating. If no pivot point provided, the orgin of coordinates in the moving reference frame will be used as the pivot point. 
Note that the frame velocities (both linear and angular velocities) must be defined in the inertial stationary frame of reference,i.e. ground reference frame (and expressed using the basis of inertial stationary frame), however, the Poivot point is in the moving reference frame.

Examples of linear and angular velocity funcitons together with their usage in the Forcing is shown below:

\begin{lstlisting}[style=XMLStyle]
<CONDITIONS>

<FUNCTION NAME="LinVel">
    <E VAR="u" VALUE="2*sin(PI*t)" />
    <E VAR="v" VALUE="0.1" />
    <E VAR="w" VALUE="0" />
</FUNCTION>

<FUNCTION NAME="AngVel">
    <E VAR="Omega_x" VALUE="0" />
    <E VAR="Omega_y" VALUE="0" />
    <E VAR="Omega_z" VALUE="0.3*cos(2*PI*t)" />
</FUNCTION>

</CONDITIONS>

<FORCING>

    <FORCE TYPE="MovingReferenceFrame">
        <LinearVelocity> LinVel <LinearVelocity/>
        <AngularVelocity> AngVel <AngularVelocity/>
        <PivotPoint> 0.2 0.0 0.0  <PivotPoint/>
    </FORCE>

</FORCING>
\end{lstlisting}


The moving frame functions defines the velocity of the body frame observed in the inertial reference frame $$\mathbf{u}_{frame} = \mathbf{u}_0 + \mathbf{\Omega}\times (\mathbf{x}-\mathbf{x}_0)$$. This means that these functions (such as the \inltt{LinVel} and \inltt{AngVel} in the above example) are defined and expressed in the stationary inertial frame (ground frame).

Here, $\mathbf{u}_0 = (\text{u, v, w})$ is the translational velocity, $\mathbf{\Omega}=(\text{Omega\_x, Omega\_y, Omega\_z})$ is the angular velocity.
$\mathbf{x}_0=(0.2, 0.0, 0.0)$ is the rotation pivot and it is fixed in the body frame.
Translational motion is allowed for all dimensions while rotational motion is currently restricted to z (omega\_z) for 2D, 3DH1D and full 3D simulaitons.

Finally, note that when using \inltt{MovingReferenceFrame} force type, for any open part of the computational domain that the user specifies the velocity, such as inlet and free stream boundary conditions, the \inltt{USERDEFINEDTYPE="MovingFrameDomainVel"} tag should be used for all of velocity components. For example if boundary \inltt{ID=2} is the inlet with \inltt{Uinfx} and \inltt{Uinfy} the values of inlet velocities defined as parameters, the boundary condition for this boundary becomes:

\begin{lstlisting}[style=XMLStyle]
<REGION REF="2"> 
   <D VAR="u" USERDEFINEDTYPE="MovingFrameDomainVel" VALUE="Uinfx" />
   <D VAR="v" USERDEFINEDTYPE="MovingFrameDomainVel" VALUE="Uinfy" />
   <N VAR="p" USERDEFINEDTYPE="H" VALUE="0" />  
</REGION>
\end{lstlisting}

for the wall boundary conditions on the surface of the body, we need to use \inltt{MovingFrameWall} tag as shown below:

\begin{lstlisting}[style=XMLStyle]
<REGION REF="0"> 
    <D VAR="u" USERDEFINEDTYPE="MovingFrameWall" VALUE="Uinfx" />
    <D VAR="v" USERDEFINEDTYPE="MovingFrameWall" VALUE="Uinfy" />
    <N VAR="p" USERDEFINEDTYPE="H" VALUE="0" />  
</REGION>
\end{lstlisting}

The outlet and pressure boundary conditions are the same as before.


\subsection{Programmatic}
This force type allows a forcing function to be applied directly within the code, thus it has no associated function. 

\begin{lstlisting}[style=XMLStyle] 
<FORCE TYPE="Programmatic">
</FORCE>
\end{lstlisting}


\subsection{Noise}
This force type allows the user to specify the magnitude of a white noise force. 
Optional arguments can also be used to define the frequency in time steps to recompute the noise (default is never)
 and the number of time steps to apply the noise (default is the entire simulation).  

\begin{lstlisting}[style=XMLStyle] 
<FORCE TYPE="Noise">
    <WHITENOISE> [VALUE] <WHITENOISE/>
    <!-- Optional arguments -->
    <UPDATEFREQ> [VALUE] <UPDATEFREQ/>
    <NSTEPS> [VALUE] <NSTEPS/>
</FORCE>
\end{lstlisting}


\section{Coupling}

Nektar++ Solvers can be run in parallel with third party applications and other Nektar++ solvers, where run-time data exchange is enabled by the coupling interface.
The interface is configured in the \inltt{COUPLING} tag as
\begin{lstlisting}[style=XMLStyle] 
<COUPLING TYPE="[type]" NAME="[name]">
    <I PROPERTY="SendSteps"         VALUE="1" />
    <I PROPERTY="SendVariables"     VALUE="u0S,v0S" />
    <I PROPERTY="ReceiveSteps"      VALUE="1" />
    <I PROPERTY="ReceiveVariables"  VALUE="u0R,v0R" />
    ...
</COUPLING>
\end{lstlisting}
The coupling type can be any of the following:
\begin{itemize}
    \item "File"
    \item "Cwipi" 
\end{itemize}
while the name can be chosen arbitrarily.
Inside each coupling block, the send and receive frequencies are defined by the \inltt{SendSteps}  and \inltt{ReceiveSteps} parameters, respectively.
Which variables are to be sent or received is specified by the \inltt{SendVariables} and \inltt{ReceiveVariables}.
By default, the send and receive frequencies is set to zero, which disables the corresponding exchange in this coupling.
An empty \inltt{SendVariables} or \inltt{ReceiveVariables} list has the same effect.

\begin{center}
    \begin{tabularx}{0.99\textwidth}{lllX}
        \toprule
        \textbf{Option name} & \textbf{Required} & \textbf{Default} & 
        \textbf{Description} \\
        \midrule
        \inltt{SendSteps}      & \xmark   & \texttt{0} &
            Frequency (in steps) at which fields are sent. Sending is disabled if set to zero.\\
        \inltt{SendVariables}      & \xmark   & \texttt{<empty>} &
            Comma-separated list of sent variables. Sending is disabled if the list is empty.\\
        \inltt{ReceiveSteps}      & \xmark   & \texttt{0} &
            Frequency (in steps) at which fields are received. Receiving is disabled if set to zero.\\
        \inltt{ReceiveVariables}      & \xmark   & \texttt{<empty>} &
            Comma-separated list of received variables. Receiving is disabled if the list is empty.\\
        \bottomrule
    \end{tabularx}
\end{center}



\subsection{File}
This coupling type allows the user to exchange fields at run time by reading from and writing to files.
Besides the basic parameters which define the exchanged variables and the exchange frequency, the file coupling type requires the \inltt{SendFileName} and \inltt{ReceiveFunction} parameters to be set.
The Coupling name is not used for this type and can be ignored.
\begin{lstlisting}[style=XMLStyle] 
<COUPLING NAME="coupling1" TYPE="File">
    <I PROPERTY="SendSteps"         VALUE="1" />
    <I PROPERTY="SendVariables"     VALUE="u0S,v0S" />
    <I PROPERTY="SendFileName"      VALUE="Dummy0out_%14.8E.pts" />
    <I PROPERTY="ReceiveSteps"      VALUE="1" />
    <I PROPERTY="ReceiveVariables"  VALUE="u0R,v0R" />
    <I PROPERTY="ReceiveFunction"   VALUE="CouplingIn" />
</COUPLING>
\end{lstlisting}
\inltt{SendFileName} specifies a file name template to write the field data to. 
Currently, only \inltt{.pts} files are supported and the file is only created once fully written, avoiding race conditions between sender and receiver.
Receiving is implemented by evaluating a session function specified in the \inltt{ReceiveFunction} parameter.
The coupling waits for the file given in the receive function to appear.

\begin{center}
    \begin{tabularx}{0.99\textwidth}{lllX}
        \toprule
        \textbf{Option name} & \textbf{Required} & \textbf{Default} & 
        \textbf{Description} \\
        \midrule
        \inltt{SendFileName}      & (\cmark)   & - &
        File name where the sent fields should be written to. Required if sending is enabled. Time dependent file names are supported.\\
        \inltt{ReceiveFunction}      & (\cmark)   & - &
        Function to evaluate to obtain the received fields.Required if receiving is enabled.\\
        \bottomrule
    \end{tabularx}
\end{center}


\subsection{Cwipi}

\begin{notebox}
    The Cwipi coupling is only available when Nektar++ is compiled with OpenMPI and CWIPI
\end{notebox}
The Cwipi coupling uses CWIPI\footnote{http://sites.onera.fr/cwipi/} to facilitate real time data exchange and linear interpolation over MPI.
All data transfers are non-blocking to minimize the computational overhead.
The interface must be enabled with the command line option \inltt{--cwipi} and a unique application name, e.g:\begin{lstlisting}[style=BashInputStyle] 
DummySolver --cwipi 'Dummy1' Dummy_3DCubeCwipi_1.xml
\end{lstlisting}
CWIPI uses the current applications and the name of the coupling to identify two peers in cosimulation setups.
The name of the remote application must be provided by the \inltt{RemoteName} parameter.
Unlike the File-type coupling, a linear interpolation in time is applied to the received fields if non-unity values are set for \inltt{ReceiveSteps}.
\begin{lstlisting}[style=XMLStyle] 
<COUPLING NAME="coupling1" TYPE="Cwipi">
    <I PROPERTY="RemoteName"        VALUE="Dummy1" />
    <I PROPERTY="SendSteps"         VALUE="1" />
    <I PROPERTY="SendVariables"     VALUE="u0S,v0S" />
    <I PROPERTY="SendMethod"        VALUE="NearestNeighbour" />
    <I PROPERTY="ReceiveSteps"      VALUE="1" />
    <I PROPERTY="ReceiveVariables"  VALUE="u0R,v0R" />
    <I PROPERTY="Oversample"        VALUE="5" />
    <I PROPERTY="FilterWidth"       VALUE="10E-3" />
    <I PROPERTY="NotLocMethod"      VALUE="Extrapolate" />
</COUPLING>
\end{lstlisting}

Additional options which define the coupling include \inltt{SendMethod}, the method used to retrieve the physical values at the locations requested by the remote application.
The linear interpolation of CWIPI is not used for sending fields, instead the interpolation is implemented in Nektar++ directly.
Available options are \inltt{NearestNeighbour}, \inltt{Shepard} and \inltt{Evaluate}. 
The nearest neighbor interpolation gives the best performance at the cost of accuracy. 
The last option directly evaluates the expansions using a backward transform, giving superior accuracy while being the most computationally expensive. 
The Shepard interpolation can be a good compromise between computational efficiency and accuracy.

When using non-conforming domains, the current application might request values outside of the remote applications computational domain.
How to handle these not-located points is specified by the \inltt{NotLocMethod} parameter.
When set to \inltt{keep}, the point value is not altered.
With \inltt{Extrapolate}, the nearest neighbor value of the current application is used.
Note that this can be very inefficient when using many MPI ranks.


\begin{center}
    \begin{tabularx}{0.99\textwidth}{lllX}
        \toprule
        \textbf{Option name} & \textbf{Required} & \textbf{Default} & 
        \textbf{Description} \\
        \midrule
        \inltt{RemoteName}      & \cmark   & - &
            Name of the remote application.\\
        \inltt{SendMethod}      & \xmark   & \inltt{NearestNeighbour} &
            Specifies how to evaluate fields before sending. Available options are \inltt{NearestNeighbour}, \inltt{Shepard} and \inltt{Evaluate}.\\
        \inltt{Oversample}      & \xmark   & 0 &
            Receive fields at a higher (or lower) number of quadrature points before filtering.\\
        \inltt{FilterWidth}      & \xmark   & 0 &
            Apply a spatial filter of a given filter width to the received fields. Disabled when set to zero.\\
        \inltt{NotLocMethod}      & \xmark   & \inltt{keep} &
            Specifies how not located points in non-conformal domains are handled. Possible values are \inltt{keep} and \inltt{Extrapolate}.\\
        \bottomrule
    \end{tabularx}
\end{center}


\input{xml/xml-analytic-expressions.tex}
