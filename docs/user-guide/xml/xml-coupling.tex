\section{Coupling}

Nektar++ Solvers can be run in parallel with third party applications and other Nektar++ solvers, where run-time data exchange is enabled by the coupling interface.
The interface is configured in the \inltt{COUPLING} tag as
\begin{lstlisting}[style=XMLStyle] 
<COUPLING TYPE="[type]" NAME="[name]">
    <I PROPERTY="SendSteps"         VALUE="1" />
    <I PROPERTY="SendVariables"     VALUE="u0S,v0S" />
    <I PROPERTY="ReceiveSteps"      VALUE="1" />
    <I PROPERTY="ReceiveVariables"  VALUE="u0R,v0R" />
    ...
</COUPLING>
\end{lstlisting}
The coupling type can be any of the following:
\begin{itemize}
    \item "File"
    \item "Cwipi" 
\end{itemize}
while the name can be chosen arbitrarily.
Inside each coupling block, the send and receive frequencies are defined by the \inltt{SendSteps}  and \inltt{ReceiveSteps} parameters, respectively.
Which variables are to be sent or received is specified by the \inltt{SendVariables} and \inltt{ReceiveVariables}.
By default, the send and receive frequencies is set to zero, which disables the corresponding exchange in this coupling.
An empty \inltt{SendVariables} or \inltt{ReceiveVariables} list has the same effect.

\begin{center}
    \begin{tabularx}{0.99\textwidth}{lllX}
        \toprule
        \textbf{Option name} & \textbf{Required} & \textbf{Default} & 
        \textbf{Description} \\
        \midrule
        \inltt{SendSteps}      & \xmark   & \texttt{0} &
            Frequency (in steps) at which fields are sent. Sending is disabled if set to zero.\\
        \inltt{SendVariables}      & \xmark   & \texttt{<empty>} &
            Comma-separated list of sent variables. Sending is disabled if the list is empty.\\
        \inltt{ReceiveSteps}      & \xmark   & \texttt{0} &
            Frequency (in steps) at which fields are received. Receiving is disabled if set to zero.\\
        \inltt{ReceiveVariables}      & \xmark   & \texttt{<empty>} &
            Comma-separated list of received variables. Receiving is disabled if the list is empty.\\
        \bottomrule
    \end{tabularx}
\end{center}



\subsection{File}
This coupling type allows the user to exchange fields at run time by reading from and writing to files.
Besides the basic parameters which define the exchanged variables and the exchange frequency, the file coupling type requires the \inltt{SendFileName} and \inltt{ReceiveFunction} parameters to be set.
The Coupling name is not used for this type and can be ignored.
\begin{lstlisting}[style=XMLStyle] 
<COUPLING NAME="coupling1" TYPE="File">
    <I PROPERTY="SendSteps"         VALUE="1" />
    <I PROPERTY="SendVariables"     VALUE="u0S,v0S" />
    <I PROPERTY="SendFileName"      VALUE="Dummy0out_%14.8E.pts" />
    <I PROPERTY="ReceiveSteps"      VALUE="1" />
    <I PROPERTY="ReceiveVariables"  VALUE="u0R,v0R" />
    <I PROPERTY="ReceiveFunction"   VALUE="CouplingIn" />
</COUPLING>
\end{lstlisting}
\inltt{SendFileName} specifies a file name template to write the field data to. 
Currently, only \inltt{.pts} files are supported and the file is only created once fully written, avoiding race conditions between sender and receiver.
Receiving is implemented by evaluating a session function specified in the \inltt{ReceiveFunction} parameter.
The coupling waits for the file given in the receive function to appear.

\begin{center}
    \begin{tabularx}{0.99\textwidth}{lllX}
        \toprule
        \textbf{Option name} & \textbf{Required} & \textbf{Default} & 
        \textbf{Description} \\
        \midrule
        \inltt{SendFileName}      & (\cmark)   & - &
        File name where the sent fields should be written to. Required if sending is enabled. Time dependent file names are supported.\\
        \inltt{ReceiveFunction}      & (\cmark)   & - &
        Function to evaluate to obtain the received fields.Required if receiving is enabled.\\
        \bottomrule
    \end{tabularx}
\end{center}


\subsection{Cwipi}

\begin{notebox}
    The Cwipi coupling is only available when Nektar++ is compiled with OpenMPI and CWIPI
\end{notebox}
The Cwipi coupling uses CWIPI\footnote{http://sites.onera.fr/cwipi/} to facilitate real time data exchange over MPI.
See \cite{La18} for details.
All data transfers are non-blocking to minimize the computational overhead.
The interface must be enabled with the command line option \inltt{--cwipi} and a unique application name, e.g:\begin{lstlisting}[style=BashInputStyle] 
DummySolver --cwipi 'Dummy1' Dummy_3DCubeCwipi_1.xml
\end{lstlisting}
CWIPI uses the names of the current application and the coupling to identify two peers in cosimulation setups.
The name of the remote application must be provided by the \inltt{RemoteName} parameter.
Unlike the File-type coupling, a linear interpolation in time is applied to the received fields if non-unity values are set for \inltt{ReceiveSteps}.
\begin{lstlisting}[style=XMLStyle] 
<COUPLING NAME="coupling1" TYPE="Cwipi">
    <I PROPERTY="RemoteName"        VALUE="Dummy1" />
    <I PROPERTY="SendSteps"         VALUE="1" />
    <I PROPERTY="SendVariables"     VALUE="u0S,v0S" />
    <I PROPERTY="SendMethod"        VALUE="NearestNeighbour" />
    <I PROPERTY="ReceiveSteps"      VALUE="1" />
    <I PROPERTY="ReceiveVariables"  VALUE="u0R,v0R" />
    <I PROPERTY="Oversample"        VALUE="5" />
    <I PROPERTY="FilterWidth"       VALUE="10E-3" />
    <I PROPERTY="NotLocMethod"      VALUE="Extrapolate" />
</COUPLING>
\end{lstlisting}

Additional options which define the coupling include \inltt{SendMethod}, the method used to retrieve the physical values at the locations requested by the remote application.
Available options are \inltt{NearestNeighbour}, \inltt{Shepard} and \inltt{Evaluate}. 
The nearest neighbor interpolation gives the best performance at the cost of accuracy. 
The last option directly evaluates the expansions using a backward transform, giving superior accuracy while being the most computationally expensive. 
The Shepard interpolation can be a good compromise between computational efficiency and accuracy.

When using non-conforming domains, the current application might request values outside of the computational domain of the remote application.
How to handle these not-located points is specified by the \inltt{NotLocMethod} parameter.
When set to \inltt{keep}, the point value is not altered.
With \inltt{Extrapolate}, the nearest neighbor value of the current application is used.
Note that this can be very inefficient when using many MPI ranks.


\begin{center}
    \begin{tabularx}{0.99\textwidth}{lllX}
        \toprule
        \textbf{Option name} & \textbf{Required} & \textbf{Default} & 
        \textbf{Description} \\
        \midrule
        \inltt{RemoteName}      & \cmark   & - &
            Name of the remote application.\\
        \inltt{SendMethod}      & \xmark   & \inltt{NearestNeighbour} &
            Specifies how to evaluate fields before sending. Available options are \inltt{NearestNeighbour}, \inltt{Shepard} and \inltt{Evaluate}.\\
        \inltt{Oversample}      & \xmark   & 0 &
            Receive fields at a higher (or lower) number of quadrature points before filtering to avoid aliasing.\\
        \inltt{FilterWidth}      & \xmark   & 0 &
            Apply a spatial filter of a given filter width to the received fields. Disabled when set to zero.\\
        \inltt{NotLocMethod}      & \xmark   & \inltt{keep} &
            Specifies how not located points in non-conformal domains are handled. Possible values are \inltt{keep} and \inltt{Extrapolate}.\\
        \bottomrule
    \end{tabularx}
\end{center}
