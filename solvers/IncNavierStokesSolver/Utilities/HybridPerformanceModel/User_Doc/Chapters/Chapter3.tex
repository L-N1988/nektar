% Chapter 3 

\chapter{\code{serial.py}} % Main chapter title

\label{Chapter3} % For referencing the chapter elsewhere, use \ref{Chapter1} 

%----------------------------------------------------------------------------------------
Here we detail the functions of \code{serial.py}.

\section{\code{Serial Computation}}
This function is to predict the length of an average time step used when we already know our fitting constant. 

It takes the following inputs:

\begin{itemize}
\item \code{P}: The degree of the polynomial.
\item \code{Num\_Elements}: The total number of elements.
\item \code{N\_P}: The CG counts for the pressure.
\item \code{N\_V\_1}: The CG counts for velocity 1.
\item \code{N\_V\_2}: The CG counts for velocity 2.
\item \code{N\_V\_3}: The CG counts for velocity 3.
\item \code{Num\_Constants}: The number of constants that were used in the model calibration.
\item \code{constants}: A vector containing the calibration constants.
\item \code{Scheme}: A string containing the method name. (e.g. \code{IterativeFull})

\end{itemize}

It returns the following outputs:

\begin{itemize}
\item \code{Time}: The predicted length of an average time step.
\end{itemize}

It sends the following to the \code{stdout}:

\begin{itemize}
\item None
\end{itemize}

Additional Output:
\begin{itemize}
\item None
\end{itemize}

\section{\code{Operation\_Count}}
This function is to predict the length of an average time step used when we already know our fitting constant. 

It takes the following inputs:

\begin{itemize}
\item \code{P}: The degree of the polynomial.
\item \code{Num\_Elements}: The total number of elements.
\item \code{N\_P}: The CG counts for the pressure.
\item \code{N\_V\_1}: The CG counts for velocity 1.
\item \code{N\_V\_2}: The CG counts for velocity 2.
\item \code{N\_V\_3}: The CG counts for velocity 3.
\item \code{Scheme}: A string containing the method name. (e.g. \code{IterativeFull})

\end{itemize}

It returns the following outputs:

\begin{itemize}
\item \code{T\_A}: The operation count for the advection term.
\item \code{T\_E}: The operation count for the elliptic term.
\end{itemize}

It sends the following to the \code{stdout}:

\begin{itemize}
\item None
\end{itemize}

Additional Output:
\begin{itemize}
\item None
\end{itemize}

\section{\code{compare\_data}}
This is used in the calibration step. This function finds the L2 norm of the difference between the data and the model. 

It takes the following inputs:

\begin{itemize}
\item \code{constants}: A vector containing the calibration constants.
\item \code{Num\_Constants}: The number of constants that were used in the model calibration.
\item \code{Data}: A list containing the user's data of the average time step.
\item \code{T\_A}: The operation count for the advection term.
\item \code{T\_E}: The operation count for the elliptic term.

\end{itemize}

It returns the following outputs:

\begin{itemize}
\item \code{L\_2\_norm}: The L2 norm of the difference between the model and the data.
\end{itemize}

It sends the following to the \code{stdout}:

\begin{itemize}
\item None
\end{itemize}

Additional Output:
\begin{itemize}
\item None
\end{itemize}

\section{\code{Fit\_Model}}
This function runs the calibration fitting of the model. It uses Python's \code{fmin} function to minimise the difference between the model and the data. 

It takes the following inputs:

\begin{itemize}
\item \code{Num\_Constants}: The number of constants that were used in the model calibration.
\item \code{Data}: A list containing the user's data of the average time step.
\item \code{T\_A}: The operation count for the advection term.
\item \code{T\_E}: The operation count for the elliptic term.

\end{itemize}

It returns the following outputs:

\begin{itemize}
\item \code{Fit}: The values for the constants in the calibration, measured in FLOPs.
\end{itemize}

It sends the following to the \code{stdout}:

\begin{itemize}
\item None
\end{itemize}

Additional Output:
\begin{itemize}
\item None
\end{itemize}

\section{\code{Run\_Serial\_Fit}}
This function runs the entirety of the serial model calibration. It produces plots in addition to providing the user with the estimated FLOPs of their system.

It takes the following inputs:

\begin{itemize}
\item \code{Compare\_Serial}: Boolean true/false if the user would like to compare the model prediction to their data.
\item \code{Consider\_Modes}: The values of \(N_Z\) the user would like to use in order to calibrate the model. 
\item \code{P}: The degree of the polynomial.
\item \code{Num\_Elements}: The total number of elements.
\item \code{Nektar\_Modes}: The values of \(N_Z\) in the data that the model will be compared to.
\item \code{Timings}: The average time step values for Nektar\_Modes.
\item \code{Pressure}: Dictionary containing the CG iteration counts for the pressure.
\item \code{Velocity\_1}: Dictionary containing the CG iteration counts for the velocity 1.
\item \code{Velocity\_2}: Dictionary containing the CG iteration counts for the velocity 2.
\item \code{Velocity\_3}: Dictionary containing the CG iteration counts for the velocity 3.
\item \code{Scheme}: A string containing the method name. (e.g. \code{IterativeFull})

\end{itemize}

It returns the following outputs:

\begin{itemize}
\item \code{Fit}: The values for the constants in the calibration, measured in FLOPs.
\end{itemize}

It sends the following to the \code{stdout}:

\begin{itemize}
\item The results of \code{fmin}
\item The values for the fitted constants in FLOPs.
\item The mean of the difference between the data and the model.
\item The standard deviation of the differences.
\item The variance of the difference.
\end{itemize}

Additional Output:
\begin{itemize}
\item Plot of data vs. model in \code{Output/Figures/Model\_vs\_Data.png}
\end{itemize}
